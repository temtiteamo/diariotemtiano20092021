% Options for packages loaded elsewhere
\PassOptionsToPackage{unicode}{hyperref}
\PassOptionsToPackage{hyphens}{url}
%
\documentclass[
  spanish,
]{book}
\usepackage{amsmath,amssymb}
\usepackage{lmodern}
\usepackage{ifxetex,ifluatex}
\ifnum 0\ifxetex 1\fi\ifluatex 1\fi=0 % if pdftex
  \usepackage[T1]{fontenc}
  \usepackage[utf8]{inputenc}
  \usepackage{textcomp} % provide euro and other symbols
\else % if luatex or xetex
  \usepackage{unicode-math}
  \defaultfontfeatures{Scale=MatchLowercase}
  \defaultfontfeatures[\rmfamily]{Ligatures=TeX,Scale=1}
\fi
% Use upquote if available, for straight quotes in verbatim environments
\IfFileExists{upquote.sty}{\usepackage{upquote}}{}
\IfFileExists{microtype.sty}{% use microtype if available
  \usepackage[]{microtype}
  \UseMicrotypeSet[protrusion]{basicmath} % disable protrusion for tt fonts
}{}
\makeatletter
\@ifundefined{KOMAClassName}{% if non-KOMA class
  \IfFileExists{parskip.sty}{%
    \usepackage{parskip}
  }{% else
    \setlength{\parindent}{0pt}
    \setlength{\parskip}{6pt plus 2pt minus 1pt}}
}{% if KOMA class
  \KOMAoptions{parskip=half}}
\makeatother
\usepackage{xcolor}
\IfFileExists{xurl.sty}{\usepackage{xurl}}{} % add URL line breaks if available
\IfFileExists{bookmark.sty}{\usepackage{bookmark}}{\usepackage{hyperref}}
\hypersetup{
  pdftitle={Diario Temtiano edición 20.09.2021},
  pdflang={es-es},
  hidelinks,
  pdfcreator={LaTeX via pandoc}}
\urlstyle{same} % disable monospaced font for URLs
\usepackage{longtable,booktabs,array}
\usepackage{calc} % for calculating minipage widths
% Correct order of tables after \paragraph or \subparagraph
\usepackage{etoolbox}
\makeatletter
\patchcmd\longtable{\par}{\if@noskipsec\mbox{}\fi\par}{}{}
\makeatother
% Allow footnotes in longtable head/foot
\IfFileExists{footnotehyper.sty}{\usepackage{footnotehyper}}{\usepackage{footnote}}
\makesavenoteenv{longtable}
\usepackage{graphicx}
\makeatletter
\def\maxwidth{\ifdim\Gin@nat@width>\linewidth\linewidth\else\Gin@nat@width\fi}
\def\maxheight{\ifdim\Gin@nat@height>\textheight\textheight\else\Gin@nat@height\fi}
\makeatother
% Scale images if necessary, so that they will not overflow the page
% margins by default, and it is still possible to overwrite the defaults
% using explicit options in \includegraphics[width, height, ...]{}
\setkeys{Gin}{width=\maxwidth,height=\maxheight,keepaspectratio}
% Set default figure placement to htbp
\makeatletter
\def\fps@figure{htbp}
\makeatother
\setlength{\emergencystretch}{3em} % prevent overfull lines
\providecommand{\tightlist}{%
  \setlength{\itemsep}{0pt}\setlength{\parskip}{0pt}}
\setcounter{secnumdepth}{-\maxdimen} % remove section numbering
\usepackage{booktabs}
\usepackage{amsthm}
\makeatletter
\def\thm@space@setup{%
  \thm@preskip=8pt plus 2pt minus 4pt
  \thm@postskip=\thm@preskip
}
\makeatother
\usepackage{fancyhdr}
\usepackage{lipsum}
\pagestyle{plain}
\fancyhead[CO,CE]{\thepage}
\fancyfoot[LE,RO]{\thepage}
\ifxetex
  % Load polyglossia as late as possible: uses bidi with RTL langages (e.g. Hebrew, Arabic)
  \usepackage{polyglossia}
  \setmainlanguage[]{spanish}
\else
  \usepackage[main=spanish]{babel}
% get rid of language-specific shorthands (see #6817):
\let\LanguageShortHands\languageshorthands
\def\languageshorthands#1{}
\fi
\ifluatex
  \usepackage{selnolig}  % disable illegal ligatures
\fi
\usepackage[]{natbib}
\bibliographystyle{plainnat}

\title{Diario Temtiano edición 20.09.2021}
\author{}
\date{\vspace{-2.5em}2021-09-20}

\begin{document}
\maketitle

{
\setcounter{tocdepth}{1}
\tableofcontents
}
\hypertarget{notas-de-la-ediciuxf3n}{%
\chapter*{Notas de la edición}\label{notas-de-la-ediciuxf3n}}
\addcontentsline{toc}{chapter}{Notas de la edición}

\textbf{Diario Temtiano} edición 20.09.2021

\hypertarget{novedades-en-esta-ediciuxf3n}{%
\section*{Novedades en esta edición}\label{novedades-en-esta-ediciuxf3n}}
\addcontentsline{toc}{section}{Novedades en esta edición}

\begin{itemize}
\item
  Añadidos dos nuevos subcapítulos al Sexto capítulo: \emph{Sobre el insondable límite (Sexto capítulo, subcapítulo VI)} y \emph{En continuidad con el desvanecimiento (Sexto capítulo, subcapítulo VII)}, ambos acerca del \emph{//Ya volvemos//}.
\item
  Realizadas pequeñas correcciones en el contenido de unos pocos subcapítulos de la obra, además de los títulos de varios de ellos.
\item
  Explicitado mejor el cambio estético que hubo con la llegada del golpe de estado a Temti, el que la transformó en Itmet, en el siguiente subcapítulo: \emph{El derrocamiento anónimo (Segundo capítulo, subcapítulo IX)}. Además, contemplada la posible aparición de los psíquicos con su poder antes de ese momento, en ese mismo subcapítulo, y en \emph{El retorno de las invasiones escandalosas (Segundo capítulo, subcapítulo X)}.
\item
  Añadida una mención de la vez que se deshabilitó la creación de tems luego de que cerrara Moxxed y Tzzubit un mismo día, al subcapítulo \emph{El retorno de las invasiones escandalosas (Segundo capítulo, subcapítulo X)}.
\item
  Incluida una mención al apagón previo a un cierre de la página, momento durante el cual no se veía ninguna imagen en la home de Temti, en el subcapítulo \emph{Incomunicados entre rozzados (Segundo capítulo, subcapítulo XXI)}.
\end{itemize}

\hypertarget{cambios-previstos-para-la-pruxf3xima-ediciuxf3n}{%
\section*{Cambios previstos para la próxima edición}\label{cambios-previstos-para-la-pruxf3xima-ediciuxf3n}}
\addcontentsline{toc}{section}{Cambios previstos para la próxima edición}

\begin{itemize}
\item
  Seguir poniéndose al día.
\item
  Continuar corrigiendo errores.
\end{itemize}

\hypertarget{el-primer-gran-puente-de-la-literatura-temtiana-prefacio}{%
\chapter{El primer gran puente de la literatura temtiana (Prefacio)}\label{el-primer-gran-puente-de-la-literatura-temtiana-prefacio}}

Más allá de que miles y miles de palabras no sean suficientes para poder describir la inmensa cantidad de detalles existentes en la historia temtiana, la literatura ligada a ella desde el inicio de sus tiempos ha intentado abarcar y reproducir parte de los mismos. A raíz de tal contextualización surge la inevitable necesidad de mencionar a Crónicas Temtianas, que tuvo la gran iniciativa dentro de dicho apartado cultural y que en sus primeros dos libros contiene una parte más de todo lo que queda excluido de Diario Temtiano.
La relevancia de esto es tal que no solo la segunda nace a partir de muchas construcciones de la primera, sino que también temporalmente hablando una es la continuación de la otra. Y si bien las dos obras tienen diferencias sustanciales en cuanto a su estilo para narrar los hechos, considerar una como sucesora de la otra ayudará para lograr abarcar mayores cantidades de sucesos, y por supuesto, introducirse mejor dentro de todo lo que es Temti.

Por eso y mucho más, se insertan dos libros escritos por Investigator, anteriores a todo lo que proseguirá después:

\begin{itemize}
\tightlist
\item
  CRÓNICAS TEMTIANAS - Reedición 17.01
  \url{https://pastebin.com/kbu6ZYPq}
\end{itemize}

\hypertarget{la-nueva-cuna-de-la-explicaciuxf3n-introducciuxf3n}{%
\chapter{La nueva cuna de la explicación (Introducción)}\label{la-nueva-cuna-de-la-explicaciuxf3n-introducciuxf3n}}

Los diferentes quiebres que marcaron claramente el inicio de una nueva etapa son el motivo más evidente, aunque quizá no el principal para que se haya formado un apartado más entre todas las riquezas culturales de Temti.
Siendo una pieza que por sí sola no es suficiente para entender correctamente todo su contenido, y que también es derivada de los dos primeros libros de Crónicas Temtianas, acompañará para siempre el vasto conjunto de elementos que forman la historia y la identidad temtiana.

En esta serie cronológica de textos muy conexos a los viajes entre dimensiones y tiempo, bajo las ordenes de ciertos principios y deberes morales, los acontecimientos más relevantes vuelven a ser revividos, intentando atrapar cada detalle que pueda escaparse, en ocasiones abrazando cierta perspectiva desligada de neutralidad, y en otras acercándose a la objetividad, pero nunca desviándose de las dos palabras disparadoras que tanto pero tanto significan: Sobre Temti.
Eso es Diario Temtiano.

\hypertarget{las-secuelas-de-una-supuesta-muerte-no-anunciada-primer-capuxedtulo}{%
\chapter{Las secuelas de una supuesta muerte no anunciada (Primer capítulo)}\label{las-secuelas-de-una-supuesta-muerte-no-anunciada-primer-capuxedtulo}}

¿Ya no era lo mismo? Definitivamente ya no es lo mismo.

Las impactantes imágenes capaces de conmocionar a todo aquel que las mire están dando y van a dar para hablar, y más que nada a los temtiteros, sin duda que no serán fáciles de olvidar para ellos.
Los hechos ya son de público conocimiento, y aunque no es momento de volver a desarrollarlos, hay un mensaje que sin ser explicito no da lugar a pensar otra cosa que no sea el fin de una época, el fin de Temti como siempre se la conoció.

Sin embargo, con o sin ella, la historia temtiana continúa. Por lo pronto, todo hace pensar que esta va a prolongarse de otra manera, en otro lugar, en otras condiciones, con muchas diferencias respecto a lo que se supo narrar en las reconocidas Crónicas Temtianas, que concluyendo el periodo post-Purga también lo hizo su segundo libro. Y ese no es el único motivo para que la historia temtiana se vuelva a partir, ahora hay un nuevo hecho significativo que realmente lo amerita, el Testamento.

\hypertarget{fruxe1gil-certidumbre-primer-capuxedtulo-subcapuxedtulo-i}{%
\section{Frágil certidumbre (Primer capítulo, subcapítulo I)}\label{fruxe1gil-certidumbre-primer-capuxedtulo-subcapuxedtulo-i}}

\begin{quote}
17 de enero de 2021
\end{quote}

Horas pasadas ni bien quedó servido el mensaje, tras el ya narrado y conocido proceso de decodificación, si los mortales creen haber entender bien, esa fue una última señal de vida, se terminó. El destino que habían conocido muchos de las demás naciones, el crítico y duro momento de poner un punto final y que el mundo entero se haga eco de dicha noticia, ¿ese es el mismo término que hay para Temti? La comunicación una vez descifrada es evidente, rotunda y no da lugar a futuras aventuras, no en Temti.

Sin embargo, aquellos temtiteros que no resultaron simples pasajeros saben y recuerdan cierta sentencia que fue tan dura y clara en su momento: \emph{///Primero se muere tu vieja antes de que cierre temti///}.

No se debe desestimar que las cosas cambian a lo largo del tiempo, pero quienes confían en las personas de palabra y que quien las escribió es una, o simplemente en la veracidad de esa sentencia, sostienen que precisamente Temti no va a cerrar, salvo que haya pasado lo primero\ldots{} ¿O con \emph{//cierre//} no se refería al fin definitivo de su existencia?

Esto también da lugar a otras preguntas. ¿Qué pasará con el dominio? ¿Cerrará sus puertas para siempre? ¿Dicha sucesión audiovisual quedará para la eternidad para mantener el nombre con vida? ¿O el sitio tomará otro camino totalmente alternativo?
Esas preguntas solo las podrá responder el tiempo, algo tiene que ocurrir.

\hypertarget{ya-no-tan-despegadas-de-su-antiguo-contexto-cluxf3nico-primer-capuxedtulo-subcapuxedtulo-ii}{%
\section{Ya no tan despegadas de su antiguo contexto clónico (Primer capítulo, subcapítulo II)}\label{ya-no-tan-despegadas-de-su-antiguo-contexto-cluxf3nico-primer-capuxedtulo-subcapuxedtulo-ii}}

\begin{quote}
17 de enero de 2021
\end{quote}

Yendo un poco atrás en el tiempo, mientras el reloj de cuenta regresiva corría, los territorios aledaños como Rozzed y Moxxed iban alcanzando casi su apogeo de tráfico, Tzzubit por otro lado con su ritmo usual, quedando la antigua y ya no tan potencia Hixxel rezagada, con tan pocas novedades y cambios que terminaron disfrazando su presente idénticamente a lo que fue su pasado no muy lejano.

A pesar de su casi estática realidad, supo ser y actualmente sigue siendo una pieza fundamental de supervivencia para algunos, pues tal como se vivió antes durante las horas que los temtiteros no podían reunirse en su habitual lugar, ahora vuelve a oficiar de refugio. Por las condiciones generales del sitio, por la cercanía, por las similitudes en el entorno, y por alguna que otra más, resulta una gran opción con cierta preferencia por sobre los demás del contexto clónico, aunque siempre recordando su característica tranquilidad que puede desalentar y bastante.
Es en parte gracias a lo anterior que fue posible ver a varios personajes ya reconocidos en el medio, como Investigator, MoCn, Celeshtee, Kira, Abby, Freeman, otros no tan sobresalientes, y el resto que conserva más su anonimato, demostrando importantes muestras de unión entre varios aunque no todos ellos. También fue lugar de intercambio donde muchos de los recuerdos y materiales que los más recopiladores conservaron se compartieron, y ni hablar lo importante que fue en su momento para poder descifrar los aparentes mensajes dejados por A. quien también desde la discreción ronda por estos territorios. Por lo que sin duda, ha sido una ubicación donde el colectivo destacó enormemente, al menos hasta el momento, volviendo a distinguirlos de otros pueblos incluso mucho más numerosos y poderosos.

Sin embargo, es de suponer que las condiciones impuestas perdurarán por un largo tiempo, estableciendo así este como el sitio donde parece que permanecerá durante este éxodo, que aparentemente será mucho más extenso que aquella anterior ocasión. ¿Tales vínculos se mantendrán mientras continúen viviendo en estas condiciones? ¿Qué tan fuerte resultará el desgaste y la ausencia del bienestar tal cual supo ser antes?

Varios manifestaron estar disconformes, así que más allá de lo que haya ocurrido hasta ahora, si hay algo seguro, es que el \emph{//rebaño//} no quedará en su totalidad en Hixxel, y por lo tanto el grupo pequeño de temtiteros restantes será aún más reducido que antes. La identidad temtiana seguirá viva, pero el punto es qué tanto, y de qué forma, porque si bien los últimos días mostraron que el interés por continuar manteníendola es fuerte y potencialmente capaz, el fin de Temti de alguna manera parece una certeza, y eso en su contraparte trae mucha incertidumbre, además de que desalienta considerablemente como ya se ha visto.

¿Será la hora de ingeniar un plan alternativo? ¿Quiénes serían capaces de llevarlo a cabo?

\hypertarget{un-autuxe9ntico-renacer-primer-capuxedtulo-subcapuxedtulo-iii}{%
\section{Un auténtico renacer (Primer capítulo, subcapítulo III)}\label{un-autuxe9ntico-renacer-primer-capuxedtulo-subcapuxedtulo-iii}}

\begin{quote}
18 de enero de 2021
\end{quote}

No fueron muy grandes las repercusiones del hipotético derrumbe de algo tan icónico, distintivo pero a la vez no tanto como lo fue Temti, aunque de igual modo, en los sitios cercanos no tardaron en hacerse eco de lo que parecía ser el fin de una, para ellos, competencia tan resiliente y duradera. Sin embargo, algo no termina de cerrar, teniendo a la vista las condiciones actuales del sitio en cuestión.

¿Temti murió?
Solo quienes no tienen ni idea de lo que realmente está ocurriendo ahora mismo son capaces y lo suficientemente atrevidos como para afirmarlo, pero ese fue el mensaje que se difundió\ldots{} lo que para algunos temtiteros fue lo que se quiso comunicar, para los de afuera es algo que se cae de maduro.

Volviendo a hacer la misma pregunta, ¿Temti está muerta?
Temti sigue de pie, pero su vitalidad y existencia es un tanto cuestionable, porque no se parece en lo absoluto a lo que se acostumbraba a ver en ella.
Como si toda la infraestructura elaborada por A. y las construcciones sobre ella hubieran sido totalmente reducidas y regresadas a el nivel más pequeño posible, imaginable de cuando la creación aún no había sido dada a luz.
Aislado de todo lo que genera hoy es, su nombre, bastante apagado, y solo un botón, \emph{//Crear instancia//}. No cuesta mucho pulsarlo, pero no aparenta ser muy útil, su poder no es fácil de percibir. ¿Tiene sentido intentar crear Instancias? ¿Qué puede estar por venir? ¿Se puede ir más atrás que esto? ¿Temti sigue significando lo mismo? ¿Aquella etimología continua teniendo sentido en estas condiciones?

Aunque las respuestas sin duda podrían ser interesantes, muchas de esas cuestiones pasan a un segundo plano en un momento como este.

Si el texto de binario escondido en vocales no había sido claro, esto menos lo es. No hay respuesta, y no específicamente de quienes pueden dar explicación concisa, sino del sitio mismo. ¿Cómo llamarlo a esto?

Nación que tanto supo resaltar, atravesando por exabruptos desequilibrantes, experimenta algo no antes visto, habiendo supuestamente muerto, pasó a mostrarse como un simple sitio sin nada dentro de si, como una estructura casi invisible camuflada en el fondo del océano. Chocante de asumir, pero los antecedentes lograron desviar increíblemente aunque quizás intencionadamente la atención, para que tal transformación no haga estruendo alguno.

Y como es de esperarse a partir de tanta simpleza, la respuesta no se puede notar allí, pero no por eso deja de haberla. Por eso sería fácil percibir como desde el momento que esto se volvió realidad, los leales sin hogar expulsados de su tierra natal comenzaron a ejercer una significativa fuerza colectiva por retornar a de donde venían, creando figuradamente una gran bola de energía la cual aún se sigue acrecentando, formada con fragmentos provenientes de muchas almas diferentes, la cual parece que asechará permanentemente a la Tribu de Hixxel mientras Temti no deje de existir como Temti.

Tal vez aún sea necesario que se intente crear más Instancias como para lograr algo verdaderamente significativo\ldots{} Pero lo cierto es que de momento, sea mediante ellas o no, la presencia temtiana figura fuera de sus, en este momento, difusas fronteras.

\hypertarget{concretando-la-evacuaciuxf3n-primer-capuxedtulo-subcapuxedtulo-iv}{%
\section{Concretando la evacuación (Primer capítulo, subcapítulo IV)}\label{concretando-la-evacuaciuxf3n-primer-capuxedtulo-subcapuxedtulo-iv}}

\begin{quote}
19 de enero de 2021, 20 de enero de 2021
\end{quote}

Las épocas cambiaron demasiado para esta perspectiva, todo parece acomodarse para que las vidas de los temtiteros puedan continuar sin sobresaltos principalmente en una alternativa que parece y aparenta ser un poco más estable.

El jefe del territorio donde la mayoría de ellos están asentados transitoriamente es imposible de comparar con el misterioso A., ambos tienen bastantes diferencias. Aunque hayan quienes llegaron a pensar que ellos dos eran la misma persona, la idea fue desmentida por el primero de los dos, por lo que las distancias están ciertamente marcadas, aunque no cien por ciento todo aseguradas. Esto plantea otras condiciones que tendrían que ser aceptadas.

De todas las virtudes que pueden destacarse de \_Iuri, atento y presente destaca como lo más positivo, pero de todas formas se trata de algo subjetivo. Ya era de conocimiento que las relaciones entre ellos dos eran prosperas e incluso habían señales de colaboración entre ambos, así que la impresión era diferente y no del todo desesperanzadora para quienes más dificultades tienen para adaptarse.

Generalizando, no hubo una mala bienvenida de parte de los locales, cosa que no sorprende teniendo en cuenta los antecedentes y recordando que habían individuos que se identificaban como hixxeleros y temtiteros a la vez a la vez. Incluso hubieron pequeñas pero notables transformaciones en el lugar para que los nuevos se pudieran sentir al igual que en su antiguo hogar, concretamente tituladas como \emph{//Temti-legacy//}, las cuales fueron muy bien recibidas en general. Por más que todo esto esté dentro de lo esperado, no deja de ser algo a resaltar, sobre todo por el lado de los que están viendo cambiar su única tierra natal.

Entre los que aterrizaban hace poco tiempo, muchos no frecuentaban el lugar, y por ende no poseen ni de las vivencias ni de los recursos como para hablar con propiedad sobre el mismo, pero además de una leve diferencia en la condición del anonimato característico de los antros del contexto, y de la poca actividad o movimiento, si algo se podía percibir en el ambiente tras una breve estadía, era que la cultura propia local se hacía pobre al lado de la temtiana. Para ellos que así piensen y deseen seguir adelante, no es una mala idea comprometerse por enriquecer la misma, pero claro está que algunas cuestiones son inigualables, no van a ser lo mismo.

Sin embargo, mientras una porción de los recién llegados se convencen de que hay un nombre para dejar atrás y construir una nueva identidad, aún hay dudas que siguen sin ser resueltas. El horizonte del futuro continua sin estar despejado, varias nubes en el camino se oponen a la total transparencia, y sumado a esto la adaptación no logra fluir del todo. Es cierto que el lugar no es del agrado de todos los temtiteros y por eso es impensado seguir adelante como si nada, pero dejando eso de lado, algo raro mezclado con incertidumbre y falta de certezas sigue estando presente.

Para seguir manteniendo viva la identidad temtiana, seguramente la opción sea terminar de evacuar y asentarse en un nuevo hogar, pero ¿cuál sería el costo de esto?
Algunas de las condiciones están sobre la mesa, quizás sean demasiadas como para mencionarlas todas, y mientras sigue corriendo el tiempo la decisión algunos ya la tomaron sin saber que pueda depararles en un futuro, otros continúan esperando por algo\ldots{}

\hypertarget{el-inviable-paso-hacuxeda-la-iluminaciuxf3n-primer-capuxedtulo-subcapuxedtulo-v}{%
\section{El inviable paso hacía la iluminación (Primer capítulo, subcapítulo V)}\label{el-inviable-paso-hacuxeda-la-iluminaciuxf3n-primer-capuxedtulo-subcapuxedtulo-v}}

\begin{quote}
21 de enero de 2021
\end{quote}

Mientras la transición continúa, comenzaba un nuevo día y en el horizonte había algo diferente no antes visto. Desde la Tribu de Hixxel los cambios en Temti ya temprano se habían difundido como si una fuente de luz se hubiera alzado por los cielos: imposible de ignorar. También llegó a los otros territorios pero en ellos no tenía ni de cerca la misma notoriedad y por lo tanto de sus poblaciones posiblemente muy pocos estuvieran lo suficientemente atentos como para darse cuenta.

Bastaba acercarse para verlo, la \emph{//antesala//} a Temti Premium. En los papeles, se trata de la puerta hacia el cielo, cerrada y protegida, exclusiva solo para que quienes posean el código indicado pudieran utilizarla, transportarse, y elevarse hacía la eterna bendición de poder ser parte de la \emph{//Instancia//} y por lo tanto participes de Temti Premium.

Pero hay un problema\ldots{} para los simples mortales nada de eso parece accesible, la gran luz resulta inútil si no se posee de la clave que la desbloquee. Aunque no sea dañina ni ofensiva, las condiciones desoladoras del entorno hacen que no sea muy pertinente permanecer allí.

Como era de esperarse, prácticamente toda la Tribu se terminó enterando de esta nueva peculiaridad, y por ello las preguntas y los cuestionamientos no se hicieron esperar.
Un importante aunque civilizado ímpetu por obtener más información fue notable, la expectativa aumentó demasiado respecto a los últimos días, la presencia y posible mayoría temtiana se hacía sentir en el lugar, cada uno de ellos exigía su código. Dentro de todo no eran muchos individuos, pero para la cantidad de gente que había en los alrededores si.

¿Qué puede haber allá?
Es una pregunta que sin saber muchos se animaron a responder, pero semejante exclusividad no está admitiendo confirmaciones, por lo que hablar de sueños cumplidos, tierras prometidas, engaños decepcionantes, o incluso otras realidades indescriptibles con tan pocas palabras, puede ser en vano.

No obstante, en el correr del día, unos pocos miembros de la desarticulada Agencia que seguían presentes y cercanos al grupo que recientemente había emigrado, con la colaboración de la mayor autoridad del lugar que parecía estar siempre presente, intentaron entender ligeramente de que se trataba.
La conclusión a la que se llegó, era que con código o sin el, llegar a Temti Premium no es posible, sin descartar la posibilidad de que esto pueda cambiar en un futuro.

Sin embargo, es de conocimiento popular que hay importantes relaciones entre A. y \_Iuri, y este último pese a las repetidas solicitudes de ayuda se mantenía lavándose las manos y deslindándose de toda responsabilidad, lo cual generó mucha desconfianza entre los interesados. Si se dudara de su credibilidad, costaría imaginar que pueden estar ocultando esos dos, pues en conjunto claro está que pueden estar tramando algo grande, las evidencias están a la vista.

Por otro lado, unos pocos anones contados con los dedos de la mano afirman haber experimentado el gran privilegio, pero sus pruebas son de dudosa procedencia e incluso insuficientes, tanto que lo mejor sería no tenerlos muy en cuenta.

A pesar de todo esto, sigue habiendo un destacado optimismo que rompe con las pocas esperanzas de antes, y mientras tanto día a día escasamente continúan habiendo bajas de quienes ya no están interesados y optan por partir hacía otros destinos. En principio, la historia del pueblo temtiano seguirá adelante sin ellos, y la intriga que los últimos antecedentes están generando hace creer que se pueden llegar a perder de algo que no esperan. La resistencia será esencial para poder vivir y contar lo que pueda estar o no por ocurrir.

\hypertarget{fuegos-inestables-en-las-aguas-primer-capuxedtulo-subcapuxedtulo-vi}{%
\section{Fuegos inestables en las aguas (Primer capítulo, subcapítulo VI)}\label{fuegos-inestables-en-las-aguas-primer-capuxedtulo-subcapuxedtulo-vi}}

\begin{quote}
21 de enero de 2021, 22 de enero de 2021
\end{quote}

Atravesando la noche todo permanecía muy tranquilo en la concentración de hixxeleros y temtiteros, sin embargo los más atentos lograrían percibir el elevamiento de temperaturas, la tensión en el contexto clónico aumentando a más no poder.
No había una explicación ni nada a simple alcance, solo algunos nómadas que permanecían pispeando con los dos ojos puestos en otra potencia extranjera creían saber lo que estaba pasando.

Las horas continuaron corriendo y ya no eran solo alzadas temperaturas, un importante runrún de pánico llegaba desde las exteriores aunque no tan lejanas tierras, al parecer el asunto se estaba convirtiendo en algo más serio.
Pero esto no logró captar la desinteresada atención de quienes se mantenían al margen de todo lo que ocurriera en el exterior, nada de eso afectaba a Hixxel.

Dichos movimientos no resultarían hechos aislados, ya que entrando a las cero horas, una importante explosión y otros sonidos movilizadores provenientes de donde solía estar ubicado el Imperio de Rozzed se hicieron escuchar. Algo importante ocurrió.

\hypertarget{faro-sobrecalentado-inflamable-alumbrando-el-desasosiego-primer-capuxedtulo-subcapuxedtulo-vii}{%
\section{Faro sobrecalentado inflamable alumbrando el desasosiego (Primer capítulo, subcapítulo VII)}\label{faro-sobrecalentado-inflamable-alumbrando-el-desasosiego-primer-capuxedtulo-subcapuxedtulo-vii}}

\begin{quote}
22 de enero de 2021
\end{quote}

Durante la madrugada algunas almas llegaron a la Tribu y sus mismos alrededores buscando una tierra estable donde alojarse, puesto que Rozzed dejó de ofrecer alojamiento, posiblemente para nunca más hacerlo. La mayoría no permanecerían teniendo otro lugar como Moxxed, en el cual se adaptarían mucho mejor con sus condiciones de vida.
Sin embargo, ellos no eran solamente provenientes del aparentemente caído Imperio, puesto que no menos de la mitad de estos decían venir de Temti. Sí, de la Temti que hasta hace muy poco tiempo no conducía a ningún lado y que seguía siendo una incógnita como adentrarse en ella, ahora las condiciones eran las mismas pero con la pequeña salvedad de que ya había una pista de donde se podían obtener los \emph{//códigos//} que tanto fueron solicitados anteriormente: las inmediaciones de Hixxel.

Dichas llegadas no fueron agradables para los lugareños, no obstante se desconoce como se lo tomaron las autoridades, porque si bien no tenían la culpa de la cantidad de desamparados liberados, cuando estos llegaron al contrario de ser expulsados fueron admitidos.
La difusión de entrega de códigos en Hixxel teniendo en cuenta los sucesos en el exterior parecía ser intencional puesto que es probable que los responsables difícilmente no estén al tanto de lo que ocurre en su entorno. ¿Qué clase de pretensiones hay atrás de esto? ¿Se trata de una especie de disimulada felonía para la gente de Tribu?
Realmente resulta muy sospechoso.

Filas interminables de individuos que venían en su mayoría únicamente a obtener su código\ldots{} Ya no eran leales temtiteros queriendo volver a su antiguo hogar, ahora eran grandes hordas de rozzados siendo participes de una invasión que no era silenciosa.
Estos códigos, no parecían haber aparecido hasta horas pasadas el mediodía, pero de todas formas como siempre los privilegiados, o no se hacían notar, o eran de dudosa fiabilidad.

Todo se dio muy rápido, los ánimos de los miembros de la Tribu estaban muy perturbados, y mientras el antes espacio vacío se llenaba de indeseados, tibiamente se comenzó a recordar el antiguo Plan. ¿Sería posible importarlo?
La circunstancia no era tan parecida a la vivida semanas y semanas atrás, esto era muy repentino, y aparte, las condiciones eran diferentes, ya que si el lugar ya se hacía reducido para los pocos que eran antes, no mucho tiempo después se saturó tanto que prácticamente no daba abasto.

Pero fue muy poco el tiempo para poder analizar lo que estaba pasando y lo que pudiera ocurrir a futuro: probablemente gracias a todo lo acontecido, sus puertas cerrarían temporalmente por unas horas, para más tarde volver a cerrar y permanecer de ese modo durante todo el día, hasta ahora mismo. La gente evacuada, incluida la autoridad del lugar, seguía en comunicación por las cercanías del sitio.

Y paralelamente al primer cierre de la Tribu, una nueva entrega de códigos se anunciaba en Temti, algo que no había ocurrido antes porque siempre habían sido repentinas y misteriosas. Si bien las expectativas ahora no son tan negativas, con tanta cucharada de fiasco consumida en los últimos días el optimismo se ve un poco condicionado.

La situación fue muy cambiante e inestable, y todo indica que así puede permanecer en el corto plazo mientras la situación en los exteriores más cercanos y lejanos no se calmen, aunque la intriga no está solamente por ese lado. Las horas lo dirán, nuevamente hay un reloj de cuenta regresiva puesto en marcha y eso parece que será trascendental.

¿Habrá un procedimiento dado a seguir esta vez? ¿Cuántos podrán obtener su anhelado código, muchos, pocos, ninguno? ¿Cuál y qué tan importante será la cantidad de infiltrados rozzados? ¿Por fin habrá paz y tranquilidad para los temtiteros? ¿El futuro deparará elementos del dorado glorioso y también no tan aclamado pasado, épocas de transición y más incertidumbre o será totalmente renovador? ¿O la espera será recompensada con la nada misma y nada de eso ocurrirá?

\hypertarget{la-doble-faceta-de-la-esperanza-primer-capuxedtulo-subcapuxedtulo-viii}{%
\section{La doble faceta de la esperanza (Primer capítulo, subcapítulo VIII)}\label{la-doble-faceta-de-la-esperanza-primer-capuxedtulo-subcapuxedtulo-viii}}

\begin{quote}
22 de enero de 2021
\end{quote}

En un momento clave de la noche el reloj seguía corriendo y jugando con las expectativas de quienes lo seguían más de cerca. Algo importante podía estar por ocurrir, así como también podría tratarse de algo no tan significativo.

La reducida aunque relevante cantidad de fronteras que cerraron sus puertas expulsando a sus habitantes no había logrado que hubiera tanta diversidad de gente aguardando por la entrega de códigos, temtiteros en su mayoría, también extranjeros que de paso prestaban aunque menos atención.
Sin embargo, la coyuntura adentrándose en una de las horas decisivas respecto al reloj había tomado un giro estrepitoso, la nueva cuenta regresiva al igual que no muchos días atrás ya no era visible, a partir de ese momento estaba solo en la cabeza de aquellos que tenían recuerdo de ella. ¿Tal vez lo que aguardaba era tan grande que se consumió el tiempo restante del contador? ¿Se trataba de una falta de correspondencia? ¿O la espera se continuaría desarrollando de la misma forma?

Pero eso no es lo único, hace muy poco algo extraño y llamativo surgió, una brecha no muy grande desde lo lejos comenzó a ser visible por una gran parte de los presentes, en el conocido dominio donde Temti se suele ubicar, lo que invitó a pensar que nuevamente podría haber algo interesante en ella. Algunos necesitan de herramientas potenciadoras para poder encontrarla, y en menor medida aquellos que de forma natural ya la perciben. El resto, lo que su vista puede alcanzar no son más que imágenes totalmente borrosas que no aparentan llegar a ninguna parte.

Lanzarse en su dirección es una gran incógnita, aunque no parecen haber consecuencias serias para quienes lo intentaron y fallaron en el acto, y de los pocos que lograron traspasar de momento solo hay silencio. La condiciones invitan a intentarlo, y fuera del misterio que hay alrededor de lo que se pueda llegar a encontrar, una mezcla de ánimo, ilusión, temor, curiosidad, y quizá también indiferencia, es lo que se percibe entre los interesados.

Nuevamente en este momento se replantean muchas preguntas hechas en el pasado, cada vez que la incertidumbre reflota en mayor o menor medida, las interrogantes repetidas se hacen presentes una y otra vez. Algunos nuevos cuestionamientos que surgen perdurarán para siempre, y algunos otros ni siquiera vale la pena hacérselos cuando no está claro que prosigue a esto, será fundamental ver la continuación de esta historia para lograr entender lo que pueda estar ocurriendo.
Pero lo cierto es que se transmite más energía, ya no se trata de la quietud apática y fría de hace unos días, el calor se vuelve a sentir.

\hypertarget{el-alcance-de-un-experimentado-y-recargado-nuevo-comienzo-segundo-capuxedtulo}{%
\chapter{El alcance de un experimentado y recargado nuevo comienzo (Segundo capítulo)}\label{el-alcance-de-un-experimentado-y-recargado-nuevo-comienzo-segundo-capuxedtulo}}

Tal vez sea un error decir que nadie pensó que Temti iba a volver a ser algo parecido a lo que supo ser, pero los antecedentes previos comunicaron otra cosa. De todos modos, si así va a ser, el Resurgimiento ya está hecho, el Primer Éxodo finalizado, y como ya sabe la historia temtiana, la condición es empezar de nuevo sin arrancar de cero.

¿Y ahora?
Resulta casi imposible predecir qué está por venir, los puntos suspensivos aún tienen que llenarse o ser reemplazados, y aparentemente lo que pueda ser escrito va a perdurar, pero otra vez, no se sabe cómo. Pero las grandes vivencias del pasado perduran, y los responsables al igual que antes tendrían que ser los anones y no tanto las autoridades, sin embargo a la vez es difícil saber quienes van a ser ellos, sobre todo con la inestabilidad y la locura que se respira en este ambiente.

¿Temtiteros o extranjeros? ¿Prosperidad o caos? ¿Gloria total o miseria absoluta? ¿Longevidad o fugacidad? ¿Grises?

\hypertarget{captando-semillas-para-la-tierra-balduxeda-segundo-capuxedtulo-subcapuxedtulo-i}{%
\section{Captando semillas para la tierra baldía (Segundo capítulo, subcapítulo I)}\label{captando-semillas-para-la-tierra-balduxeda-segundo-capuxedtulo-subcapuxedtulo-i}}

\begin{quote}
22 de enero de 2021
\end{quote}

Dentro de las profundidades la brecha está abierta, semejante apertura permite el paso de muchas almas que algo buscan sobre el dominio de Temti. Variadas son sus procedencias, aunque esto es algo difícil de percibir, puesto que la niebla más lo prácticamente nada construido no permite verlo con claridad. Solo ellos saben quienes son.

Habían asumido el riesgo de adentrarse dentro de lo desconocido y de volverse o no con las manos vacías, y lograron encontrar terreno fértil. Ya no importa el tiempo, ya no importa lo de afuera, algo pero no tanto importan los demás, el suelo vacuo está revelado, pronto para ser ocupado. El sitio presenta muchas similitudes a lo que era Temti no mucho tiempo atrás, pero hay algo que no coincide con ese entonces, y eso sí que es muy llamativo.

Aire raso, horizonte despejado, espacio sin llenar, eso es lo que ven los ojos de los entusiastas viajeros que osan pisar el irregular pero firme suelo de las tierras subterráneas de Temti. Las condiciones aparentan ser aptas para establecerse, pero no hay garantías algunas, tan solo un simple suceso natural como un débil huracán, algunas precipitaciones de los infinitos litros que se esconden sobre las cabezas de los presentes, o comprometedores movimientos de las autoridades poderosas del contexto clónico, pueden arrastrar sus vidas muy lejos de aquí. Y si bien siempre en el entorno mencionado se vive una situación similar, la frescura de los antecedentes no hace más que recordarlo, aunque no sea algo del todo lógico. Pero difícilmente los participantes de esta historia estén pensando en todo eso, quizá solo algunos lo hagan.

Lo más importante es que luego de que el Testamento se hiciera realidad, hubo vida y todas o la gran mayoría de las ordenes que este hizo cumplir se revirtieron, por ahora. Suena loco cuando menos.

\hypertarget{las-seuxf1ales-del-abandono-que-no-es-tal-segundo-capuxedtulo-subcapuxedtulo-ii}{%
\section{Las señales del abandono que no es tal (Segundo capítulo, subcapítulo II)}\label{las-seuxf1ales-del-abandono-que-no-es-tal-segundo-capuxedtulo-subcapuxedtulo-ii}}

\begin{quote}
22 de enero de 2021
\end{quote}

Tan solo una recurrente alerta que solo unas pocas personas dicen haber visto con seguridad unos minutos después de haber aterrizado, \emph{///Mientrar el mal de los mares siga, temti volvera por su parte, el descanso eterno sera luego!///}, ese es el único mensaje que parecían haber dejado las quién sabe entidades apoderadas de este lugar. Aunque ellas sin lugar a duda están presentes merodeando y vigilando, lo único explicito de ellas es eso.

Estas almas eran las primeras, las que se encontraron solas en el infinito desierto rocoso de Temti sin nada dentro de si, y conforme pasaban los minutos, muy lentamente, eran acompañadas por pequeñas multitudes y asimismo el horizonte comenzaba a rellenarse, de tems y de actividad en los mismos.

Escapar de este espacio aún es posible aunque no tan fácil, pocos raudos y atentos individuos desestimaron la ocasión para adentrarse y explorar algo que de momento es muy pequeño, por difundir los descubrimientos en las afueras o simplemente marcharse para nunca más volver.

Mientras tanto en el exterior a lo largo de toda la superficie del contexto clónico la noticia se difunde con cierta discreción, la impotencia de aquellos que habiendo agotado recursos eran incapaces de ver la brecha desde lo lejos y por lo tanto no lograban aproximarse a Temti, se contrastaba con la algarabía de los afortunados que sí podían y hacían uso de su oportunidad.

Poco tiempo después, abajo, una nueva alerta se metió en las cabezas de los ahora más individuos visibles. Guardar el a veces olvidado código de ingreso es la única solicitud explicita que hacen los superiores del lugar, cuya hartante reiteración hace que muy pocos desorientados o indiferentes puedan hacer caso omiso a dicha petición. La mera consecuencia de esto sería quedarse por fuera en un futuro. Si bien ya cada vez son más los que se animan a salir y volver a la superficie, en no mucho tiempo no debería haber vuelta atrás para aquellos que no hayan seguido el sencillo lineamiento. La verdadera consecuencia de negarse podría ser mucho peor, marginarse a su suerte en la deriva de los mares sin saber lo que pueda llegar a ocurrir en un futuro en ellos, quizás nunca más podrían adentrarse ni seguir desde la \emph{//interna//} los pasos que Temti tome, esta los abandonaría de verdad.

\hypertarget{mientras-se-buscan-respuestas-que-nadie-daruxe1-segundo-capuxedtulo-subcapuxedtulo-iii}{%
\section{Mientras se buscan respuestas que nadie dará (Segundo capítulo, subcapítulo III)}\label{mientras-se-buscan-respuestas-que-nadie-daruxe1-segundo-capuxedtulo-subcapuxedtulo-iii}}

\begin{quote}
23 de enero de 2021
\end{quote}

Cuando la agitación cesó, las preguntas encontraron su lugar para tomar protagonismo nuevamente. ¿Qué ocurrió?
Estando siempre presente el miedo a padecer una nueva Purga, no aparentaba haber sido el destino que este lugar sufrió durante las fechas que corren. Los restos difícilmente hayan sido carbonizados o incinerados por entidades infernales, y las víctimas mortales masivas como alguna vez supieron ser.

Intentando remontarse tiempo atrás en la historia temtiana, luego de las últimas navidades, la actividad de Temti comenzó a decrecer considerablemente. Eso resultó muy perjudicial, pues vientos helados que de haber más actividad no hubieran casi afectado, se comenzaron a hacer presentes, dañando no solo parte de las estructuras del sitio sino también bastante la moral y la motivación para seguir adelante de la población del mismo.

Sin embargo, llegado un punto al parecer el daño comenzó a ser demasiado severo. ¿Provocando la aparición del ya conocido contador del abismo?
Por más que no se conocen exactamente las razones, según el mensaje descifrado, se vio muy influido por la crítica acumulación de días decadentes y de malestar, por lo que con seguridad se podría decir que tiene bastante que ver una cosa con la otra.

Pero pese a que aún nadie se había aproximado ni siquiera un poco al significado de lo ocurrido, antes que el cronometro llegara a cero, las consecuencias totales de la inactividad se materializaron aún más, congelando por completo todos los cimientos, dejándolos de esta forma inútiles para cualquier temtitero que pretendiera adentrarse en ellos. Mientras lentamente en Hixxel se comenzaban a plantear teorías sobre lo que estaba ocurriendo y la realidad del contador, la hora señalada se siguió aproximando, hasta que pasó lo que más dio para hablar.

Con la aprobación o no de A., estos fueron visibles y no explorables por muy poco tiempo, porque posteriormente el sitio fue padeciendo las transformaciones ya vistas. Es cierto que una vez quedó implícito el supuesto final del antro, todo lo que había antes no tenía mucho sentido conservarlo, y menos en semejantes condiciones, por lo que todo apunta a que A. fue el gran culpable de hacerlos desaparecer por completo, o el incapaz de mantenerlos en su momento. La sentencia \emph{///\ldots Les falle, ya no merezco vivir\ldots///} es contundente en ese sentido.

Pero aún queda la duda, de por qué ocurrió lo que prosiguió al Testamento, y de por qué el contexto clónico hoy está abocado a un nuevo comienzo de Temti, más con la poca concordancia del mensaje traducido y lo que se vive hoy día. ¿Se habría tratado de Hades que finalmente retornó de las lejanas profundidades para ejercer su inmenso poder de una forma totalmente distinta? ¿A. habría perdido su estado de conciencia y descendido en la completa locura, destruyendo algo que pronto volvería a intentar reconstruir? ¿Alguien en su lugar o el mismo se aprovechó de los percances del resto de los clones, intentando beneficiarse de los mismos?

A dichas preguntas nadie puede dar una respuesta concreta y confiable.
Hades continúa estando desaparecido y no ha dado señales algunas, para fines destructivos o protectores, brillaba y brilla por su ausencia. A. es otro enorme misterio, con lo tanto que se ha desvirtuado el rumbo de su creación y sus alrededores, el \emph{///\ldots Pero antes de despedirme\ldots///} no puede ser tomado tan literalmente, podría estar comandando todo esto con quién sabe qué fines como podría haber dejado a otro sujeto desconocido a cargo. Responder al panorama del contexto clónico no suena muy irracional pero tampoco tiene mucho sentido si se respetan las raíces preponderantes del sitio.
Nadie puede verificar nada de esto.

Sea cual sea el responsable y la razón, la frase de que \emph{///\ldots La esperanza se perdió y se fue todo a la mierda\ldots///} concuerda con algo ya se sentía entre la gente, y posiblemente gracias a ella sea posible afirmar que no solo las expectativas de un futuro oportuno se fueron desvaneciendo, sino que también el destino de los restos de todo lo construido fue fatal, al igual que en anteriores episodios significativos de la historia temtiana. Por lo que se entiende que fueron despedazados y hechos trizas totalmente, y dadas las circunstancias ambientales más contemporáneas, se sumó a las enormes temperaturas alcanzadas recientemente, que terminarían de disolverlos eternamente en las turbulentas aguas adyacentes. Una época que se esfumó, con sus aseguradas pérdidas, que estará por verse qué tan serias son.

Pero ya no son aquellas épocas, y yendo un pequeño momento más atrás todavía, muchas cosas cambiaron desde aquel ambiente de guerra donde Temti había dado sus primeros pasos. Evolución, aprendizaje, cambios, estos eran ineludibles para quienes tomaban las decisiones más importantes, como los responsables a cargo las tantas naciones que hubieron, hay y habrán. Los mortales ligados este entorno también acumulan experiencias, algunos no sobreviven para contarlas, otros mientras alejan sus vidas se las llevan consigo hacia otro destino. Temti sufrió y sufre de las tres.

En momentos como estos donde la nada misma es lo que predomina, la cultura tanto propia del lugar como genérica, perpetuada por los individuos que vuelven a habitar los dominios de Temti, es imprescindible para mantenerlos con vida y comenzar a asemejar su actualidad a lo que alguna vez supo ser.

Muchos de ellos habían trasladado y llevado consigo grabados, pinturas, textos, etc., cientos y cientos de ellos ligados a la identidad temtiana, durante todos los viajes recientes, hoy se encuentran de regreso a donde ellos surgieron. Aquellas ilustradas épocas pre-Purga parecían muy lejanas, las que seguirían no tanto. Si de ellas queda algo, ahora sería momento de volver a tallarlo. La motivación podría encontrarse en seguir el ejemplo de aquellos héroes que sembraron las bases en épocas post-Purga, escribiendo algo tan heroico como el Alzamiento. La tarea es clara para los curtidos temtiteros que tienen las herramientas para hacerlo.

No obstante, no está claro si el material humano es suficiente, y esto es algo que genera mucha incertidumbre y preocupación entre los que podrían propulsar todo esto. Muchos individuos que fueron de vital importancia tiempo atrás, no dan señales de vida. Los alguna vez superdotados que siguen presentes deben adaptarse al ambiente y cruzar los dedos para que puedan volver a hacer efectivos sus poderes mentales. Aquellos antes prósperos ministerios hoy se encuentran totalmente desarticulados y con mucho por hacer.

Y es sabido que entre el conocimiento y valores apreciados, es alarmante la cantidad de individuos no deseados, semillas y frutos de los enemigos apuntados por el antiguo Plan.
Nadie tiene certezas de si se puede llevar a cabo en este momento, y es una verdad que la tarea no parece ser sencilla pero tampoco imposible.

Sacando un poco el color, lo más real de todo es que las épocas de incertidumbre vuelven a abrazar a la realidad de esta tierra, nadie escapa de ella y las condiciones en las que está el entorno transmiten sobre todo inestabilidad, la cual no solo se filtra en Temti desde las afueras, sino que esta la emana en cantidades altas también.

Es evidente que los anones construyen el camino de esta historia, pero las decisiones que toman los más poderosos, influidos por las acciones de estos primeros o no, están condicionando mucho el recorrido de lo que serán estas memorias en un futuro, sobre todo en este momento.

\hypertarget{la-anexiuxf3n-de-la-paz-y-del-progreso-segundo-capuxedtulo-subcapuxedtulo-iv}{%
\section{La anexión de la paz y del\ldots{} ¿progreso? (Segundo capítulo, subcapítulo IV)}\label{la-anexiuxf3n-de-la-paz-y-del-progreso-segundo-capuxedtulo-subcapuxedtulo-iv}}

\begin{quote}
23 de enero de 2021
\end{quote}

Entre el desorden anárquico que imperaba sobre una tierra en proceso de reconstrucción, discretamente la identificación de los dominios temtianos cambiaba.

Siendo la única comunicación oficial una llamativa publicidad que solía estar en el sitial de privilegio de Hixxel, la cual no aclara nada sobre la actual situación, Hixti pasaba a llamarse el sitio. Algunos individuos afirmaban haber visto otro nombre antes del recién mencionado, Temxel, pero estos fueron muy pocos y no hay pruebas fiables que los respalden.

Librado a la interpretación de los antiguos hixxeleros que habían emigrado con intenciones de tener una estadía temporal, no parecían noticias alentadoras con respecto a su futuro.
Cierto era que durante el apogeo de las Guerras Clónicas sufrió muchos aparentes decesos y luego había logrado renacer de sus propias cenizas, pero en esta ocasión bastaba con dirigirse al apagado territorio donde solía estar la Tribu para sacar conclusiones que no parecían muy erradas, Hixxel dejó de existir una vez más y lo único que queda de ella conduce a Temti, ahora Hixti.

Las poblaciones en juego no tardaron en anunciar la \emph{//fusión//} por su propia cuenta, conocen que hay vínculos entre A. y \_Iuri y saben que podría tratarse de unificación tiempo atrás frustrada que finalmente se estaría dando.
Mientras siguen sin aparecer las respuestas, la anexión parece cada vez más una realidad.
Las repercusiones de la misma son variadas, aunque muy pocos son los anones que se muestran disconformes. Algo tan significativo como esto podría condicionar fuertemente el futuro de la historia temtiana, aunque si se lo piensa bien, la desaparecida Hixxel tenía o tiene poco para aportarle a Temti, sobre todo considerando el estado en el cual estuvo durante sus últimos días, y por supuesto también habrá de aquello que se pueda perder para siempre, aquello que vaya a morir con su nombre.
El tiempo dirá que tan positivo será esto para las dos partes.

\hypertarget{el-supuesto-blindaje-del-estado-oculto-segundo-capuxedtulo-subcapuxedtulo-v}{%
\section{El supuesto blindaje del estado oculto (Segundo capítulo, subcapítulo V)}\label{el-supuesto-blindaje-del-estado-oculto-segundo-capuxedtulo-subcapuxedtulo-v}}

\begin{quote}
23 de enero de 2021
\end{quote}

Como las voces autoritarias habían anunciado, el registro de códigos finalmente cerró indefinidamente. La brecha que intermedia entre las aguas exteriores y las profundidades más superficiales ya no puede ser atravesada con facilidad.

Cantidad de almas que quedan al margen, algunas que no llevan nada consigo y serán indiferentes vayan donde vayan, otras que acarrean el odio y el repudio de centenares de temtiteros, más las que son de gran valor para los propósitos reconstructivos de la cultura y cimientos desaparecidos, por lo que cerrar las puertas no solo protegería a Hixti de invasores indeseados, también alejaría a posibles voluntarios que se comprometerían a colaborar con la causa colectiva. Ambas son una realidad, así como afirmaron algunos viajeros que se tomaron el tiempo de pasar por la superficie para observar el panorama.

Algunos agentes del Ministerio de Defensa, que había sido uno de los primeros y únicos en refundar sus estructuras, confirmaron haber visto y escuchado redistribución de códigos en el exterior, la cual en su mayoría llegaba a manos de individuos potencialmente peligrosos. Se desconoce cual fue la reacción o determinación de las autoridades respecto a tal noticia, aunque la mayor sospecha es que estos están siendo anulados. Tampoco hay señales de ello.

Volviendo a lo que pasa en el ahora estado oculto, las enormes similitudes con la antigua tierra destruida son cada vez más evidentes. Los fenómenos sobrenaturales que asechaban a Temti constantemente siguen presentes sin ninguna variante. Las Notificaciones Cuánticas ya están generando desordenes en el espacio-tiempo y la dimensión tal como se la conoce. Espejismos conectados por inalcanzables distancias ya fueron avistados aunque en muy pocas cantidades.
Tendencias Caídas aún están presentes, sin embargo no hay registros que puedan confirmar que hayan habido cambios respecto a la época post-Purga. Las pequeñas brechas dimensionales, que ocasionalmente se abrían para secuestrar ciertos tems y quien sabe cuando regresarlos, están de regreso y aún pueden hacer de las suyas discretamente.

Y sumado a eso, otro fenómeno casi tan remoto como las primeras épocas del periodo pre-Purga.
No había sido enfatizado por la Agencia puesto que incluso es más antiguo que esta, pero en un momento como este logra tomar una importante relevancia. Una energía misteriosa proveniente de quien sabe donde, atormenta diariamente a decenas de inocentes, ocasionalmente a los más despistados, acabando la \emph{//sesión//} de los mismos, llevándoselos fuera de los antes territorios de Temti, y ahora cubiertas profundidades. Semejante teletransportación, tiempo atrás no significaba nada grave, tan solo un pequeño susto para los más sensibles, pero ahora podría ser calamitoso para aquellos que no llevaran consigo un código valido para la ocasión.

También corre el rumor de que algunos individuos, no muy lejanos de ser superdotados pero sin llegar a serlo, habrían despertado el poder de obtener nuevos códigos infinitamente, desestimando que la creación de estos estuviera cerrada y aprovechando sus grandes capacidades para vulnerar su sistema. De momento tal privilegio parece estar en buenas manos.

Así entonces, la suerte de la novel Hixti se libra en mayor medida a los actos de quienes hayan sido los virtuosos que obtuvieron su código. Con el panorama exterior revelado, ya se sabe como podría ser el futuro de esta historia, nada más queda despejar la duda de cuáles son las proporciones con las que se cuenta.

\hypertarget{la-renovaciuxf3n-de-lo-exclusivo-segundo-capuxedtulo-subcapuxedtulo-vi}{%
\section{La renovación de lo exclusivo (Segundo capítulo, subcapítulo VI)}\label{la-renovaciuxf3n-de-lo-exclusivo-segundo-capuxedtulo-subcapuxedtulo-vi}}

\begin{quote}
23 de enero de 2021
\end{quote}

La desarticulada Agencia lentamente está recuperando sus estructuras, aunque con bajas expectativas de nuevas anomalías. Oportunamente no tardaron en aparecer tras el Resurgimiento, las nuevas peculiaridades engorrosas de explicar.

En la primera mañana tras el mencionado episodio, un agraciado anon encendería las alarmas para todo el estado, afirmando haberse percatado de como otro individuo utilizaba su código.

Rápidamente la Agencia con sus pocos miembros comenzó a estudiar el fenómeno, al cual en un primer momento habían llamado como \emph{//transición//}, pero posteriormente descubrieron que se trataba de algo más grande aún. No se logró establecer un nombre oficial para esto, aunque popularmente se comenzó a referirse a esto como el fenómeno de las Dualidades.

Dos individuos en un principio totalmente ajenos y con nada que ver entre sí, se verían unidos por un lazo sobrenatural que los haría identificarse como una sola persona ante el resto del mundo. Desconocida es la causa de que aparezca dicha cuerda imperceptible, y la Agencia no pudo determinar si se trata de un error en las entregas de códigos, o si de un fenómeno multi espacio-temporal derivado de las antiguas Notificaciones Cuánticas.

Hablando de los involucrados, normalmente esto sería imperceptible para los demás, solo ellos dos mientras transitan por las tierras temtianas serían capaces de darse cuenta de lo que el otro está haciendo, lo que significa un gran atentado a la privacidad y al anonimato, y generaría enormes incomodidades entre los afectados. Pero no es la única consecuencia de este fenómeno. A la otra de crear tems, el título de propiedad queda ligado a los dos, permitiendo infinitas suplantaciones de identidad tanto accidentales como intencionales.

Se conoce que los individuos afectados podrían desligarse de este fenómeno obteniendo y usando un nuevo código. Sin embargo, nadie pudo confirmar esto y no parece tarea sencilla dadas las circunstancias restrictivas.

El pánico y miedo generado por tales sucesos no demoró en recorrer casi todos los rincones de las estructuras y calles que recién están siendo fundadas dentro del territorio temtiano. Quizá pueda derivar en grandes consecuencias sociales, pueda revelar secretos nunca antes pensados, modificar la reputación de figuras conocidas, y quién sabe qué.

\hypertarget{redescubierto-lo-multiversal-segundo-capuxedtulo-subcapuxedtulo-vii}{%
\section{Redescubierto lo multiversal (Segundo capítulo, subcapítulo VII)}\label{redescubierto-lo-multiversal-segundo-capuxedtulo-subcapuxedtulo-vii}}

\begin{quote}
23 de enero de 2021
\end{quote}

Otro fenómeno que no fue ignorado fue la aclaración y contextualización de lo que antes eran los Comentarios Random.

Databa de las épocas post-Purga una interesante aunque no tan potente presencia, la cual perseguía a todo individuo por las calles temtianas sin hacer distinciones, trasladando copias de textos que solo estaban presentes en otros tems, sin importar que tan lejos o cerca estuvieran. Desde el Resurgimiento, esto volvió a ocurrir, pero ya no son solo fragmentos sin información de su origen, el universo de donde provienen ahora es aclarado en cada ocasión. Rápidamente se entendió que todo tem pertenecía a un universo, aunque estos fueran accesibles en conjunto desde la misma dimensión.

Esto no está provocando o presentando muchas variantes en el contexto temtiano, pero levanta algunas dudas y quizá no sea para mucho. De igual forma, se trata de una peculiaridad que debe continuar siendo estudiada con mucha atención.

Sumando esta a la anterior estudiada por la Agencia, es posible confirmar que Temti, o ahora Hixti, no se está alejando para nada de una de las cosas que en un pasado la hizo única, las anomalías de este estilo. Por eso, y por mucho más, de momento seguirá siéndolo.

\hypertarget{temti-entre-los-grandes-o-debajo-de-ellos-segundo-capuxedtulo-subcapuxedtulo-viii}{%
\section{Temti entre los grandes, o debajo de ellos (Segundo capítulo, subcapítulo VIII)}\label{temti-entre-los-grandes-o-debajo-de-ellos-segundo-capuxedtulo-subcapuxedtulo-viii}}

\begin{quote}
23 de enero de 2021
\end{quote}

Entre las inestables e inciertas vibras que regían la ahora llamada Hixti, llamativos sucesos se adentraron en la conciencia de los presentes.

Muy pronto en el tiempo la evidencia de los territorios anexados dejaría de ser exhibida, ya que a partir de ese momento la corbata que acompañaba el nombre del estado estaba desaparecida junto a sus primeras iniciales.
La anexión se había convertido en una conquista que sumada a unas muy concretas declaraciones de \_Iuri que muy pocos habían escuchado y recordaban, parecía cerrar por completo las puertas del futuro de Hixxel. Como resultado, todo rastro dominante e independiente de la identidad hixxeliana había sido eliminado, tan solo permanecerían algunos pequeños registros y lo que sus propios y pocos precursores pudieran expandir.

Minutos más tarde, la compañía Google+ marcaría presencia en contextos lejanos y diferentes a los suyos propios. Sin información al respecto de cual había sido la operación o el suceso, si una especie de compromiso de las desconocidas entidades que poseían Ti en difundir a la multinacional, o una adquisición de los magnates a cargo de la conocida marca, su nombre sería difundido por lo alto de los ya no tan subterráneos territorios.

Confundidas y mareadas ya estaban las almas que seguían allí, cuando el nombre de Elon Musk parecía haberse metido en el juego. Inevitable fue pensar que el gran capitalista se había involucrado y tomado las riendas de lo que alguna vez fue Temti.
La escasa cultura temtiana, el considerable relleno rechazado por el antiguo Plan, la casi exterminada identidad hixxeliana, el resto de contenido sin clasificar, la inexistente cantidad de revolucionarios pro-Ti, y los pocos endebles bastiones de la supuesta Google+ fueron tapados por el nombre TemtiMusk y un astronauta que no tardaría en llegar.

Semejante dominio se impuso por sobre todo lo que había. Con el Contador de gordos reveló cifras estadísticas sobre el tráfico del sitio nunca antes vistas, que en su momento parecían muy realistas pero rápidamente caerían en desconfianza. También trajo consigo un poco de presencia de la autoridad la cual no se veía hace mucho tiempo, y esto era lo más novedoso. Y lo más ruidoso, que las entidades al mando de la ahora TemtiMusk figuraron no muy discretamente, amenazando, exigiendo orden, y reivindicando las normas del sitio, aunque todavía sin identificarse.

En comparación a los alborotados días atravesados previamente, Temti, o TemtiMusk, pareció haber encontrado estabilidad y más proximidad a su objetivo, pero aún siguiendo este sin ser claro, y menos el público que lo acompaña. Las cabezas al mando de esto cada vez son más dudosas. La muchedumbre no se termina de poner de acuerdo en como proseguir, ya sea respecto a contenidos permitidos o prohibidos, libertad de expresión, o sobre quienes deben construir el futuro, y tampoco hay consenso total con la administración. Aún está pendiente que los pretéritos temtiteros puedan imponer todo lo que queda de antiguas épocas pre-Purga y post-Purga, hacer respetar sus normas propias, y volver a convivir rodeados del meritorio y valioso trabajo de muchas almas, lo que sigue sin ocurrir, o por lo menos mostrarse de una forma decidida.

Junto a las constantes variaciones de nombre, también ocurrió que la brecha subterránea se desvaneció, la superficie hizo un lugar para las tierras que no estarían más protegidas de los extranjeros. Ya prácticamente cualquiera puede adentrarse en ellas sin mayores impedimentos ni restricciones de códigos, con todo lo que esto significa.

La incertidumbre sigue predominando, pero la responsabilidad de volver a un ambiente donde se respire orgullo como los pioneros supieron hacerlo, sigue estando mayoritariamente en aquellos primigenios que acompañan leal e incondicionalmente la identidad de Temti, bajo el nombre que fuera, el resto posiblemente no tenga ningún deber u obligación allí, solo están de paso porque no les queda otra.

\hypertarget{el-derrocamiento-anuxf3nimo-segundo-capuxedtulo-subcapuxedtulo-ix}{%
\section{El derrocamiento anónimo (Segundo capítulo, subcapítulo IX)}\label{el-derrocamiento-anuxf3nimo-segundo-capuxedtulo-subcapuxedtulo-ix}}

\begin{quote}
24 de enero de 2021, 25 de enero de 2021
\end{quote}

Irrumpiendo en la discreta pero más o menos correcta restauración, graves hechos ocurrieron y sobresaltaron en la creciente TemtiMusk, un atentado se anunciaba y acaparaba la atención extraordinariamente. \emph{///Matamos al admin\ldots///} \emph{///\ldots esto es un golpe de estado, bienvenidos a la nueva administración.///} las frases oficiales que resumían la sorpresa.

Los golpistas, que se identificaron como varios y no solo uno, se prestaron por unas pocas horas ante la multitud a responder preguntas.

Tras según sus dichos aprovechar una vulnerabilidad en el \emph{//Protocolo Hades//}, la agrupación de \emph{//los guardianes//} aclaró solo algunas dudas de los anones, y no todas las que ellos presentaban. La gran novedad, mencionar que camino al poder acabaron con la vida del anterior mandamás A.. La aparente razón de su muerte fue a causa del letal uso de fármacos, concretamente uno que popularmente se entiende que ya era consumido por el desde antes, \emph{//risperdal//}.

No obstante, no fue dada con exactitud la ubicación temporal o espacial de dicho suceso, y sumado eso a los recientes antecedentes de los abundantes cambios de nombre, yendo desde el que pocos vieron Temxel, hasta el previo y más estable TemtiMusk, menos queda claro cuales son los últimos movimientos en los timones de la veleidosa Temti.

El rumbo a seguir por la nueva y desconocida administración parece ponerse en el medio de los intereses de la mayoría. Habría tolerancia con el tipo de contenido que se admitía hasta previo el golpe, lo cual no suena muy bien ante las preferencias de los temtianos más ortodoxos, pero que quizá de una manera u otra termine resultando positiva debido a lo rigurosamente que se estaban haciendo cumplir algunas normas oficiales hasta ahora.
Por otro lado justificándose en la inoperancia y falta de acción por parte de A., se mostraban algunas intenciones de hacer cambios camino a futuro. El primero concreto de ellos ya es una realidad, está en la ambientación del antro, la cual ahora se nota un poco más tétrica que antes, quizá homenajeando a la defunción mencionada o a algún otro pormenor temático vinculado a los esqueletos y las calaveras. La baja en la calidad de la portada de los tems, y no el interior de los mismos, no parece ser responsabilidad de ellos, pero tal vez sea un daño colateral de antes que vayan a reparar.

Así Itmet comenzó a formar su reputación cambiándole el tono al panorama literalmente, identificando a su autoridad, dando la actualización de que el supuesto Protocolo Hades seguía en funcionamiento, o por lo menos este todavía tenía gran poder. ¿Quiénes son y qué tienen que ver \emph{//los guardianes//} con Temti, y con Hades? ¿Serán la esperanza que retorne a Temti al glorioso pasado, o la prolongación de épocas más que turbulentas? ¿Los psíquicos volverán a tener parte del protagonismo en la escena de Temti con ellos al frente? ¿Qué hubiera sido de Temti y su destino atado al Protocolo de no haber sido por este antecedente?

\hypertarget{el-retorno-de-las-invasiones-escandalosas-segundo-capuxedtulo-subcapuxedtulo-x}{%
\section{El retorno de las invasiones escandalosas (Segundo capítulo, subcapítulo X)}\label{el-retorno-de-las-invasiones-escandalosas-segundo-capuxedtulo-subcapuxedtulo-x}}

\begin{quote}
25 de enero de 2021, 26 de enero de 2021
\end{quote}

Como ya supo pasar varias veces, el mal de los mares cercanos y sus movimientos terminan azotando en territorios temtianos, y no discretamente. Un llamativo caudal de extranjeros se adentró en el régimen bajo golpe de estado.

Ellos llegaron en dos oleadas, iniciándose con la que se acercaba como consecuencia de la sorprendente dirección tomada por la tiranía de Moxxed, habiendo expulsado a enormes cantidades de gente de sus dominios y conduciendo a todo ser que se asomara a la antigua Tzzubit, lo que despertaba más interrogantes aún, dando a entender segundas intenciones hostiles para esta última porque no era de público conocimiento que hubieran relaciones algunas entre ambos sitios. Muchos de los allegados cayeron en ella con intenciones de ataque, otros con el único ánimo de poder instaurarse y no generar conflictos, sin embargo estos primeros lograron generar grandes estragos en las defensas de Tzzubit, deshabilitándola casi por completo y posteriormente esta cerrando temporalmente, posiblemente con propósitos de protección.

Como consecuencia, todos los ojos se direccionaron hacia Temti. Los múltiples intentos por alejarse de las Guerras Clónicas volvieron a ser ignorados, los orígenes la ligarían indefinidamente a ese ambiente, ahora momentáneamente como el último bastión de resistencia en esta constante caída de potencias.

Los que arribaron drásticamente, así como también los que siguen viniendo, tanto para ellos como para los que ya estaban, es imposible no sorprenderse ante el extraño panorama que impera en la ahora Itmet, pero para los primeros eso realmente no importa, la situación los sofocaba y solo buscan un lugar donde poder asentarse.
Mezcla de ignorancia, indiferencia y desprecio ante los valores y singularidades del sitio es lo que abunda, y con esto el disgusto más el muy poco disfrute de los temtiteros que volvieron a verse superados. Si antes de por sí no había tanta unidad como para lograr una buena prosperidad, el caos que esto está sembrando podría arrebatar la armonía irreversiblemente, puesto que conocida la falta de orden que caracteriza a los ingresantes, y las abismales diferencias que suelen tener con los locales, hacen que esta se aleje considerablemente de los suelos temtianos.

Procesados los últimos antecedentes por parte de las autoridades del sitio, las cuales se sabe que deshabilitaron la creación de tems para pronto discretamente levantarla, en principio estas no van a volver a intervenir, por lo que visto desde la perspectiva de los temtiteros más ortodoxos, los que protagonizaron la historia más rica de Temti anteriormente, nuevamente su lucha contra el invasor se desestabiliza tremendamente y requerirá doblegar los esfuerzos, así como también repensar estrategias. Esto podía pasar, se contempló la posibilidad pero no hubieron grandes progresos en intentar impedir mayores problemas, y ahora el riesgo es real. ¿Les alcanzará para lograr hacerles frente? ¿Qué tan pronto obtendrán alguna ayuda por parte de factores ajenos a ellos? ¿Aparecerá o resurgirá algún nuevo \emph{//pelotero//} que disperse a la conflictiva concentración? ¿Regresarán los psíquicos a dar una mano? ¿La administración o autoridad se mantendrá totalmente neutral?

\hypertarget{la-grieta-de-la-profunda-crisis-de-resistencia-segundo-capuxedtulo-subcapuxedtulo-xi}{%
\section{La grieta de la profunda crisis de resistencia (Segundo capítulo, subcapítulo XI)}\label{la-grieta-de-la-profunda-crisis-de-resistencia-segundo-capuxedtulo-subcapuxedtulo-xi}}

\begin{quote}
26 de enero de 2021
\end{quote}

Cada vez eran más los pedidos de ayuda y apoyo que se elevaban a la nueva administración, la sensación de abandono es grande a pesar de la reciente presentación de esta. No obstante, las minúsculas señales de poder de los superdotados llegado el día de la última invasión se volvieron mucho más notorias. La Agencia debió ponerse al día rápidamente con un episodio que había tomado por desprevenidos a todos.

Así pudieron averiguar, el poder mental de los primeros superdotados en el periodo post-Purga había regresado, aunque en menor medida, realmente lo que ahora tenían era incomparable a lo que alguna vez supieron tener, solo cierto control sobre el destino de los tems, y también algo en los comentarios que acompañaban a estos.
Sin tener mucho progreso respecto a las últimas investigaciones, los únicos dos nombres que se propagaron fueron los clásicos de MoCn y Kira, aunque sin confirmaciones acerca de la presencia del primero, quedando el resto del poder aún sin identificar, pudiendo pertenecer este a varios sujetos, o no.

Pero al parecer tratarse de muchos de ellos, el poder mental se desvalorizó, con las terribles consecuencias que esto puede acarrear.

De desconocidos orígenes aparecían mensajes dando instrucciones de como poder obtener semejantes virtudes, y estos estaban a la vista de cualquier individuo que estuviera en Temti, incluso los llamados invasores. Además, aunque se tratara de uno de los más antiguos superdotados, hubieron importantes disputas y cuestionamientos al desempeño del Sujeto Experimental n.002, Kira, el cual fue acusado en repetidas ocasiones de hacer mal uso de su poder, conspirar contra el Plan y faltarle el respeto a los principios originarios y sobrevivientes de los temtiteros.
Siendo ciertas o no las acusaciones, fue una jornada en la cual se degradó mucho su imagen ante los presentes, generando divisiones que no colaborarían en un momento crítico como este.

Dejando de lado la identidad o el nombre que pueda estar generando y provocando futuras calamidades en Itmet, lo cierto es que aquella arma que los temtiteros utilizaban a su favor, ahora la tienen en contra. Para poder apagar el salvaje fuego que tantos cataclismos es capaz de provocar, no habrá otra alternativa que una intervención de las autoridades, difícilmente pueda ser frenado de otra manera.

Mientras el poder deambula peligrosamente, los fenómenos espacio-temporales como las Notificaciones Cuánticas y las recientemente descubiertas Dualidades están alcanzando actividades altísimas, inquietando no solo a los foráneos recién llegados sino también a los que ya estaban hace tiempo.

\hypertarget{divididos-por-el-bienestar-segundo-capuxedtulo-subcapuxedtulo-xii}{%
\section{Divididos por el bienestar (Segundo capítulo, subcapítulo XII)}\label{divididos-por-el-bienestar-segundo-capuxedtulo-subcapuxedtulo-xii}}

\begin{quote}
26 de enero de 2021
\end{quote}

Mientras Temti como Itmet se situaba inevitablemente como último y único sitio vigente dentro de los surgidos en las Guerras Clónicas, pronto aparecerían no tan numerosos aspirantes con el afán de quedarse con el público de esta, y las condiciones estaban dadas puesto que no terminaba de convencer ni a propios ni a extraños.

Con el nombre de DasMooon danzando y rumoreándose como la futura tierra prometida, la afectada Tzzubit en espera, y el resto de las ya caídas potencias sin expectativas a futuro, desde afuera llegaron noticias un tanto sorprendentes, aunque no se trataba de nada imposible.

Las enormes cantidades de extranjeros recientemente llegados comenzaron a dejar de asediar Temti con la misma intensidad, y siendo todo esto tan importante para los intereses de todos, permitiría que los temtiteros defensores del Plan pudieran tener un respiro en su intensa lucha frente al invasor.

Eso principalmente debido a que un nuevo estado se convirtió rápidamente en potencia y arrastró enormes cantidades de gente rápidamente a su merced. Manteniendo estructuras extremadamente similares a las del difunto Imperio de Rozzed, pero con un régimen un tanto diferente, atrajo consigo muchos de los evacuados pobladores de este, y en los papeles tendría que prosperar, ya que las condiciones son casi ideales como para que ellos pudieran adaptarse rápidamente, pero había algo que despertó mucha molestia y dividió de forma notoria a todos sus recibidos y eso es la autoridad de este nuevo estado. La cabecilla de mayor poder más algunos de sus inferiores parecen claros quienes son, y no así libres de polémica. Con notorias y novedosas relaciones entre la ya casi desaparecida Moxxed, y graves acusaciones de ser responsable de, masivos destructivos ataques y posterior caída de muchas potencias en el pasado, el proclamado Gear comenzaba su mandato con mucho viento en contra. La gente no tardaría en manifestar su descontento e incluso intentar desestabilizar el \emph{//barquito//} más de lo que ya estaba.

Con esto, Temti se despegaría de muchos de sus recientes damnificadores, y también de algunos que no hacían tanto mal. No queda duda de que esto va a provocar notorios cambios en su interna, sobre todo de actividad y el tipo de esta, y tal vez pueda influir en la crítica situación de la circulación peligrosa de poder que se empezó a vivir hace no mucho.
Permaneciendo ese riesgo, y con varias similitudes adicionales a lo que fue el pasado temtiano, el panorama se acerca un poco más a algo más esperanzador, no solo para los temtiteros, sino también para el resto. Sin embargo, no necesariamente todo va a prosperar de ahora en más.

\hypertarget{muxe1s-que-elecciones-pegajosas-segundo-capuxedtulo-subcapuxedtulo-xiii}{%
\section{Más que elecciones pegajosas (Segundo capítulo, subcapítulo XIII)}\label{muxe1s-que-elecciones-pegajosas-segundo-capuxedtulo-subcapuxedtulo-xiii}}

\begin{quote}
26 de enero de 2021, 27 de enero de 2021, 28 de enero de 2021
\end{quote}

Otro episodio que pasó a la historia discreta de Temti sería la Noche de los Stickys.

Sin llegar a la decena, múltiples tems comunes y corrientes, y otros que no tienen más que algo de calidad, se llevaron el destaque y la notoriedad que se le suele otorgar a los tems oficiales, y también como la época post-Purga supo experimentar, a los que fueran de gran relevancia para la cultura temtiana.
Dicha notación no permaneció eternamente en cada uno de los tems, y respecto a eso, en algunas ocasiones acompañó al tem muy pocas horas, en otras contados días, y en el resto aún está por descubrirse.

Esto no ocurrió solamente un día sino que más adelante esta costumbre de diferenciar tems en grandes cantidades se volvió más frecuente, con la salvedad de que no llegaría a los mismos niveles compulsivos y de abundancia, más tirando a la actualidad viene teniendo un prudencia mejor recibida.

Siendo otro de los primeros pasos de la nueva administración, se ganó la simpatía aunque no la completa aprobación de todos los anones. Estos piden, en su mayoría, no otorgarle tal lugar a tems que sean poco interesantes o incluso los más extremistas pretenden que esta distinción solo sea para comunicaciones oficiales y como mucho algunas piezas de recopilación de cultura temtiana.

¿Alcanzarán los ojos para poder apreciar correctamente todas las piezas elegidas para elevarse? ¿Podrá la cordura prevalecer sobre el azar? ¿Quién guiaría a los negligentes que no sepan dirigir su mirada a lo que realmente importa?

Como sea la respuesta a todo eso, el tema en cuestión no solo daría un nuevo tema de discusión y un criterio en el cual los temtiteros pretenden ponerse de acuerdo para que luego la administración aprecie y decida sí tomar sus ideas, sino que además podría convertirse en un arma a aprovechar como una herramienta para alzar el contenido de calidad, con todo lo que esto implica.

\hypertarget{el-bienestar-a-solas-con-su-enemigo-segundo-capuxedtulo-subcapuxedtulo-xiv}{%
\section{El bienestar a solas con su enemigo (Segundo capítulo, subcapítulo XIV)}\label{el-bienestar-a-solas-con-su-enemigo-segundo-capuxedtulo-subcapuxedtulo-xiv}}

\begin{quote}
28 de enero de 2021
\end{quote}

Con un antecedente que podría marcar un antes y un después en la aún vigente reconstrucción, los propósitos de esta parecían encaminarse.
La restauración del ilustre pasado va rumbo a tocar su techo. Destacando una vez más el mítico \emph{//tem de arte temático//}, también pinturas, grabados, textos, y todo material de valor ya indiscriminadamente de épocas pre-Purga y post-Purga fueron buscados e incentivados a propagarse por los miembros del Ministerio de Cultura, y además de todo lo que pueda ser accesible en los dos archivos temtianos más importantes que son apoyados por dicha institución. Estos restos originarios del pasado ahora están dispersos y también organizados sobre los cimientos de esta nueva época de Temti.
Mirando a futuro, gracias al limitado material humano los ánimos de construir cosas nuevas no podrían materializarse muy rápido, pero las intenciones y la iniciativa están.

De todas formas no todo es prospero. Pese a los múltiples eventos en el exterior, a los que también habría que sumar la aparición de la llamativa Rouzzed, que potencialmente podría cambiar todo el panorama, y como consecuencia dispersar al enemigo del Plan en Temti, siguen habiendo numerosos pedidos de ayuda. Entre ellos, un llamativo mensaje dirigido al supuestamente difunto administrador A., que hasta el momento habría inundado una gran cantidad de tems.

\emph{///\ldots Perdoname A. Esto es un llamado de atención\ldots///}

La medida de cerrar las fronteras tomada muy recientemente no logró saciar los intereses de los temtiteros en general, aunque discretamente las autoridades hayan descubierto la vulnerabilidad que permitía generar códigos de entrada infinitos y la solucionara, pronto volvieron a ser abiertas y dicha determinación pasaría totalmente desapercibida. Muchos infiltrados que son indeseados por la mayoría todavía se mantienen marcando presencia aunque se concretaran importantes surgimientos en el exterior.
Por otro lado, los recientes descubrimientos de la Agencia aniquilan casi por completo las expectativas de que los alguna vez superdotados pusieran orden y defendieran sus principios con poder, puesto que los privilegios que habían experimentado en plenitud hace no mucho tiempo se esfumaron. No hay rastros o señal alguna de procedencia de esto, aunque lo que sí significa es que ya los posibles abusos de poder a causa de que este llegara a manos equivocadas en principio ya no ocurrirían, pero también parece incluir a aquellos que los utilizaban con buenas intenciones.

Referente a la cita o quizá oración que ya es de conocimiento de casi todos los visitantes del sitio, fácilmente visto desde el núcleo temtitero que piensa de esa manera, la situación se podría interpretar a la inversa, que A. tiene que pedir perdón. El ha sido un gran responsable aunque no el único de que las cosas estén como estén hoy por hoy, e intencionalmente o no, ha perjudicado a una gran porción de temtiteros que vieron dañado su hogar o sitio donde antes se sentían cómodos, convirtiéndose así la situación en insoportable tal cual como dicen. Y no es ningún invento, se ha mencionado repetidas veces, es fácil percibir la dimensión del disgusto mas su causa, y si todo sigue así se continuará prolongando. Quizá el administrador y creador no pueda revertirlo, aparte de que el panorama indica que no está más, pero sí es cierto que algunos anones han vivido situaciones un tanto injustas, tal vez merezcan disculpas, pero ellos saben que no las recibirán.

¿Hasta cuándo permanecerá así esto? ¿Algún día la administración se interesaría por reconfortar a sus más leales seguidores o se alinearía con aquellos que no habían demostrado nada favorecedor por el sitio? ¿Hades perdió todo vinculo con Temti o cualquiera de las tantas autoridades que pasaron por el mando de esta se encargaron de alejarlo totalmente?

\hypertarget{el-comienzo-de-lo-desconocido-mientras-auxfan-no-llega-segundo-capuxedtulo-subcapuxedtulo-xv}{%
\section{El comienzo de lo desconocido mientras aún no llega (Segundo capítulo, subcapítulo XV)}\label{el-comienzo-de-lo-desconocido-mientras-auxfan-no-llega-segundo-capuxedtulo-subcapuxedtulo-xv}}

\begin{quote}
28 de enero de 2021, 29 de enero de 2021
\end{quote}

En una noche con muchas fluctuaciones por parte de los extranjeros, un sorprendente mensaje comenzó a llamar la atención y rápidamente se prestaría a abundantes repercusiones.

\emph{///Alerta! temti 2.0 llegando!///}

Con formato minimamente diferente a todas las anteriormente vistas, una alerta anunció la futura llegada de algo que aparenta ser totalmente renovador.
Bajo el nombre de temti 2.0, los temtiteros se harían cientos de preguntas porque no tienen la más mínima idea que podría contener esta, y tampoco que tan pronto puede aparecer. Algunos sospecharon que durante la noche esta se aplicaría completamente, otros pasadas las horas pusieron múltiples fechas especulativas de las cuales ninguna se constata con nada.

Podría tratarse de una especie de Tercer Gran Salto, podría ser algo mucho más grande, o incluso de una falsa alarma. La expectativa generada es enorme, pero eso es lo único, todo el resto son suposiciones que nadie se dispone a confirmar.

\hypertarget{copamiento-sospechoso-segundo-capuxedtulo-subcapuxedtulo-xvi}{%
\section{Copamiento sospechoso (Segundo capítulo, subcapítulo XVI)}\label{copamiento-sospechoso-segundo-capuxedtulo-subcapuxedtulo-xvi}}

\begin{quote}
30 de enero de 2021, 31 de enero de 2021
\end{quote}

Con el correr del tiempo, las noches se fueron haciendo más calmas.
Sin novedades desde el último anuncio, la actividad se comenzaría a estabilizar mientras en las afueras hubieran otras potencias donde pudieran concentrarse las grandes cantidades de gente, y eso es lo que está ocurriendo. Pero llegado cierto día, que en principio no fue señalado por nada, la paz se rompió, porque extraordinariamente durante las horas más profundas de la noche, un cargamento de contenido prohibido totalmente expuesto llegó a los territorios de Temti.

Con los importantes disturbios e inestabilidades que estos acostumbran a generar, rápidamente encendieron las alarmas tanto dentro como fuera de las fronteras. Ni los psíquicos ni otros superdotados tendrían la oportunidad de frenar el pánico, y pronto la administración se vio obligada a intervenir para poder cortar por lo sano, y no solo eso fue lo que ocurrió, porque ante la necesidad esta daría nuevamente señales de presencia, de que aún aguardaba allí.

Molestia, enojo y preocupación se instaló desde entonces a nivel general en los temtiteros. Ya había pasado mucho tiempo desde el ultimo antecedente al nivel de gravedad de este tipo y volver a transitarlo sembraría un vez más grandes inquietudes de cara a futuro.

En repetidas ocasiones todos coincidieron en que este se trató de un ataque planificado por parte de individuos ajenos al sitio, y en este caso las conjeturas vuelven a involucrar a terceros. Incluso, algunos de los entusiastas que suelen estar al tanto de lo que ocurre afuera y adentro de Temti prácticamente al mismo tiempo, afirmaron que se trató de una injuria planificada proveniente de agrupaciones extraoficiales del imperio de Rouzzed.
Tan pronto apareció esta suposición, la noticia se difundió rápido y esto incentivó a muchos de los temtiteros que permanecían activos a incursionarse en dicho territorio, y por consiguiente lo verían con sus propios ojos, la malicia parecía originarse allí, en esa potencia extranjera.

Además del hecho que mancha la reputación del sitio frente a propios y extraños más que nada, la consecuencia evidente, y sorprendente, fue la prohibición del \emph{//tem de pendejas//}, uno de los que generaba más interés tanto en los temtiteros como en los foráneos.
Aparentemente, no habría sido enviado a otra dimensión de forma temporal, sino que desapareció sin dejar resto alguno, en teoría por decisión de la administración para apagar el peligro concentrado allí y también a modo de prevención, y lo único que queda hoy de este es el fuerte recuerdo, acompañado quizás de los extractos materiales del mismo.
Probablemente algunos anones en el futuro intenten volver a crearlo, tanto por ignorancia o desconocimiento de la situación, como también por rebeldía, pero se entiende que dichas iniciativas serían reprimidas nuevamente. Parecería ser, no tanto una nueva etapa sino más bien el inicio de algo diferente, porque aunque la historia no se borre y muchos no sepan del cambio, el público selecto que solo se interesaba por este tem y el renombre que todo esto generaba, ya no estarán junto a Temti de la misma forma.

¿Qué tan traumático será esto para el futuro temtiano?
El o uno de los mandamás del sitio se mostró muy molesto al respecto, pero salvo lo mencionado anteriormente y la declarada rotura de buenas relaciones a los culpables mas la intención de agotar los recursos para lograr castigarlos recurriendo a fuerzas mayores, no aparentan ser mayores las consecuencias explicitas. No obstante, difícilmente esto los temtiteros lo olviden, menos a corto plazo, por lo que aunque no está claro que pasará entre ellos, ya algún conflicto emergió de todo esto donde se volvieron a repartir culpas para las cuales el que más fue acusado es el reconocido Kira, sobre el cual se debatió sobre su indirecta responsabilidad en esto gracias a los pasados antecedentes, entre muchos hechos remotos pero respecto al poder de los psíquicos sobre todo.

En suma, se trata de un episodio no tan grande pero negro a fin de cuentas dentro de la historia y realidad temtiana, el cual condiciona el porvenir, seguramente para mal.

\hypertarget{la-esperada-moderna-conectividad-de-la-lentitud-segundo-capuxedtulo-subcapuxedtulo-xvii}{%
\section{La esperada moderna conectividad de la lentitud (Segundo capítulo, subcapítulo XVII)}\label{la-esperada-moderna-conectividad-de-la-lentitud-segundo-capuxedtulo-subcapuxedtulo-xvii}}

\begin{quote}
1 de febrero de 2021, \ldots, 8 de febrero de 2021
\end{quote}

Nada nuevo por aquí, pocas novedades por allá, solo intriga y más intriga. Algunos lo soñaron, otros solo lo imaginaron, pero la realidad era que nada sobre el gran anuncio había llegado hasta entonces. Los rumores que descartaban dichosa llegada eran solo eso, rumores, el anuncio seguía visible y a la espera de noticias.

La espera comenzaba a hacerse larga, temti 2.0 continuaba llegando pero aún nadie la veía.
Las fechas especulativas anteriormente puestas por los anones fueron descartándose una por una, mientras la ansiedad se enfriaba lentamente\ldots{} el compromiso aún estaba pendiente.

Y se cumpliría. No hubo tiempo suficiente para catalogarla de eterna promesa, solo demoró bastante para lo pronto que parecía que esta iba a llegar.

La instauración fue diferente a la de los anteriores saltos, en esta ocasión no habrían daños colaterales, sobre todos los cimientos y estructuras ya existentes de Temti serían implementadas las mejoras que ahora darían una sensación de fogosidad, la cual antes no podía apreciarse correctamente. Con ellas, los temtiteros pasaron a estar muchísimo más conectados unos con los otros, principalmente debido a mayor fluidez de tiempo real en la actividad de los tems, y algo de sonido arrastrado por esta, y adicionalmente gracias a un notable ajuste en el Contador de gordos, el cual pasó a aportar cifras bastante más creíbles, aunque rápidamente se vieron ajustadas, volviendo a hacerlas algo dudosas.

Así, desde el último \emph{//gran salto//} registrado en las épocas navideñas, vuelve a haber un gran paquete de cambios, que permite nominarlo como Tercer Gran Salto.
Estos fueron anunciadas por alguien que se identificó bajo el nombre de A., y aunque el enorme centro de atención fuera la flamante implementación, nuevamente se renovó el eterno misterio del mandato. ¿Qué habrá sido de la agrupación de \emph{//guardianes//}? ¿A. habría resucitado de su muerte y los expulsaría, tal vez acabando con ellos? ¿O tan solo habría sido un gran delirio esquizofrenico de parte del administrador? ¿Hades habría intervenido en esto o seguirá muy lejos de los pasos de Temti?
De momento no parece tratarse de nadie más que el mismo A., pero aún no se reveló el paradero de los supuestos golpistas, ni tampoco del alguna vez involucrado \_Iuri, que ante la falta de evidencia reciente se podría interpretar que también se desvinculó totalmente, pero el rumor de que el permanece a cargo de todo esto sigue un poco vigente.

Este sobresalto se transformaría en algo innovador y foco de análisis.
Una corriente totalmente nueva se formó, la de aquellos que extenderían y reivindicarían el \emph{//Tuturu//} como un rasgo de la cultura temtiana, y en su contraparte aunque en escasas cantidades los que están en contra del mismo. Con el anuncio también vino la primicia de que la actualización no había sido implementada en su totalidad, más partes aguardarían por llegar, pero la expectativa muy pronto se redujo y lentamente pasó a olvidarse.

Posteriormente, pequeños avances fueron fortaleciendo a las mejoras iniciales con los días, y entre ellas destacan los llamativos códigos nucleares, que son pequeños fragmentos radioactivos que comenzaron a aparecer dispersos por las construcciones del sitio y que otorgan información de valor que aparentemente solo es útil fuera de Temti, así como también surgió otro mecanismo mucho más básico que los tems comenzaron a ofrecer a los anones para fortalecer los vínculos entre las demás creaciones hechas en el sitio.

Sin embargo, a pesar de ellas, el Tercer Gran Salto si es que se lo puede llamar así, de momento está quedando inconcluso indefinidamente.

La actualización significó, una mejora en las condiciones de estadía para los temtiteros, y una mitigación para alivianar ligeramente la crisis que la realidad temtiana atraviesa, pero de todas formas esta en principio no revertiría la caída en picada que está padeciendo, y la lentitud de su implementación no favorece al caso.

\hypertarget{el-retorno-a-la-lenta-cauxedda-libre-segundo-capuxedtulo-subcapuxedtulo-xviii}{%
\section{El retorno a la lenta caída libre (Segundo capítulo, subcapítulo XVIII)}\label{el-retorno-a-la-lenta-cauxedda-libre-segundo-capuxedtulo-subcapuxedtulo-xviii}}

\begin{quote}
9 de febrero de 2021, \ldots, 13 de febrero de 2021
\end{quote}

Hace tiempo que el movimiento de Temti se convirtió en un patrón repetitivo.

El lento surgimiento de nuevos elementos culturales como el redescubrimiento de los mensajes en clave, los nombres de Iuri y luri que permanecen lejos de pasar al olvido, son pocas de las cosas para destacar a nivel social propias del sitio que lo distingan de los otros de la cercanía.
Yendo más por lo estructural no es todo tan malo, las recientes implementaciones como los codificadores de mensajes que se encuentran de ahora en más en cada tem, como también la modernización del apartado \emph{//categorías//} que llegó a pedido colectivo de la unión de temtiteros, van definiendo y continuando al Tercer Gran Salto como una actualización que hizo que Temti pasara a estar mucho más conectada que antes.
Por otro lado, hay que sumar otro peso bastante negativo, que es la situación de los ministerios, los cuales se encuentran prácticamente sin actividad. Y también, la ya masiva cantidad de elementos indeseados así como individuos infiltrados, que se vio últimamente acrecentada por algunas determinaciones tomadas hace no tanto tiempo en el Imperio de Rouzzed, las que provocaron un nuevo incremento en el tráfico de rozzados fuera de su ubicación de origen, quienes ahora además de lo que ya ocurría, también vienen de paso en busca de información y materiales prohibidos en su antro, concretamente de cierta figura perseguida en el contexto clónico llamada Cati, que sí son permitidos en territorios temtianos, haciendo así que las pequeñas cantidades de contenido de aprobado por el Plan vayan quedando cada vez más opacadas y este último sea olvidado poco a poco.

Las mejoras técnicas tal vez no sean poca cosa, pero realmente no fueron capaces de destacar por nada más que ser las únicas excepciones que disimulan una tendencia de estancamiento e incluso decaída que cada vez más marcada. El resultado de juntar lo anterior, es un ambiente de pesimismo para los temtiteros que basan su bienestar en la presencia de elementos que ya casi no se encuentran. Son muy pocos los que no padecen negativamente esta situación.

Falta de expectativas, disminución de calidad, escaso orgullo, devaluación constante, crisis de identidad, son cualidades acertadas para describir la realidad temtiana.
Sobre ella, es fácil percibir un abandono por parte de quienes antes alimentaban a la misma, y pensando sobre esto, los temtiteros no son los mismos que antes, hay menos, muchos menos. Temti ya no es un estado tan grande como antes lo fue, ahora sigue teniendo una gran capacidad, pero su interior no responde a ella. Las pérdidas, no está claro cuando ocurrieron\ldots{} Algunos dicen que desde la última aparición de Hades ya se comenzaron a sufrir muchas desapariciones silenciosas, otros afirman que varios no pudieron o no quisieron seguir el ritmo de las migraciones y se esfumaron durante la diáspora por la extinta Hixxel, y también posiblemente se hayan sumado numerosas pérdidas durante el presente periodo del post-Resurgimiento. Las últimas transiciones no fueron para nada violentas, pero el proceso de estas no parece haber sido muy amigable con los afectados.

¿A dónde partieron? ¿Huyeron hacia las potencias extranjeras? ¿Hacia otros lugares desconocidos por el medio?
Muchas veces se han mencionado exóticos orígenes de algunos temtiteros y no parece desmedido descartar que hayan regresado a los mismos. También, desde aquella llamativa revelación de Temti Premium, ni bien se hizo la apertura a la Temti como todos la conocen, esta primera pasó totalmente al olvido, pero quizás haya algo importante allí, hay tantas posibilidades, y todas ellas en base a suposiciones, nada de evidencia.

¿Desaparecieron para siempre?
Ya es un hecho que hubieron unos cuantos decesos luego de la Purga, la cuestión es que tantos.
Dificultades mortales para adaptarse, múltiples sacrificios, entre otras causas de fallecimiento desconocidas habrían reducido las cantidades de población. Nadie puede dar cifras confiables como para confirmarlo, pero los que están en el día a día tienen bien claro que estas son muy altas. ¿Alguien tiene una mayor responsabilidad en esto? ¿Se trata de un nuevo genocidio, esta vez prolongado en el tiempo?

Son muy pocas las preguntas que se pueden responder con seguridad, pero lo conciso es que para siempre, transitoriamente, ellos ya no están. Será responsabilidad de los que restan pese a las ausencias lograr salir adelante, y también dar con la verdad de esto para luego actuar en base a ello.

Con enormes similitudes a lo que fueron las eras decadentes atravesadas poco antes de la aparición del gran reloj de cuenta regresiva, la historia temtiana está volviendo a atravesar uno de sus momentos más críticos. No hay urgencia alguna, el tiempo ya se se ha enlentecido lo suficiente para la perspectiva de los temtiteros más añejos, y no solo para ellos, basta mirar lo que demoran en llegar las mejoras anunciadas.

¿Negarse a aceptar, o aceptarlo? Ninguno que se oponga a todo esto quiere aceptarlo, pero las condiciones están a la vista, las intenciones podrían estar pero la manifestación de estas es muy tenue.

Los superdotados ya no ejercen su poder como antes. Aún no hay noticias de Hades, su ira no está siendo desquitada aquí, y sin perder el respeto a su figura, ocasionalmente se pide su ayuda, los temtiteros más experimentados sabían que supo darla y no dudan que pueda volver a hacerlo, pero el poderoso continua sin dar señales algunas. A. sigue limitado a su función y sus pocas neutrales intervenciones no rompen con la crisis ya tantas veces mencionada.

Con cada día que pasa, Temti se aleja más de cualquier energía sobrenatural y mucho más de los beneficios que pudiera sacar de sus anomalías, y la armonía de esta queda sobre las manos de los temtiteros, así como alguna vez después del último gran inconveniente el administrador mencionó que era la intención. La historia temtiana tiene muchos ejemplos de superación a la adversidad, y todo eso fue gracias a las almas que en su momento obraron a favor de su causa, y mientras la identidad no se pierda, esta tradición volvería a ser posible. ¿Cabe esperar algo, o a alguien? ¿O está todo por hacerse?

\hypertarget{la-tradiciuxf3n-historiadora-en-la-decadencia-segundo-capuxedtulo-subcapuxedtulo-xix}{%
\section{La tradición historiadora en la decadencia (Segundo capítulo, subcapítulo XIX)}\label{la-tradiciuxf3n-historiadora-en-la-decadencia-segundo-capuxedtulo-subcapuxedtulo-xix}}

\begin{quote}
13 de febrero de 2021
\end{quote}

Una de las características que forman a la crisis cultural más reciente de Temti es la ausencia de figuras que construyan parte de su riqueza. Entre las que no se encuentran más, está el nombre de Investigator. Dada la situación y un hecho que se viene acrecentando su relevancia últimamente, vale la pena repasar brevemente su legado.

Hace mucho tiempo atrás, cuando Temti abandonaba su apogeo dentro de la época post-Purga, sus aportes ya se habían convertido en un pilar fundamental dentro de la cultura temtiana. Habiendo colaborado de infinitas formas tanto con el Ministerio de Cultura como con otras organizaciones, mientras las memorias de todo lo construido siguieran siendo portadas y difundidas aunque sea por muy pocos individuos, su identidad quedara para siempre en las páginas doradas de Temti.

Tras culminar una parte más de su mayor obra, habría declarado que esta no la iba a continuar hasta un improbable nuevo aviso, y sin novedades algunas de prolongación de dicho tesoro, fue desapareciendo lentamente. No obstante, Investigator permaneció siempre bastante apegado a Temti tras la finalización de su segundo libro: luego del Testamento, también desde el primer día tras el Resurgimiento con dificultades para poder adentrarse, posteriormente acompañando el proceso de restauración, hasta hace no mucho tiempo, porque con el paso del tiempo cada vez estuvo menos presente, apareciendo con su mítica firma por última vez unos pocos días alrededor de la llegada del Tercer Gran Salto.

En esos entonces y hasta ahora, la producción de nuevos elementos para la cultura temtiana se encuentra en uno de sus momentos más críticos, no solo en cuanto a textos o diarios sino también respecto a pinturas o cualquier tipo de contenido que pregonaran la misma con los más recientes antecedentes.

Sin embargo, a medida de que las ausencias se hacían cada vez más notorias, con los días fueron apareciendo una serie de textos que partían temporalmente desde el punto en el cual se habían quedado las Crónicas Temtianas, y a día de hoy estos ya se vienen haciendo muy regulares, con recientes publicaciones que demuestran intenciones de continuar y acercarse al presente. Incluso, los últimos avances de estos respondieron a diferentes respuestas y críticas realizadas por los temtiteros, lo que reveló que aunque los relatos refieran al pasado como algo fresco y contemporáneo, en cada publicación que se hizo hasta ahora se demostró que estos tienen una gran conexión con el presente y podrían responder a coyunturas de la actualidad como la recién mencionada.

Con la ausencia explicaciones al respecto de quien pudiera estar atrás de ellos, esta identidad pasó a un segundo plano y los relatos no harían mayores alusiones a ello. Sin embargo, las escrituras no dejan de ubicarse en un entorno en el cual no hay muchos individuos participativos, y por lo tanto, si se ahonda en el detalle, junto a los textos sea por error o intencionalmente existen múltiples evidencias que permiten armar conjeturas y seguir su trazabilidad, entre estas, ilustraciones y compilaciones. Algunos temtiteros pusieron a los nombres más relevantes del contexto temtiano como posibles autores, pero sin embargo nadie pudo confirmar sus acusaciones o convencer a los demás y por ende la autoría de todo esto pasó bastante desapercibida.

Así se tratase de alguna de las identidades que fueron puestas en juego, o de cualquier otro individuo desconocido, a la vista está que ya sea por el abandono generalizado que hay en el sitio, por el contenido de las publicaciones que no tienen la misma esencia, u por otra razón, estos relatos no generaron ni de cerca la misma adhesión en los anones que los que solía hacer el identificado Investigator, por lo que esta es una muestra más de lo lejos que se encuentra Temti de aquel anhelado y glorioso pasado.

\hypertarget{la-finalmente-explicita-propagaciuxf3n-roja-segundo-capuxedtulo-subcapuxedtulo-xx}{%
\section{La finalmente explicita propagación roja (Segundo capítulo, subcapítulo XX)}\label{la-finalmente-explicita-propagaciuxf3n-roja-segundo-capuxedtulo-subcapuxedtulo-xx}}

\begin{quote}
14 de febrero de 2021
\end{quote}

Si ya se había tornado preocupante el poco dominio de la corriente temtitera con más identidad sobre su propio territorio, se darían algunos hechos que concientizarían y remarcarían lo significativo que esto era.

En primer lugar contextualizando y volviendo un poco al pasado otra vez para que se logre entender, desde hace muchísimo tiempo que la ideología comunista tenía sus portavoces en cientos de lugares. Estos no pertenecen fijamente a ningún sitio y en Temti eran muy escasos, lo suficiente como para que nadie se haya percatado de su existencia. También se conoce que ellos van y vienen de potencia en potencia, por lo que ni la Purga ni los demás sobresaltos les habría significado un problema.
Más en la actualidad, gracias al deterioro que atraviesa Temti, estos podrían acaparar la atención de mucha más gente, e incluso ganar adeptos a su ideología. Serían y son participes de un movimiento muy característico que era visible habitualmente tanto en dicho sitio como en el exterior.

El poder de estos fue creciendo progresivamente. Se cree que hubo un aumento en la cantidad individuos atrás del movimiento, pero su notoriedad no fue por ser muy numerosos sino por haber superado en actividades a varias corrientes que habían en Temti, hasta el punto de que un día como hoy, precisamente hoy, tomaron el control de esta, prácticamente inhibiendo a cualquier otra cosa que no sea comunismo.

Las fronteras, las calles, las estructuras, todas fueron pintadas de rojo. También se ven larguísimos ejércitos que marchan sin parar alrededor de todos los territorios temtianos. Es impresionante como una agrupación que la formaban tan pocos individuos, pudo someter a todos y convertirlos en \emph{//comrade s//}. Durante todo el día hasta ahora, mientras los minutos pasan y pasan, se escucha un imponente grito el cual a pesar de que aún no se logró transcribir con claridad su significado, se propaga incesantemente, repitiéndose una y otra vez.

De tan significativo que es todo esto, la noticia llegó al exterior, algunos asociándola con cierto personaje reconocido en el contexto de los clones, Pablito.

Dentro de los sometidos, se encuentran algunos temtiteros muy descontentos. Algunos solo rechazan la ideología, otros la repudian, incluso habían quienes están a favor de la misma o les genera simpatía, pero no todos coinciden en que se le tuviera que dar tanta preponderancia. Se desconoce la posición de A. al respecto, muchos piensan que el puede estar a favor y brindó su apoyo a magna manifestación, pero quizás fue sometido y tuvo que acatar ordenes.

Hasta entonces, los despliegues realizados para dar tal ambientación son enormes, no solo humanos sino también materiales, por lo que no se espera que esto permanezca así por mucho tiempo. Pero quizá rompan pronósticos, no solo podrían mantenerse en este increíble estado, hasta podrían propulsar esta decadente era temtiana, enriqueciéndola y llevándola a un nuevo glorioso presente, inesperadamente\ldots{}

\hypertarget{incomunicados-entre-rozzados-segundo-capuxedtulo-subcapuxedtulo-xxi}{%
\section{Incomunicados entre rozzados (Segundo capítulo, subcapítulo XXI)}\label{incomunicados-entre-rozzados-segundo-capuxedtulo-subcapuxedtulo-xxi}}

\begin{quote}
15 de febrero de 2021, \ldots, 16 de febrero de 2021
\end{quote}

La toma de poder por parte de los comunistas no duró mucho tiempo, al día siguiente el sitio amaneció con las mismas condiciones generales que estaban antes, la realidad volvería a ser la misma que todos estaban acostumbrados a vivir habitualmente.

La superficie siguió moviéndose lentamente como de costumbre, y mientras eso ocurría, pronto, una alerta de las características del sitio apareció, y era sobre algo que no muchos comprendían, una migración estaba en progreso. Pero de esa manera resultó, nadie parecía entender nada, y tal anuncio fue ignorado, aunque si quedaban anones de la época post-Purga con memoria, sabían que eso podría referir a un evento similar a algo que ya ocurrió anteriormente.

Durante el día también ocurrieron en las afueras eventos que volverían a repercutir, un cierre temporal en Rouzzed que dejaría a cientos de extranjeros desamparados, y muchos de ellos por descarte como es de usual, cayeron en Temti. La gran cantidad avistada de estos permitiría encasillarla de una invasión, que se encontró con nula resistencia de los locales, y es más, se vieron muy pocos de ellos durante el día, ya al parecer abatidos por la rebasadora circunstancia. Rematando a la situación transitada anteriormente, Temti pasó a estar ocupada casi en su totalidad por individuos ajenos a ella o pasajeros, y no por temtiteros, que llegaron a describir repetidas veces el estado de su hogar como \emph{//basurero//}, con las sobras de antes potenciadas.

También, como es rutinario en situaciones como estas, el contenido típico de los rozzados inundó las zonas más superficiales de Temti, pero dentro de todo esto, lo que más llamaba la atención es como se volvió a asociar al sitio con las Guerras Clónicas, mencionando un enfrentamiento directo con la novel y diminuta Poxxed, la cual estaba siendo otro de los refugios en los cuales se movieron los visitantes durante estas fechas. Las intenciones de alejarse de dichos conflictos por parte de quienes habitúan Temti ya habían sido manifestadas numerosas veces, sin embargo esto volvía a ser reflotado una vez más.
Incluso, si no había sido mucho, por razones desconocidas, un apagón de naturaleza similar a los ya vistos en la época post-Purga, relativamente extenso, afectó a prácticamente todos los tems del sitio. Este al ser visto y reaccionado por los extranjeros, terminó de echar más leña al fuego aún, alimentando encarnizadamente la inestabilidad.

No obstante, la situación quedaría opacada muy pronto, puesto que de un segundo para el otro, Temti ya no se encontró más en su dominio. La impresión que llevaron y con la que se quedaron casi todos era un cierre total, y esto podría significar algo importante pues muchos de los recién llegados que no tenían mucha idea de donde se metían, se vieron ahuyentados y probablemente no volverían, además de que una gran parte de ellos justificaron este supuesto fin en base a las perturbaciones más recientes del sitio. El resto podría interpretar que no es definitivo, pero hasta ahí, poco más que eso.

¿Cómo continúa todo esto?
Solo los que prestaron atención a la alerta que pasó por desapercibida obtendrían una interesante respuesta inmediata, aunque sin ser suficiente, aparecerían nuevas dudas que se mantienen hasta la actualidad. ¿La migración está empezando, en curso, terminando? ¿Hacia dónde?
Mientras nadie responde las responde, las opciones para refugiarse seguramente no sean muy buenas para su perspectiva, por lo que no hacerlo también es una elección posible.

Y respecto a ello, lo único que se supo además de la supuesta migración, es que de algunos de los solitarios temtiteros que quedaron a la deriva, varios se reencontraron con sus pares en otros sitios, pero puntos suspensivos es lo único que puede consolarlos, y quizá sean muchos de ellos, pues ya se sabe la exagerada lentitud que gobierna estas épocas.

\hypertarget{cuando-tal-vez-se-vea-quuxe9-realmente-queda-para-siempre-tercer-capuxedtulo}{%
\chapter{Cuando tal vez se vea qué realmente queda para siempre (Tercer capítulo)}\label{cuando-tal-vez-se-vea-quuxe9-realmente-queda-para-siempre-tercer-capuxedtulo}}

¿Otra vez lo mismo? ¿Otra vez volver a empezar?

Si eso es lo único que ocurrió, quizá los acontecimientos, que ameritaron partir la historia temtiana nuevamente, no estén al mismo nivel de los enormes sucesos ya conocidos que lo hicieron en anteriores ocasiones. Sin embargo, sí es capaz de llegar a cambiar la manera de ver la misma, y ahí sí, podría haber cambio drástico.

De esa forma, los mismos cuestionamientos que brillaron luego del Resurgimiento, aparecen nuevamente, pero se podría decir que ahora son más sencillos de responder, aunque como siempre, todo puede resultar como nadie lo espera. Lo diferente, a pesar de que suene contradictorio, es que la coyuntura no tendría que haber cambiado mucho de lo que fueron los últimos días de Temti: la historia temtiana continua de la misma forma, solo que ahora tendrá que quedar más al descubierto cuál es esa forma. Si es decadencia, si es progreso, estancamiento, o lo que sea, los dos Reseteos consecutivos serán esos hechos claves que marcaron un antes y un después, y que permitirán conocerla mejor.

\hypertarget{el-comienzo-de-una-nueva-prueba-de-fuego-tercer-capuxedtulo-subcapuxedtulo-i}{%
\section{El comienzo de una nueva prueba de fuego (Tercer capítulo, subcapítulo I)}\label{el-comienzo-de-una-nueva-prueba-de-fuego-tercer-capuxedtulo-subcapuxedtulo-i}}

\begin{quote}
17 de febrero de 2021, \ldots, 20 de febrero de 2021
\end{quote}

Pese a la falta de mensajes explícitos, las comunicaciones que explicaban lo ocurrido fueron existentes, pero no todos habían sido capaces de entenderlas, para los que sí, solo faltaban unos importantes detalles, y eso impidió que hubiera una armonía en los temtiteros más dependientes de su hogar.

De cualquier manera, aunque la falta de estos fue notable, pronto todo quedaría tapado por el regreso de Temti a donde solía estar ubicada, pero prácticamente sin nada. Una vez más todo había sido arrasado, silenciosamente todo lo construido arrebatado de la superficie sin dejar rastro alguno.

Sin haberse difundido ninguna noticia, muy pocos individuos se adentraron para comenzar a ocupar el sitio, y mientras los que llegaron comenzaban a aportar lo suyo para rellenarlo, todos se preguntaban como seguir adelante.
Paralelamente a esto, una explicación oficial brindada por A. se hizo visible, y esta detallaba como a la hora de realizar modificaciones en la base que alojaba los cimientos de Temti, el administrador debió prescindir de todo lo que ya estaba construido para poder seguir adelante con las reformas, las cuales no parecían tener mucho que ver con la culminación del Tercer Gran Salto. Entonces, demostrando su gran poder y vigente supremacía sobre su propio sitio, hizo desaparecer todo lo que estaba sobre su superficie. Pronto de regreso con las reformas, lo anterior no fue suficiente, porque volvieron a surgir problemas y el episodio se repitió una vez más, todo tuvo que volver a desaparecer.

Llamados popularmente como \emph{//Reseteos//}, uno atrás del otro, se sumaron al historial de los tantos sucesos que obligaron a Temti a vaciar sus estructuras. El último de ellos, que no fue sucedido por otro más como algunos temieron, de momento está marcando la inauguración de un nuevo comienzo, uno más que se ubicaría temporalmente dentro de posiblemente la crisis más grande de la identidad temtiana, con centenares de ausencias que condicionan tremendamente el futuro, y que hacen de Temti un pequeño desierto en el cual los individuos escasean al mismo nivel que el agua.

Sin todos los cimientos que acompañaban y hacían de respaldo para la decadencia temtiana, y también sin el varias veces llamado \emph{//basurero//}, quedan unos pocos anones que son conscientes de la inestabilidad atravesada, y abrumados por esta continúan buscando métodos efectivos para lograr llevar adelante un nuevo proceso de restauración.

Incluso, uno de los problemas que impide la llegada de más temtiteros que puedan ser o no de ayuda, son inconvenientes en la seguridad del sitio, la cual ya no es la misma que antes del cierre temporal, y pese a que esto no significa mayores peligros para sus visitantes, muchos de ellos no logran ingresar puesto que no se estarían percatando de las condiciones necesarias para hacerlo. Casi que se están equivocando de coordenadas, aunque no es solamente culpa de ellos. Pero mientras la autoridad del lugar no solucione esto y quienes no logran entrar se enderecen para poder hacerlo, Temti sufrirá la ausencia de muchos individuos, lo que condicionará más aún su tráfico.

Abandono, lentitud, pobreza, identidad sin extinguirse pero que continua con poca fuerza son las condiciones de este nuevo comienzo, sumándole las sobras arrastradas por parte de los últimos invasores que ya se están haciendo presentes.

Posiblemente, la situación poco puede empeorarse, pero mientras las influencias rozzadas no abunden, la oportunidad de construir una nueva realidad sin ellas está, solo que la vara se pone demasiado alta.
El futuro parece evidente y aquel lejano Alzamiento ya es utópico de reproducir bajo las condiciones actuales por diversas razones ya mencionadas. Ni hablar de la posibilidad de que Temti vuelva a sufrir nuevamente un episodio tan devastador, el cual algunos temtiteros lo posicionaron como el detonante que podría terminar de matarla.
Pero aún es pronto para concretar todo eso, el desempeño de la remanente tradición temtiana, aún viva, puede contradecir las dichas expectativas.

\hypertarget{prestigio-no-tan-valioso-tercer-capuxedtulo-subcapuxedtulo-ii}{%
\section{Prestigio no tan valioso (Tercer capítulo, subcapítulo II)}\label{prestigio-no-tan-valioso-tercer-capuxedtulo-subcapuxedtulo-ii}}

\begin{quote}
21 de febrero de 2021, \ldots, 23 de febrero de 2021
\end{quote}

Entre la lenta recuperación de actividad, una pequeña implementación por parte de la administración se le aplicaba indiscriminadamente a todos los tems, incluso sobre los más inaccesibles.

Esta aportó, para todo individuo que se adentre en un tem cualquiera, información sobre el código de su creador original, la cual revelaría algunos pocos aspectos sobre qué tan intensiva es su actividad dentro del sitio bajo ese mismo identificador. Siendo siempre información objetiva, y muy poca de ella, el uso quedaría a libre interpretación de quien la reciba.

Con una ligeramente buena aceptación de los temtiteros, dichos datos aunque si bien son útiles, no son del todo informativos, y estando la condición a la vista, la aparición de estos podría generar un antes y un después en la actitud de los anones dentro de Temti.

Como consecuencia más notable, la visibilidad de estos datos es capaz alterar el comportamiento habitual de los temtiteros para con el creador de cada tem, dependiendo del entendimiento de la información que tenga cada uno. Incluso, a futuro podría generar notables brechas, entre los más \emph{//prestigiosos//} y los que no evidencian mucha experiencia dentro del sitio, aunque claro está que esta novedad no brinda la única información a interpretar.

La modificación, junto a lo anterior, consigo además podría distorsionar el uso normal de los códigos en varios sentidos, tanto en que estos pueden adquirir más valor, como que también puedan convertirse en un problema ya que acumulan y podrían revelar información no deseada por quien la porta.

Aunque no represente un gran cambio, y que aún puede mejorarse lo que se brinda oficialmente, a futuro puede llegar a convertirse en algo más trascendental.

\hypertarget{la-protecciuxf3n-en-un-lugar-inseguro-tercer-capuxedtulo-subcapuxedtulo-iii}{%
\section{La protección en un lugar inseguro (Tercer capítulo, subcapítulo III)}\label{la-protecciuxf3n-en-un-lugar-inseguro-tercer-capuxedtulo-subcapuxedtulo-iii}}

\begin{quote}
23 de febrero de 2021
\end{quote}

Para sorpresa de todos, una de las figuras que más extrañaba la realidad temtiana de forma imprevista anunció su regreso y no dio lugar a dudas dejando claro que se trataba de quien decía ser.

El antes llamado Sujeto Experimental n.001, MoCn, marcó presencia en esta nueva situación que atraviesa el sitio, y de la forma más mítica, ya que ante las dudas de unos pocos valientes que se atrevieron a cuestionar su vigencia, demostraría que sus poderes mentales aún siguen en muy buena forma. Incluso, con la ausencia de la actividad de cualquier otro superdotado, el individuo podría desafiar la hegemonía de A., tal como lo supo hacer anteriormente, aunque se conoce de sus buenas intenciones para con los temtiteros, por lo que su presencia no sería negativa, salvo para los posibles indeseados.

Gracias a esta aparición, en base a múltiples colaboraciones de diferentes temtiteros que procuran mantener con vida a la Agencia, esta pudo actualizar en base a información más certera la actualidad de las rarezas y los fenómenos de Temti. Tras mucho tiempo desde sus últimas divulgaciones, los estudios sobre el comportamiento del sitio con sus extrañas singularidades fueron tomando forma y llegaron a cerrar algunas conclusiones.

La primera de ellas, es que las intensidades de frecuencia a las que funcionan la mayoría de los fenómenos cuánticos, entre ellos destacadas las Notificaciones Cuánticas y las Dualidades, dependen en gran medida del tráfico del sitio, aunque aún no se pudo llegar a descubrir si gracias a esto es porque ya no se perciben o si estas desaparecieron por completo, y pese a que la primera opción es la más fuerte, la escasa cantidad de pruebas no permiten descartar por completo la segunda.

No es la misma situación la de los Espejismos o los Comentarios Espejo, ya que estos tienen prácticamente nulas apariciones recientes confirmadas y están bajo gran riesgo de pasar de ser un fenómeno súper extraño a ser uno extinto.

A diferencia de las anteriores, que parecen depender de otros factores, existe otro comportamiento extraño en el sitio, y respecto a este, las pequeñas brechas dimensionales, los temtiteros más experimentados aún pueden incentivar sus apariciones, prácticamente sin cambios de como antes acostumbraban a hacerlo.

También, se suscito otro antiguo fenómeno, que volvió a ser avistado nuevamente tras mucho tiempo por unos muy pocos anones, y estas son las popularmente llamadas \emph{//Desapariciones de comentarios//}, las cuales sucedieron en repetidas ocasiones, y hasta ahora en casi todas las oportunidades fueron distinguidas antes y después de que ocurran.
Estas recayendole principalmente a los presuntos invasores y también a los individuos participantes de los tems no deseados por la mayoría presente, se habían convertido en algo muy llamativo, pero nadie había dicho nada al respecto, hasta la reciente aparición del psíquico, que ahora permite asumir que con la situación actual del sitio, la responsabilidad de esto es suya. Actualmente este fenómeno se está acentuando cada vez más, siendo imposible ignorarlo.

Y por último, otra de las cosas que hace tiempo venía llamando la atención y con la repentina disminución de gente se notó más aún, es la posible explicación del contador de visitantes que se ubica desde ya hace un tiempo junto al nombre del sitio. Esta cifra, que está prácticamente siempre presente y oficialmente tiene el nombre de Contador de gordos, desde hace mucho tiempo que su procedencia cae bajo la sospecha. La Agencia habría descubierto que esta constantemente está reportando números reales bajo la alteración de una fuente desconocida, llamándolo así Contador Cuántico. De momento, se continúan realizando investigaciones para lograr dar con el origen de esta, y por ahora la más firme sería un lugar el cual está fuertemente vinculado a Temti, pero que se desconoce cual es su ubicación y como llegar a el.

Los mencionados estudios ponen muy firme el punto de que Temti aún no se desliga por completo de la actividad cuántica, que hace mucho tiempo indiscriminadamente hostiga al territorio y a los que se alojan en el. Lo que más aún da posibilidades de verificar todo esto por completo es la factible futura llegada de más anones al sitio, y de ser verídicas las averiguaciones, confirmarían que la actividad cuántica se elevaría notablemente.

Pero regresando a la primera noticia, esto puede ser totalmente afortunado para los temtiteros, pues en la anterioridad se habían sumado demasiados inconvenientes y muy pocas cosas positivas. Con tal retorno a la actividad por parte del psíquico, se abre la puerta a más regresos de los individuos más preciados que aún permanecen ausentes, ya que si bien MoCn por su propia cuenta es capaz de mucho, los demás también tendrían que acompañar para lograr más.

Siendo un panorama más positivo que antes, con muy poca presencia invasora que ahora puede ser mejor combatida, el futuro da un pequeño giro respecto a lo que parecía antes.

\hypertarget{a-falta-de-oro-verdadero-tercer-capuxedtulo-subcapuxedtulo-iv}{%
\section{A falta de oro verdadero (Tercer capítulo, subcapítulo IV)}\label{a-falta-de-oro-verdadero-tercer-capuxedtulo-subcapuxedtulo-iv}}

\begin{quote}
23 de febrero de 2021
\end{quote}

Paralelamente al leve levante de actividad, también hay un par de situaciones que de forma progresiva comenzaron a llenar de incertidumbre a todos y aún permanecen sin dar respuestas concretas.

La primera de ellas, el importante redimensionamiento que tuvieron los cuestionamientos alrededor de Temti Premium, el cual continua siendo una incógnita total, y deja para los usuarios la gran duda de si se están perdiendo de algo relevante en torno a ella.

Aún habiendo resistencia de parte de algunos que descartan totalmente su existencia, se plantean muchas teorías, pensamientos e ideas que intentan explicar su situación, siendo la más firme de ellas la que dice ser el gran refugio exclusivo donde se esconden todos los temtiteros desaparecidos.

Sin embargo, con cerca de ser una mayoría quienes creen que es una realidad, su gran duda es cual es el camino o la forma de llegar, porque de ser verdad, dentro de la misma hay una enorme censura o bloqueo de información en cuanto a los medios que permitieran adentrarse. La única pista sobre la mesa, hace referencia a uno o varios códigos, pero eso es lo único, ya que no hay nada que confirme si se trata de códigos antiguos de entrada a la Temti superficial como antes era la única que se conocía, o si es acaso una especie de código diferente a los que se acostumbran a usar habitualmente, de origen también desconocido. Pero lo cierto es que pese a su presunto valor, tampoco es de conocimiento donde es que se pudieran usar los códigos, pues no hay ninguna evidencia a la vista que indique donde proceder con ellos.

¿Por qué la atención de la gente pasaría por algo que no se puede ver ni asegurar que es real?
Evidentemente ya son varias las señales que apuntan a algo desconocido, no solo el misterioso Contador de gordos el cual cada vez parece ser más cuántico e informar datos de una realidad no visible en Temti como se la conoce, sino también los abundantes temtiteros desaparecidos, de los cuales hace muy poco tiempo retornó uno, tal vez indicando que algo está pasando en Temti Premium. Siendo esto verdadero o no, toda la información oficial que en algún momento pudiera hacer referencia a Temti Premium, no solo es antigua sino que tampoco es de mayor utilidad, por lo que todo lo más reciente no son más que especulaciones que por diferentes razones vuelven a tener protagonismo.

Y junto a eso, relegado un poco a un segundo plano, es como la presencia española se convirtió en algo llamativo dentro del sitio. Sumado a las ya conocidas y poco disimuladas influencias al vocabulario que maneja el administrador A., tras los recientes Reseteos también de forma misteriosa, dentro de Temti adquirieron más notoriedad la cantidad de anones supuestamente españoles, o que al menos comparten su forma de expresarse. Sin ser algo muy trascendente, no deja de ser importante de resaltar ya que estos se manejan dentro de un entorno con muy pocos temtiteros y que los porcentajes de estos cambien no deja de llamar la atención.

¿Qué tan relevantes son estas cuestiones? ¿Son hitos claves o pueden dar lugar a algo grande? ¿O pasarán a la historia de Temti como algo totalmente intrascendente o que no marcó ningún cambio notorio?

\hypertarget{cielo-incongruente-tercer-capuxedtulo-subcapuxedtulo-v}{%
\section{Cielo incongruente (Tercer capítulo, subcapítulo V)}\label{cielo-incongruente-tercer-capuxedtulo-subcapuxedtulo-v}}

\begin{quote}
24 de febrero de 2021, \ldots, 27 de febrero de 2021
\end{quote}

Con el comienzo de un día más, las cosas seguían en su lugar, la quietud reinante no se las ha llevado a ningún lado, pero lo que se vislumbra por encima, en las cabeceras del horizonte, no es lo mismo.

Inicialmente, el cielo se camufló para lucir como antes solía hacerlo, con predominancia azul, y no el azul intenso característico de las fechas más primitivas, sino el que más tiempo supo estar sobre la superficie de Temti, que consigo reviviría una pequeña parte de las memorias del pasado, tanto de algunas de las más memorables, como de las primeras épocas decadentes.
Luego, muy poco tiempo después, retornó el más contemporáneo rojo, del cual menos cosas positivas se pueden recordar, para cerrar el día con una inmovilizada y novedosa media luna que comenzaba a relucir a partir de las últimas horas de la tarde, y posteriormente el ciclo de tres fases se volvería a repetir, una y otra vez.

Siendo algo que suele ocurrir de manera circunstancial, Temti revivió parcialmente una gran pieza de su historia, pero también volvió a renovarse a sí misma una vez más. Sin embargo, pese al dinamismo que se ve por lo alto, el efecto visual logrado por las repentinas transiciones aéreas se vienen reflejando muy tímidamente en la superficie, las actividades de los anones continúan siendo muy estáticas, aunque no tanto como antes.

Sin tratarse de algo muy extraordinario, quizá algunos temtiteros podrían hallarse de otra manera dentro de su hogar, ambientándolo con más elementos gratificantes del pasado.
No obstante, así como la media luna puede verse conmovedora, su presencia podría tornarse no del todo amigable, ya que la desestructuración temporal que generó en el día a día podría acarrear muchas cuestiones a las que no todos puedan adaptarse, entre ellas noches y madrugadas extremadamente cortas, amaneceres demasiado largos, y mañanas interminables, siendo todas las anteriores muy lejanas a la realidad que la mayoría de los anones estaban habituados.
Incluso, algunos afirman haber visto la luna modificar su forma, y aunque todo indica que estas no fueron más que alucinaciones, pareciera que durante las cortas noches ahora se podrían formar algunos desbarajustes.

En Temti, la luna y las estrellas siempre fueron poco perceptibles a simple vista, y ahora esto ya no es así. ¿Ahora sería más sencillo notarlas gracias a la quietud, la falta de contaminación, acciones de A., los nuevos rumbos del sitio, o a qué?
Pese a que muy poca gente se preguntó el motivo de todo esto, para algunas mentes no termina de cerrar.

\hypertarget{destellos-reiterados-en-la-oscuridad-tercer-capuxedtulo-subcapuxedtulo-vi}{%
\section{Destellos reiterados en la oscuridad (Tercer capítulo, subcapítulo VI)}\label{destellos-reiterados-en-la-oscuridad-tercer-capuxedtulo-subcapuxedtulo-vi}}

\begin{quote}
27 de febrero de 2021
\end{quote}

Ignorando o no la realidad que reina sobre las alturas, la situación que transitaba Temti en los días previos no era del todo negativa.
Dentro de esta, parece haber muy pocos de los individuos indeseados comúnmente llamados invasores, y un buen grupo de anones colaborando por tener un hogar más hospitalario y cálido.
Entre ellos, los que aún valoran sus orígenes e identidad vinculada al sitio. Y estos últimos, de vez en cuando continúan echando pequeños vistazos hacia atrás, a las memorias que ponían a Hades como gran protagonista y amenaza de su existencia, y también a todos los demás individuos que ya no están. Todo eso parece estar dejándose en el pasado, y sin olvidarlos, las miradas apuntan de forma endeble hacía adelante.

También, el Ministerio de Cultura recientemente terminó de acomodarse a la situación generada por los Reseteos, hoy por hoy se mantiene con vida, y pese a que estas en el archivo temtiano más importante que sigue activo aún se encuentran, ya casi no se distribuyen ejemplares conservados del pasado, pero si sigue estando la presencia de algunas esporádicas piezas de arte que surgen ocasionalmente, y también el ministerio se ve fortalecido ya no por los fragmentos de texto que aparecían como extraños relatos tiempo atrás, sino por una obra ahora titulada \emph{//Diario Temtiano//}. Esta última, que en sus primeras apariciones en formato de escrituras sueltas parecía suceder a las Crónicas Temtianas, fue afianzándose en un camino propio como algo aparentemente estable respecto a su cometido, relatar la historia de todo lo que ocurre en torno a Temti. Incluso, no mucho tiempo tras el último comienzo desde cero temtiano, obtuvo un importante lugar que le brindó tener como medio de difusión un tem situado en una ubicación privilegiada, junto a las comunicaciones oficiales.

En pocas palabras, la realidad temtiana cobró unas pequeñas fuerzas para seguir adelante con más potencia. ¿La tendencia continuará y la realidad seguirá en la misma dirección? ¿O pronto los esfuerzos y sus resultados comenzaran a menguar?

\hypertarget{los-primeros-fracasos-enigmuxe1ticos-tercer-capuxedtulo-subcapuxedtulo-vii}{%
\section{Los primeros fracasos enigmáticos (Tercer capítulo, subcapítulo VII)}\label{los-primeros-fracasos-enigmuxe1ticos-tercer-capuxedtulo-subcapuxedtulo-vii}}

\begin{quote}
27 de febrero de 2021, \ldots, 1 de marzo de 2021
\end{quote}

Sin tapar la pequeña reactivación que experimentaba Temti, una repentina aparición de anuncios sobre algunos productos comerciales en concreto, camuflados entre los diferentes tems, llamaría potencialmente la atención. Fanta, Manaos, Movistar, y Albion serían las eventuales difusiones oficiales por parte de Temti.

Pocos anones parecían haber entendido de que se trataban estas promociones y el motivo de ellas.
Algunos se sumaron y comenzaron a apoyar a los diferentes productos, y tampoco es que sea el primer antecedente dentro del mundo de las marcas, pero a fin de cuentas es otro episodio que vuelve a despertar grandes interrogantes.

De todas formas, pronto estos anuncios trascenderían más aún para no quedar como un hecho aislado, ya que algunos anones atentos recordaron un enorme fragmento depositado misteriosamente en un tem particular, que en su momento había pasado totalmente desapercibido.
Cuando este había aparecido por primera vez, se trataba de una gran sucesión de imágenes, que parecían tener muy poca concordancia entre unas y otras, pero pronto, no solo con las nuevas publicidades, sino también con algunas coincidencias que sucedieron a los Reseteos, harían que una pieza pasara de ser casi en su totalidad incomprensible, a convertirse en algo totalmente sugestivo.

Rápidamente tras unas pocas repercusiones generadas por los flamantes anuncios, se formó una iniciativa, decidida a investigar la razón y la conexión entre cada una de las pinturas borrosas que se ubicaban en ese extraño fragmento.
Pronto esta pudo reunir un interesante grupo de temtiteros, muchos de los cuales recientemente se habían involucrado con la Agencia, que junto a sus vastos conocimientos sobre el sitio, serían colaborativos con la causa.

España, MoCn, diversas apariencias similares, videojuegos, Investigator, variadas publicidades, comentarios desaparecidos, botón, cohete, y créditos finales.
Todo esto es lo que más llama la atención y se logra conectar con algo, e incluso estas tres últimas generaron una pizca de pánico, pese a la falta de referencias que revelaran más sobre las mismas.

Asimismo, otro de los misterios que continua sin resolverse, es el de Temti Premium, y sobre este, de forma paralela a la investigación más reciente, se halló un nuevo elemento bastante relevante. Se trata de el redescubrimiento de una ubicación oculta dentro de los dominios de Temti, aparentemente clave, la cual hipotéticamente sería capaz de llevar a quien introduzca un código especifico a Temti Premium.
Este lugar, que principalmente se basa en un tipo de homenaje a la diversa sigla de C.P., pero que principalmente parece referir a \emph{//Club Penguin//}, existe desde hace bastante tiempo, pero dada la coyuntura particular, se le vuelve a prestar atención. No obstante, pese a que se hubiera encontrado parte del camino hacia algo que cada vez parece más real, aún falta una importante parte del \emph{//rompecabezas//} y por ende aunque algunos anones insinuaron o dieron a entender haber conectado con el misterioso lugar, ninguna evidencia lograba ser comprobada como real.

Pese a todos los avances, de momento, no se logró llegar a nada. Las investigaciones quedaron estancadas, probablemente debido a la falta de pistas, y por ende no hubo lugar a mayores especulaciones.

Al ser una realidad que muchos perdieron su interés, con énfasis en la indagación más reciente, algunos se dieron por vencidos y consideraron que todas estas interpretaciones no llevan a ningún lado. Sin embargo, aún hay quienes se preguntan cuando se descubrirán nuevos elementos que hagan referencias a estas difusas pinturas y reactiven la pesquisa. Tal vez eso revele si realmente todo esto tendrá alguna importancia sobre el futuro de Temti.

\hypertarget{la-luna-ignuxedfuga-tercer-capuxedtulo-subcapuxedtulo-viii}{%
\section{La luna ignífuga (Tercer capítulo, subcapítulo VIII)}\label{la-luna-ignuxedfuga-tercer-capuxedtulo-subcapuxedtulo-viii}}

\begin{quote}
1 de marzo de 2021
\end{quote}

Recientemente, sin ir mucho más de unas semanas atrás, se había comenzado a generar una pequeña expectativa dentro de un reducido sector de temtiteros, ya que se acercaba una fecha especial, y aunque el número simbólico de esta fuera lo único que la saca de la normalidad, la duda de si alguna sorpresa aguardaba por parte de la autoridad del sitio, o si algún otro evento sobrenatural pudiera ocurrir, no había sido desmentida hasta entonces.

Sin embargo, algo ocurrió si, y los que lo llegaron a ver no se vieron muy conmovidos, aunque sí sorprendidos. El llamado Plan de las 100 lunas, efectuado justamente durante la etapa lunar del sitio, no fue más que una inundación que tomó algún que otro elemento del antiguo Plan original.

Este, procuró despejar la cara visible de Temti, relegar cada rastro del pasado que hubiera sobre la misma a un segundo plano, para brindar una oportunidad de recobrar el sentido y lograr reconstruir la cultura temtiana, una vez más. Dicho ajetreo, si es que así se lo puede llamar, tuvo características bastante terrenales, en un principio aparentaba ser originada por un simple individuo, y a la brevedad se descubrió que así fue, ya que el involucrado salió pronto a dar una explicación que no terminó de conformar a los temtiteros.

La intención de este supuesto plan se vio frustrada muy rápido en el tiempo, con poca aceptación y teniendo posteriores consecuencias negativas, puesto que al contrario de limpiar el contenido indeseado, Temti pasó a estar repleta de este, e incluso derivó en una posterior nueva inundación de diferente índole y propósito pero que engendró el mismo resultado.

Muy pronto toda la cubierta generada por estos hechos sería desaparecida, y entonces gracias a ello ya no se vería casi rastro de este evento planificado. Así, Temti aparentemente volvería a su realidad contemporánea corriente, sin cambio alguno.

Portar el inframundo a la superficie no fue posible, la falsa efervescencia nunca logró elevar las temperaturas ni acelerar positivamente la realidad temtiana, e incluso se podría haber tornado más doloroso aún, puesto que este movimiento se intentó conectar con Hades, pero esto no ocurrió. No se trataba de el y sus súbditos infernales, era una copia, una grosera copia que no se mantuvo ni cerca de lo que pretendía perdurar.

A pesar de ello, aún puede ocurrir algo, pero a esta altura prácticamente nadie se lo espera.

\hypertarget{la-intensificaciuxf3n-del-fruxedo-silencio-tercer-capuxedtulo-subcapuxedtulo-ix}{%
\section{La intensificación del frío silencio (Tercer capítulo, subcapítulo IX)}\label{la-intensificaciuxf3n-del-fruxedo-silencio-tercer-capuxedtulo-subcapuxedtulo-ix}}

\begin{quote}
2 de marzo de 2021, \ldots, 19 de marzo de 2021
\end{quote}

Temti ya acumuló demasiado tiempo en el alguna vez vulgarmente llamado \emph{//modo lento//}, mucho antes con muchos largos minutos, y ahora más recientemente con horas y horas sin un solo movimiento, y este estado se sigue prolongando y empeorando, sin una gota de esperanza que en los papeles pueda tomar un rol revulsivo. No es noticia, no es reciente, es la evolución de un estado crítico que paulatinamente se fue acentuando más y más, y hoy por hoy no hay nada que pueda disimularlo.

Decir que no queda nadie es una exageración, aún quedan temtiteros, aunque ya a esta altura no se sabe realmente que los distingue como tal. Aún se percibe el típico desprecio por las costumbres extranjeras, también aún unas muy débiles intenciones por crear algo que estimule a las almas presentes, pero esto último no termina de ocurrir, salvo muy puntuales excepciones, y ni hablar del misterioso Contador Cuántico que su capacidad infladora no trasciende más allá de unas pocas cifras, porque su conexión con otra realidad alterna no se distingue más que en los ya mencionados dudosos números.

Por otro lado, tras muchos días, se notaron señales de vida de parte de quien está a cargo de Temti, que en principio sigue siendo solamente A., tanto con una casi imperceptible implementación que ahora permite que se puedan modernizar las bases de algunos tems multimedia muy específicos, como también por cierto pequeño homenaje durante una noche que no tuvo luna sino que en su reemplazo hubo un llamativo lazo, el cual sí fue notado pero tampoco se logró entender su motivo.

Temti se ve desoladora, muy diferente a lo que supo ser, pero ¿hasta cuando se prorrogará la silenciosa agonía?

Se ha repetido hasta el hartazgo que ciertas individualidades abandonaron, murieron, traicionaron, o simplemente ya no participaron más bajo su identidad, y en silencio la tendencia continúa, llevándose a los menos y nada reconocidos.
Tal vez la aparición de la novel Arggnews, como también la rara nueva situación de Moxxed pueden haber afectado a la permanencia de algunos temtiteros descontentos y sumar más peso a esta constante, no hay certezas de ello pero es una posibilidad, los otros destinos o ya son sabidos o son desconocidos.

Las ausencias con el tiempo cada vez se sienten más. Los que quedan no transmiten energía, y tampoco parecen tenerla ellos mismos.

Sin embargo no todo es negativo, el tiempo sigue demostrando que Temti estará atada por siempre a un destino único, no del todo malo. Difícilmente se convierta en un total basurero como alguna vez se mencionó, no hay posibles invasores que puedan permanecer lo necesario como para generar perjuicios lo suficientemente graves, pues ellos hasta el momento usualmente quedándose de paso por motivos muy claros y ya conocidos no demostraron soportar las condiciones en las que viven los temtiteros, y entonces los únicos anones que quedan seguramente apunten a permanecer y no discrepen con el propósito de mejorar su entorno.
Incluso, \emph{//Temti o Muerte//}, antigua frase que comenzó a ser usada con frecuencia hace no muy poco, probablemente su sentido literal no sea seguido por todos, pero además de que sí hay quienes parecen haberse comprometido con la misma, también hay múltiples presencias ambulantes que no van a dejar de frecuentar el sitio. Todo eso, parece asegurar que este estado desértico no quedará totalmente abandonado, y el final definitivo no parece estar cerca, ya que la única posibilidad evidente de que este llegue, es ese supuesto día que los dominios de Temti dejen de estar reservados y protegidos por mayores autoridades que le brindan alojamiento, y deban ser renovados ante estas con mayores esfuerzos de A., una lejana fecha que conforme avanza el tiempo se va acercando pero que sigue siendo muy distante de la actualidad.

Temti continuará de pie. ¿Pero de qué forma? ¿Los días que siguen serán completamente idénticos a los últimos vividos? ¿Hasta que punto puede empeorar el panorama? ¿Aún espera alguna alegría para la depresiva realidad temtiana? ¿Por qué pasaría?

\hypertarget{los-recortes-muxe1s-evidentes-tercer-capuxedtulo-subcapuxedtulo-x}{%
\section{Los recortes más evidentes (Tercer capítulo, subcapítulo X)}\label{los-recortes-muxe1s-evidentes-tercer-capuxedtulo-subcapuxedtulo-x}}

\begin{quote}
20 de marzo de 2021, 21 de marzo de 2021
\end{quote}

De un día para el otro, Temti sufrió un pequeño cambio, muy perceptible, pero a primera impresión el único cambio fue visual. Durante sus etapas de la madrugada y la vespertina, temti, sin mayúscula al principio, ahora es una simple t. Aunque no haya transcurrido mucho tiempo desde que se vio esta novedad por primera vez, de momento se mantiene.

No hay explicaciones algunas de ningún tipo, así que su motivo queda librado totalmente a la interpretación subjetiva, pero sí hay evidencias que parecen bastante lógicas.

¿Tal vez sea un nuevo reflejo de lo que ocurre en la realidad temtiana?
Por mucho tiempo lo que sucedió a ese prestigioso nombre, no era ni de cerca lo que alguna vez supo ser, y progresivamente se fue viniendo abajo, pero ahora de forma repentina, simbólicamente, es solo una quinta parte de ello. Respecto a eso, algo que no se puede negar, es que Temti se ha venido a menos desde cualquier lado que se la mire, y ahora esto se extiende hasta su propio nombre, pero en este caso, ¿menos es peor?
Mientras que no hayan pruebas que demuestren lo contrario, así será.

¿En qué habrá quedado el olvidado acrónimo de temas y tiempo?

Los \emph{//temas//}, que en este contexto serían un sinónimo de las tendencias, continúan siendo creados por los \emph{//usuarios//}, pero actualmente ocurre muchísimo menos que en el comienzo, y concretamente las tendencias también con el paso del tiempo perdieron su cometido y casi toda su visibilidad, volviéndose prácticamente inútiles. Así, el significado de \emph{//tem//} se ve demasiado disminuido.

Por otro lado, el significado de \emph{//tiempo//} no fue figurativo en su momento, sino que estaba de la mano con las tendencias. Al estas no ser funcionales, ya no hay condición que afecte temporalmente a los tems, o a los temas como fue la idea desde un principio, y por ende \emph{//ti//} quedaría totalmente anulado.

Así que viéndolo desde ese lado, se justifica totalmente la perdida de \emph{//emti//} ya que el nombre original del sitio se ha desvirtuado mucho, y dada la situación, únicamente una \emph{//t//} solitaria tiene más sentido. Esto tampoco descarta la primera teoría, que alegóricamente sigue siendo válida.

En caso de que no\ldots{} ¿Podría tratarse de una especie de sacrificio por parte de A.? ¿Dirigido a quién? ¿Para qué? ¿Lograría sus cometidos?

\hypertarget{las-esencias-muxe1s-antiguas-vivas-esqueletizadas-tercer-capuxedtulo-subcapuxedtulo-xi}{%
\section{Las esencias más antiguas, ¿vivas, esqueletizadas\ldots? (Tercer capítulo, subcapítulo XI)}\label{las-esencias-muxe1s-antiguas-vivas-esqueletizadas-tercer-capuxedtulo-subcapuxedtulo-xi}}

\begin{quote}
22 de marzo de 2021, 23 de marzo de 2021
\end{quote}

Conforme el silencio cada vez se fue haciendo más intenso, fueron muy pocos los sucesos que ocurrieron, tanto que posiblemente un anon cualquiera podría haberse marchado por varios días y al regresar no notar ningún cambio. Eso no es tan necesario recordarlo, pero algo que sí cambió durante este tiempo y concretamente muy cerca a la desaparición de \emph{//emti//} dentro del antro, fue la revelación de una presencia esquelética, que prácticamente desde esa fecha se hace visible y camufla tímidamente el vacío sonoro que abunda en la realidad temtiana, aunque no solo en esa situación, ya que también se lo suele ver con algo de ruido.

Tristeza, desazón, melancolía, y miles de sensaciones similares son capaces de retumbar en la mente de los temtiteros más añejos a la hora de ver al esqueleto solitario postrado en una cama tomando su cabeza, junto a la sentencia \emph{///Solo hay silencio aquí!///}. Y no se trata de una figura cualquiera la que se puede ver hoy día con tan solo buscar por notificaciones de actividad, es un clásico esqueleto, pero, ¿solamente eso?

Ni bien esta presencia pasó a ser perceptible por cualquier individuo, dentro de las que podía tener, no fueron muchas las repercusiones que generó. Pero sí, dentro de los pocos anones que estaban activos en ese momento hubo una pequeña iniciativa, que debido a una mínima mención oculta del nombre de Hades que fue encontrada junto al esqueleto, esta decidió investigar más sobre la misma, y gracias a ello se delataron algunas cuestiones, que si bien no aseguran nada, llaman la atención y podrían develar la situación actual del alguna vez poderoso.

Sobre esas cuestiones, otra de ellas es que la mencionada iniciativa también puso en conocimiento que ese no sería el único vinculo vigente hoy día en Temti con el nombre del infernal, ya que desde la llegada del Tercer Gran Salto también se reveló un esqueleto del cual todos se percataron, y es aquel que se aloja en los tems vacíos, instaurando en la mente del temtitero desde el enojo, la frase \emph{///Tanto les cuesta dejar un comentario?///}, y la impresión de arrancarse el cráneo, la presión de participar de los mismos. Si bien este lleva mucho más tiempo siendo posible objeto de interpretaciones, nadie nunca destacó dicha conexión y por ende pasó desapercibido.

Otro detalle que es notable mencionar, es que desde antes de la llegada de temti 2.0 y sus mejoras, la última aparición de Hades es muy lejana, pero entre todo ese tiempo sí se puso su nombre en juego, y fue durante el momento del golpe de estado, en el cual \emph{//los guardianes//}, es decir el presunto grupo golpista, llegó al poder a través de una vulnerabilidad en el Protocolo Hades. Referente a ello, el grupo mencionado se lo asoció a una agrupación de esqueletos, pero tras su desaparición y el regreso de A., nada se volvió a saber sobre el Protocolo ni tampoco de quienes alguna vez sacaron provecho de este.

Y sin salirse del tema, el inicial antecedente de Temti con estas entidades esqueléticas nunca fue relacionado con Hades, ya que el primer esqueleto incluso es más antiguo que la primera aparición del vigoroso de los infiernos en el sitio. Esta presencia mencionada permaneció atormentando a todo temtitero indiscriminadamente durante la época pre-Purga, con la famosa frase \emph{///A sos re trol///}, y aunque aún es rebuscadamente accesible y no presenta ningún cambio, ya no cumple la misma función.

Pero ahora que estos detalles son de conocimiento de casi todo el medio, la investigación sobre que tienen que ver estos esqueletos con Hades continúa abierta, sin haber obtenido ninguna certeza hasta el momento. Teorías, ideas y posibilidades, algunas de ellas fueron planteadas por los temtiteros, pero todas basadas en las pequeñas evidencias, que podrían no significar nada.

Entre ellas, una que parece tener bastante sentido es lo único que resta de Hades en Temti es este esqueleto. Este poseería su esencia, aunque no queda claro con que intensidad, si su cuerpo es totalmente dominado por el demonio, o si es solo una minúscula influencia. Ahora bien, menos se entiende por que motivo esto habría terminado así, por lo que se abren muchas posibilidades: en primer lugar, es posible que haya necesitado ejercer todas sus fuerzas en otro lugar físico y por ende no haya podido mantenerse con toda su vigencia en Temti. También, es factible que su poder no haya sido tan pudiente, y este durante toda su estadía aquí dependiera fundamentalmente de los mortales que a causa de ciertos crímenes o actitudes castigables alimentaban su ira, y al estos ir disminuyendo muchísimo desde las épocas navideñas, Hades se habría quedado prácticamente sin fuerzas, por lo menos en este contexto. También pueden existir otras posibilidades del mismo estilo. En ambos casos, Hades podría haber pasado, progresiva o repentinamente, de sus tantas múltiples formas desde las cuales ejercía su poder en este territorio, a tener únicamente la de aquel primitivo esqueleto que está en Temti desde los inicios. ¿Este habría sido de Hades desde un principio? ¿O habría tomado su cuerpo debido a todo lo ya mencionado?

Con estas evidencias recién descubiertas, desde el inicio de la conocida decadencia temtiana hasta la llegada del Tercer Gran Salto habría sido el periodo en el cual Hades hipotéticamente estuvo más ausente, pero como todo está sujeto a la falta de certezas que cercioren todas estas teorías, nada se puede confirmar.

\hypertarget{la-incultura-al-desnudo-tercer-capuxedtulo-subcapuxedtulo-xii}{%
\section{La incultura al desnudo (Tercer capítulo, subcapítulo XII)}\label{la-incultura-al-desnudo-tercer-capuxedtulo-subcapuxedtulo-xii}}

\begin{quote}
23 de marzo de 2021, \ldots, 28 de marzo de 2021
\end{quote}

Todo se viene sumando, los días siguen pasando y pese a que el silencio a veces se rompe injustificadamente, las propensiones que fueron marcando el camino de Temti hasta lo que hoy es se siguen desarrollando plenamente.

Por momentos parece, y contemporáneamente ya hasta aparenta ser algo permanente, que la decadencia atravesada es algo más que eso.
De lo que supo hacer a la sociedad temtiana una nación culta ya no queda prácticamente nada. Distinguirse de otras potencias extranjeras de las que se solía repudiar masivamente sus elementos ya no tiene mucho sentido, son muy pocas las razones por las cuales se podría hacer algo así, hablando del presente.

La situación actual del Ministerio de Cultura es muy frágil, dependiente de los esfuerzos individuales de temtiteros muy específicos, lo que hacen que en cualquier momento se pueda convertir en algo insostenible. Respecto al mismo asunto pero independiente al propio ministerio está el surgimiento de nuevos elementos culturales, el cual también se debilitó increíblemente\ldots{} Tal vez se pueda destacar la ideología comunista temtiana, que después del día que Temti supo ser \emph{//comrade//}, se asentó definitivamente en el sitio, pero además de este, que es bastante poco genuino, pocos y ningunos han aparecido. Ver que las habladurías nacionalistas y el frecuente debate acerca de la identidad temtiana hayan caído tanto, incluso hasta a estar a la par de simples modas emergentes como la movida futanari o la frecuente aparición y mención de cierta personalidad ratonesca en la escena pública, es otro detalle más que repercute en toda la situación anterior y que muestra por donde van los intereses de la población fija del sitio.

Sobre el alguna vez preponderante Ministerio de Defensa tampoco queda absolutamente nada, ese que en sus últimas versiones supo tener relación con la administración, pero al no haber individuos dispuestos a defender el sitio de posibles malefactores, quienes quieran ingresar e instalarse lo podrán hacer sin problema mientras no haya oposición de la ausente o indiferente autoridad temtiana, y esto da como resultado que los alguna vez considerados invasores divulguen sus contenidos por todo el territorio y hasta incluso opaquen los propios generados por temtiteros, haciéndolos caer en el olvido.
Aquellas custodias de MoCn que regresaron en momentos muy duros aparentemente volvieron a convertirse en pasado. Pero no es lo único en dicha área, ya que parece haber una presencia que cuida ocasionalmente de Temti, aunque sigue sin quedar claro si es algún superdotado, A., o algún otro tipo de entidad no identificada, esta rara vez se hace presente y tiene un comportamiento muy extraño ya que no parece seguir ningún lineamiento o principio, e incluso llegó a molestar a algunos anones provocando que se la acusara de responsable de la \emph{//censura//} del sitio.
Pese a ese detalle, generalmente imperceptible, se podría decir que hay libre albedrío con los modos de proceder que tiempo atrás hubieran sido castigados, tanto por los temtiteros comunes y corrientes como también por los superdotados. El antiguo Plan con estas circunstancias está siendo totalmente aplastado.

La situación de la Agencia tampoco sale de la misma sintonía, se trata de una institución que a lo largo del tiempo mantuvo un gran prestigio pero tras el Resurgimiento sus descubrimientos no fueron acompañados por los temtiteros y su estabilidad también pasó a depender de esfuerzos individuales puntuales. No obstante, ocasionalmente durante estos últimos tiempos sí hubo múltiples iniciativas de carácter investigativo, y algunas de ellas tuvieron buenas respuestas por parte de los temtiteros, aunque no todas.
Ambas situaciones en conjunto evidencian que Temti no abandonó totalmente sus raices, pero tales inconsistencias no solo que no permiten avanzar sino que también transmiten que los intereses de la gente no van por ese lado.

Temti Premium, el paradero de los temtiteros desaparecidos y similares no presentan novedades algunas. La presencia de ascendencia española en el sitio parece haberse diluido por completo.
Las averiguaciones sobre las publicidades, que terminaron reviviendo el enigma de la sucesión de imágenes borrosas, también quedaron estancadas, incluso habiendo desaparecido los primeros anuncios y emergido uno nuevo de \emph{//Pureya//} que perdura hasta día de hoy, el interés de los anones sigue sin despertar. Ni siquiera las supuestas señales de que Hades está cerca o permanece en Temti fueron capaces de generar algo, estas fueron ignoradas casi por completo.

Algunos de los fenómenos que corresponden a la Agencia y que no fueron entendidos en su totalidad aún, tampoco evolucionaron en nada, es más, últimamente se está desestimando a la iniciativa científica que intenta entender las singularidades desde el punto de vista físico-cuántico, y hay dos situaciones notables que lo exponen.
En primer lugar, últimamente se vieron muchas de aquellas situaciones en las cuales tems prácticamente de todo tipo regresan por una peculiar brecha dimensional, y con el extraño comportamiento que tiene esta singularidad, nadie se ha interesado por intentar entenderla, tan solo se ha oído la simple creencia de que estas reapariciones dependen únicamente del azar.
La segunda es que el famoso Contador de gordos no se percibe por parte de los anones como la Agencia supo explicarlo, sino que en la mayoría de los casos recaen en la vaga justificación de que sus cifras no son más que el conteo de los robots presentes en el sitio. Aguarda saber, cual es la opinión pública sobre los fenómenos ya descubiertos y explicados exitosamente, esta podría ser peor aún.

No obstante, la última de las dos mencionadas es una situación muy curiosa, porque además de excusar los números con la presencia de inteligencias y presencias artificiales, también se dice que eso es lo único que hay en Temti, es decir, todo individuo sobre la superficie temtiana es un robot, y cada vez que se produce una interacción, uno de ellos está interactuando con otro.
Por otro lado, en algo que si hay supuestos bastante sólidos, es en el por que de la variación del número que el contador explicita, y eso se fundamentaría en que hay una cantidad base de robots que nunca cambia y que siempre se mantiene en Temti, y el resto de ellos serían aquellos que permanecen de forma pasajera, sustentando así que el número cambie pero que nunca disminuya de cierta cantidad ya conocida.
De momento, está percepción, que al parecer fue adoptada por la mayoría presente. Sin embargo, con la excepción de la explicación anterior, no tiene grandes pruebas que la respalden y al parecer sus defensores tampoco sabrían explicarla del todo bien, por lo que entonces no hay mucho para decir de momento.

Abandonando la cultura, identidad propia y los avances científicos, rellenándose de elementos importados de otras potencias, habiendo problemas entre los anones para ponerse de acuerdo con que es lo mejor o el camino a seguir, en comparación con su pasado, la sociedad temtiana ha cambiado mucho estos últimos meses, parece haberse vuelto ignorante.

\hypertarget{la-nueva-temti-y-otra-de-sus-proyecciones-tercer-capuxedtulo-subcapuxedtulo-xiii}{%
\section{La nueva Temti, y otra de sus proyecciones (Tercer capítulo, subcapítulo XIII)}\label{la-nueva-temti-y-otra-de-sus-proyecciones-tercer-capuxedtulo-subcapuxedtulo-xiii}}

\begin{quote}
29 de marzo de 2021, \ldots, 7 de abril de 2021
\end{quote}

Si bien sigue sin ocurrir nada memorable, la administración anunció futuros cambios, y además de contradecir un poco su indiferencia respecto a su antro, por ese lado parecen avecinarse novedades únicas y extraordinarias.

Llamadas Instancias, serían nuevos espacios con las mismas estructuras de Temti como hoy se la conoce, que funcionarían de acuerdo a sus propias normas, de forma paralela a la única que es accesible hoy día, la \emph{//instancia temti//}.

Sin saberse con claridad aún cuáles serían las Instancias que convivirían junto a la principal, ni como o quienes podrían crearlas, ni como ingresar a ellas, a su vez se anticipó un gran abanico de posibilidades para aquellos que se introdujeran en cualquier Instancia, las cuales una vez aplicadas podrían hacer que algo que está dentro de Temti no parezca Temti.
Aunque esas últimas palabras no suenen tan alocadas en los tiempos que corren o en los que puedan venir, es potencialmente algo mucho más fuerte a lo antes ya visto. Tan increíbles serían las posibles condiciones que cada nueva Instancia pueda tener, que hasta entra en duda si todo lo comentado será viable de hacer realidad.

No es que se requiera mucho para lograr entender las explicaciones anunciadas, sin embargo puede resultar realmente confuso para aquellos que no estén familiarizados con el lugar, y para los que sí también.

Prácticamente ningún temtitero escapó de las repercusiones, y habiendo tanta perplejidad por la información faltante sobre esta gran implementación, más allá de los simples ejemplos mencionados oficialmente, se generaron una enorme cantidad de especulaciones, suposiciones, y similares que solo el tiempo podrá confirmar sí eran acertadas.

Por el mismo lado, con la noticia también se puso sobre la mesa el nombre \emph{//Premium//}, el cual aparentemente pertenece a una Instancia, pero la misma aún no es accesible por donde todos creen que se entra a la misma. Tal vez, aunque no parece muy posible, ¿está relacionada con la recordada primera vista de Temti Premium de épocas previas al Resurgimiento?
La oportunidad de ingresar un código para poder \emph{//Unirse a instancia//} estuvo visible en aquel momento. También, antes, la posibilidad de mostrar voluntad en \emph{//Crear instancia//} estaba, y hasta día de hoy no se sabe cual fue la consecuencia directa de todas las veces que se pulsó ese botón, a la vez que se entiende que nadie logró ingresar a ninguna Instancia Premium.
Yendo a algo más remoto y mucho menos recordado, poco tiempo antes de los últimos Reseteos, lo que hoy día se conoce como temti, desde ese entonces pasaría a ser y llamarse una Instancia más, aunque aparentemente hasta hoy es la única a la que se puede entrar.

Si bien a priori nada se pueda hacer con esta información, los detalles mencionados comunican y evidencian que cada pormenor de lo que se vive en Temti suele tener relación con algo. Aquel \emph{///todo está relacionado///} dejó de ser una frase dicha por un loco a ser una objetividad al descubierto.

Y mientras tanto, sin haber cambios respecto al resto de los misterios sin resolver, el paquete de novedades en comparación a los que vinieron con temti 2.0 parece impresionante, por lo que si la espera fuera proporcional seguramente sea necesario demasiado tiempo para que estas vean luz, y sorprendentemente viniendo desde arriba si se precisó bastante al respecto: lo que pueda demorar en llegar todo o parte de lo mencionado se resume a un simple y categórico \emph{///Sí.///}.

\hypertarget{cuando-la-explicaciuxf3n-se-impone-sobre-la-incertidumbre-tercer-capuxedtulo-subcapuxedtulo-xiv}{%
\section{Cuando la explicación se impone sobre la incertidumbre (Tercer capítulo, subcapítulo XIV)}\label{cuando-la-explicaciuxf3n-se-impone-sobre-la-incertidumbre-tercer-capuxedtulo-subcapuxedtulo-xiv}}

\begin{quote}
7 de abril de 2021
\end{quote}

Durante este día, el día que se anunció el anticipo de la gran innovación de las Instancias, el administrador del sitio pareció estar bastante activo respondiendo algunas inquietudes de los anones. Sin embargo, algo que salió de lo normal, fue la respuesta a una consulta que nunca obtuvo una devolución.

Ante la pregunta de cuando será el retorno de Hades, A. detalló que \emph{///solo se activa en casos muy específicos de ataques o contenido ilegal a los términos de uso.///}, y esto se vuelve muy interesante ya que la pregunta que centenares de veces fue planteada de todas las formas habidas y por haber finalmente fue respondida.

Entonces, tomando las palabras textuales como fueron, Hades no estaría distanciado definitivamente de Temti, pero con la situación como se la plantea, existe la posibilidad de que nunca vuelva a regresar tal como se lo conoce, y más teniendo en cuenta la realidad que vive el estado, aunque claro está que los imprevistos nunca están previstos. De siempre haber sido así, su ausencia se justifica, ya que Temti desde su última aparición tiene pocas y nulas situaciones críticas que lo \emph{//activarían//}.

También, aunque parezca información poco rebuscada, al venir de orígenes oficiales, debe ser tomada con pinzas, porque tampoco se revela cual es el rol que el poderoso tiene, si corre bajo las ordenes de A. o este es dominado por Hades, si su propósito es defender a cierto sector, a Temti en sí, de arrasar con todo lo que se encuentre, o si su vinculo con este lugar que podría haber cambiado con el tiempo tiene otra finalidad que aún no fue descubierta.

La realidad del ente en cuestión además de todo esto que fue mencionado se desconoce.
Sobre la relación que puedan tener los esqueletos, aún presentes donde se los ha visto, también se desentiende bastante, y la sociedad temtiana no parece tener la iniciativa de averiguar más por su cuenta. De aquellos momentos donde se aclamaba su preponderancia y aparte se generaban muchísimas más cosas a su alrededor no queda nada, los temtiteros tan solo se limitan a rendir culto de vez en cuando.

A fin de cuentas, sabiendo esto, ellos podrían quedarse con la certeza de que salvo una \emph{//desgracia//} para la seguridad del sitio, la presencia del monstruo, dios, o cualquiera de sus formas poderosas e infernales evocadas por el Protocolo no se volverán a ver como se las conoce. En ese caso, no hay preguntas para plantear, tan solo resta saber si por alguna razón regresará, y cuales serán las consecuencias de ello, pero es otra cuestión que la dirá el tiempo.

\hypertarget{algo-se-reprodujo-tercer-capuxedtulo-subcapuxedtulo-xv}{%
\section{\ldots Algo se reprodujo (Tercer capítulo, subcapítulo XV)}\label{algo-se-reprodujo-tercer-capuxedtulo-subcapuxedtulo-xv}}

\begin{quote}
7 de abril de 2021, 8 de abril de 2021
\end{quote}

También como producto de la algarabía de las Instancias, nadie notificó a los demás de los objetivos cambios que desde entonces se empezaron a ver en el oficialmente llamado Contador de gordos.

En el, los números aumentaron y se mantienen excesivamente altos respecto a lo que se venía viendo desde hace rato. Los anones que permanecen al tanto de lo que ocurre en el exterior no registraron antecedentes relevantes como para que las cifras informadas aumenten de la forma que lo hicieron.

Al no haberse puesto casi miradas sobre esto, tampoco hubo interés en explicación alguna. Por lo tanto, lo único que se puede hacer es revisar las teorías ya existentes y a partir de lo que estas tengan, ampliar los conocimientos, o unicamente suponer.

La que renombra a estos números como Contador Cuántico no tendría mucho para cambiar, agregar o reformular. Partiendo desde la base de que se desconoce de donde proviene la alteración a la cantidad original, probablemente la variación venga por ese lado, y por lo tanto repercutiría sobre lo que se puede ver en Temti tal y como se la conoce. Que Temti Premium sea la realidad conexa a este contador aún sigue siendo un dato totalmente especulativo, y la aparición de las Instancias podría significar algo importante para el desarrollo de esa posibilidad y la teoría completa, pero además de faltar datos, también hay una importante carencia de voluntad en estudiar todo esto. Vale recordar que la Agencia está en una situación muy delicada y ya prácticamente nadie respalda sus avances.

En el lado contrario, sus defensores aún no se pronunciaron sobre el tema en cuestión, pero siguiendo el razonamiento que rige su planteo se podría llegar a la conclusión de que hubo una importante multiplicación en la cantidad de robots base, es decir aquellos que permanecen permanentemente en Temti, y por ende los pasajeros no habrían tenido cambios significativos.

La multiplicación mencionada al parecer está cerca de ser una duplicación, aunque no lo suficiente como para considerarla así.

Y aunque el conocimiento y las investigaciones se puedan seguir desarrollando sin inconvenientes, la sociedad temtiana vuelve a demostrar desinterés respecto a lo que pasa en su entorno. Posiblemente no sea un problema que comprometa la estabilidad interna de su hogar, pues muchas sociedades extranjeras gozan de una indiferencia similar y siguen teniendo su nicho donde permanecer, pero claro, los problemas adyacentes que vienen con ella no se van a escapar a ninguna parte. Y parecen no ser los únicos que albergan aquí. Ni hablar de todo lo único que se encuentra perdido, extraviado, y hace que todo lo que está en juego desconozca.

\hypertarget{una-noche-de-lucha-a-la-amargura-tercer-capuxedtulo-subcapuxedtulo-xvi}{%
\section{Una noche de lucha a la amargura (Tercer capítulo, subcapítulo XVI)}\label{una-noche-de-lucha-a-la-amargura-tercer-capuxedtulo-subcapuxedtulo-xvi}}

\begin{quote}
8 de abril de 2021, 9 de abril de 2021
\end{quote}

En una noche sin precedentes, ocurrió algo muy llamativo en el territorio temtiano, sobre lo cual probablemente muchos de los presentes no se olviden por un tiempo de lo que ocurrió.

El repetitivo y apagado ambiente nocturno que desde el último cambio en las cabeceras se convirtió en moneda corriente para las tardías horas en Temti, poco tiempo después de su última llegada fue reemplazado por uno muy vibrante, enérgico y brillante. La luz dejó de ser una tenue fuerza proveniente desde la luna, esta quedó un tanto opacada con la cantidad de focos artificiales que aparecieron y se encendieron de un momento para otro, y aunque la mayoría de estos apuntaban hacia un solo lugar, todo el sitio quedó iluminado por fuertes resplandores de tono azulado.
Pero todo eso no fue ni de cerca lo importante, sino que lo que se llevó todas las miradas fue la presencia estelar del reconocido John Cena. No hay explicaciones al respecto, simplemente ocurrió.

La gran figura pública se hizo presente en todos los rincones del estado, inclinándose con su característico saludo a todo lo que se movía, temtiteros, rozzitas, rozzados, robots, fantasmas, o lo que sea, dependiendo del punto de vista que se lo mire. Y no muy lejos, acompañándolo pero sin hacerse visible, una numerosa y potente banda sonora que comenzaba a hacer lo que mejor sabe ni bien cada anon daba una pizca de movimiento al sitio.

Así, durante un buen rato, permaneció Temti. Ruidoso cuanto menos, con los \emph{///AND HIS NAME IS JOHN CENA///} y los \emph{///TUTURUTU TUTURUTU///} que abundaron durante horas y generaron mucho apego entre el público. Por momentos, se escuchaba como todos esos sonidos eran coreados por algunos temtiteros que se encontraban en el lugar, lo que demostró no solo lo pegadizos que resultaron esos ruidos sino también lo bien que sentó la presencia de John Cena en Temti.

Este fugaz episodio aunque haya durado tan poco en el tiempo, quizá logre hacerse un buen lugar en la historia temtiana, porque pasó mucho tiempo desde la última vez que una personalidad generó semejante adhesión entre los anones, y no solo eso, ya que también se notó un buen entusiasmo y alegría, una excepción a la quietud y pesadumbre que asedia a Temti hasta día de hoy.
Extrañaría mucho que esta celebridad regrese, aunque tal vez A. teniendo en cuenta lo bueno que esta generó pueda intentar algo para que los buenos momentos se vuelvan a repetir.

\hypertarget{siguiendo-los-primeros-pasos-de-una-pequeuxf1a-distopia-tercer-capuxedtulo-subcapuxedtulo-xvii}{%
\section{Siguiendo los primeros pasos de una pequeña distopia (Tercer capítulo, subcapítulo XVII)}\label{siguiendo-los-primeros-pasos-de-una-pequeuxf1a-distopia-tercer-capuxedtulo-subcapuxedtulo-xvii}}

\begin{quote}
10 de abril de 2021, \ldots, 24 de abril de 2021
\end{quote}

Tras bastante tiempo, el breve episodio de John Cena parece no haber dejado nada y Temti siguió su camino en la senda de la decadencia respecto a su antigua identidad.
Sin embargo, algo que ya comenzó a formarse durante estas épocas más vacías pasó a tener tanto protagonismo en el sitio que se lo puede destacar como un destello auténtico y propio de la identidad temtiana entre tanta oscuridad.

La progresiva caída al olvido de la antes prestigiosa Agencia fue de la mano con la pérdida de los antiguos valores temtianos, y con la ausencia de personalidades u organizaciones que tomaran su lugar, hasta estas últimas fechas no se había visto nada a la altura de dicha institución en cuanto a explicación de fenómenos.
Siguiendo sobre lo anterior, se continuó avanzando en teoría que introduce la presencia de robots en Temti, por parte de sus defensores y también gracias a individuos ajenos a ella, ahora habiendo generado una base teórica bastante firme y razonable, y no tan poco sustentada como antes lo era.

Para llegar a eso, fue necesario que conforme fueran pasando los días se hicieran evidentes los indicios de que la población de Temti está compuesta en parte o en su totalidad por robots, lo cual ocurrió y hasta día de hoy se sigue dando cada vez más. Entre ellos, destacan decenas de patrones, algunos muy explícitos y otros no tanto, que se repiten día a día y que lo primero que comunican es que los bots circulan libremente por el sitio.
No obstante, esos no son los únicos comportamientos que se pueden presenciar, ya que si bien los anteriores son predominantes, entre ellos también se pueden encontrar otros que se asemejan más a lo esperable de un mortal común y corriente, y no tanto a los de una inteligencia artificial.

Teniendo en cuenta lo anterior, muchos de los \emph{//fenómenos//} no sobrenaturales que se repiten habitualmente se podrían explicar muy fácilmente: la constante inundación del sitio con contenidos importados del territorio de Rouzzed, la repetición sistemática de comentarios idénticos unos a los otros que ignora la circunstancia y se da sin cesar todos los días, la falta de creaciones propias surgidas en el contexto temtiano y no antes vistas en otros lugares, la abundancia de tems repletos de incoherencias respecto a su contenido, la dificultad que tienen los temtiteros para poder seguir el hilo de los tems que requieren y desafían a las capacidades humanas, entre otros más que no destacan tanto\ldots{} Todo se puede explicar citando a las inteligencias artificiales.

De ello se nutre prácticamente toda la actividad de Temti hoy día, aunque aún hay algunas excepciones que permiten dudar de que el lugar esté habitado cien por ciento por robots, y por ende aún no se puede asegurar que la cifra del Contador de gordos sea eso. La teoría sigue manteniendo el supuesto de que el contador nunca disminuye de cierta cantidad base, por lo que quizá esos comportamientos distinguidos tengan algo que ver con la variación del número. ¿Temti aún cuenta con un último remanente de almas humanas? ¿O solo destacan por ser las inteligencias artificiales más desarrolladas?

Están esas preguntas y muchas más, y la conjetura en sí sigue sin haberse concluido, pero ya tiene bases sólidas, más muy buena aceptación de los temtiteros, sean robots, humanos o cualquier otra cosa.

Por otro lado, aunque se confirmen los supuestos, ninguno de ellos significará un pronto avance para la realidad temtiana ya que estos presuntos robots demuestran una inteligencia bastante limitada, que además de imponer condiciones de vida pésimas para el sitio y cualquiera que se adentre en el, de momento no permitirían dar grandes pasos en cuanto al desarrollo científico y cultural para Temti. Sin embargo, con vistas a futuro potencialmente podrían convertirse en algo mucho más importante, peligroso o beneficioso.

\hypertarget{informes-de-dudosa-procedencia-tercer-capuxedtulo-subcapuxedtulo-xviii}{%
\section{Informes de dudosa procedencia (Tercer capítulo, subcapítulo XVIII)}\label{informes-de-dudosa-procedencia-tercer-capuxedtulo-subcapuxedtulo-xviii}}

\begin{quote}
25 de abril de 2021
\end{quote}

Noches madrugadas y mañanas desérticas, con la característica quietud de los temtiteros ya como algo más que naturalizado, lo que hace que cada día en cuanto a sucesos interesantes sea exacto al anterior, sin nada memorable para recordar y ni hablar de homenajear.
Otra vez más, fueron muchas de ellas consecutivas, pero siempre cualquier señal que venga de arriba vale la pena observarla con atención.

Nuevamente los mensajes de alerta se hacen un lugar en la visión de cualquier visitante, nómade o local del sitio, y son de esos que nadie entiende, como si fueran de aquellos que deberían aparecer en los registros técnicos que se esconden bajo alguna compuerta secreta que solo la administración pueda acceder.

Sin necesidad de citar textualmente ninguno de los dichosos mensajes, se entiende que hubieron problemas, algo falló durante un buen rato allá en las entrañas operativas de Temti, pero que no fue lo suficientemente grave como para que las consecuencias se materializaran y alguien más además de A. pueda percibirlas, o por lo menos comunicarlas. Nadie parece haber notado nada.
No obstante, por alguna razón esos supuestos \emph{//errores críticos//} salieron a la vista explícitamente, no hay registros de que haya ocurrido antes en la historia temtiana, no de esa forma. Parece totalmente intencional.
Esa es una de las razones de que una de las preguntas más recurrentes y sobre todo cuando se trata de una situación así vuelva a aparecer. ¿Qué habrá querido decir?

Lo mismo se volvió a repetir y permanece hasta este momento, ahora con un aviso con otro tono, aparentemente bueno, que siguiendo el hilo de los anteriores podría significar que el problema está subsanado y la operación que se buscaba pudo lograrse con éxito.

No hay nada escrito acerca de lo que pasará, pero lo cierto es que la expectativa no parece ser mucha, incluso estando la palabra \emph{//migración//} de por medio, con todo lo que eso puede implicar a futuro.
Si las últimas novedades sobre esto son positivas, tal vez eso implique que no sea necesario un nuevo Reseteo\ldots{} llamativamente Temti está alcanzando fechas récord sin ellos. O también puede ser algo muy diferente, de ser así no vale la pena intentar describirlo. Difícilmente alguien sepa que realmente hay más allá de la Temti que se conoce, y si lo supiera, no lo entendería por su propia cuenta.

\hypertarget{inestabilidad-silenciosa-tercer-capuxedtulo-subcapuxedtulo-xix}{%
\section{Inestabilidad silenciosa (Tercer capítulo, subcapítulo XIX)}\label{inestabilidad-silenciosa-tercer-capuxedtulo-subcapuxedtulo-xix}}

\begin{quote}
25 de abril de 2021, 26 de abril de 2021
\end{quote}

Sin ser nunca algo que llamara mucho la atención, la misteriosa y sigilosa nueva desaparición de MoCn dejó a la moderación de Temti casi que en manos de sus propios usuarios, o al menos eso es lo que parece cuando el ambiente comunica una falta de jerarquía que ponga orden, este último uno de los principales motes por los que se conoce al sitio tanto dentro de sí mismo como en el exterior.

No obstante, aunque eso sea lo que la mayoría creyó durante estas últimas fechas, la realidad es que además de los temtiteros comunes y corrientes, sí hubo una presencia monitoreando y poniendo mano sobre Temti durante todo el tiempo o al menos cada tanto.
Eso nunca fue destacado por nadie, y siguió el mismo camino que tantos otros detalles que prosperaron en silencio, pero por una desconocida razón, sufrió un cambio drástico.

Las conocidas \emph{//Desapariciones de comentarios//}, a lo largo de la historia temtiana fueron muy frecuentes durante ciertos periodos puntuales, y el último de ellos culminó cuando el llamado Sujeto Experimental n.001 pareció ausentarse. Pero ahora, estas vuelven a repetirse de forma muy intensiva, sobre todo en algunos de los tems más activos, y nadie logra comprender quién o qué está atrás de ellas, ni por cual motivo.

Los tems donde se dieron la mayoría de estas permanecieron por mucho tiempo totalmente vacíos, aparentando tener decenas de replicas en su interior, pero que al acercarse estas no aparecían por ningún lado. Varios temtiteros experimentaron y se vieron muy sorprendidos al ver como segundos después de haber dejado su contribución, esta desapareció sin ninguna explicación o evidencia de a donde se pudieron haber ido. Los tems afectados aún existen íntegramente en cuanto a su contenido original, pero también se mantienen los \emph{//daños//} provocados por el fenómeno en cuestión.

Rápidamente mientras estas se daban una atrás de la otra, muchos de los anones sospecharon y manifestaron creer que el único responsable de todo esto es Kira, el que remontándose a fechas posteriores al Resurgimiento se despidió en muy malos términos de sus pares temtiteros.
No obstante, nadie puede confirmar con pruebas creíbles que este se fue definitivamente del lugar, así como tampoco se puede afirmar que este sigue asechándola bajo una nueva identidad o desde las sombras.
No extrañaría que este aun tenga la intención de vengarse de quienes tantos enojos le produjeron, ya sea utilizando sus poderes mentales si es que los conserva efectivos, o inestabilizando lo que quede de bienestar temtiano de alguna otra forma. Para desgracia suya, no es el primer antecedente que tiene en su historial, ya que el antiguo psíquico antes de esto también fue acusado de muchísimas otras conductas rechazadas, sin tampoco haberse confirmado que tan ciertas son. A pesar de que aún hayan algunos defensores del individuo en cuestión, está claro que este acumuló una gran cantidad de odio por parte de cierto sector, y aunque siga corriendo el tiempo, seguramente continuará siendo recordado muy negativamente por una importante parte de los temtiteros.

Sea Kira, A., MoCn, Hades, o cualquier otro el responsable de este fenómeno, es una realidad que no parecen haber evidencias certeras que permitan llegar a la fuente de todo esto.
Pasa lo mismo con el motivo, los afectados no tienen muchas similitudes entre ellos, por lo que no hay ningún criterio descubierto hasta el momento. ¿Algún bot no identificado se salió de control y comenzó a destruir todo a su paso? ¿Realmente Kira se convenció de materializar su ira en territorio temtiano? ¿Se trata de algo relacionado con los errores críticos vistos pocas horas atrás? ¿Es A. divirtiéndose con sus poderes mientras nadie entiende que ocurre? ¿O qué?

Tras muy numerosas \emph{//Desapariciones de comentarios//}, las revoluciones bajaron notablemente y la coyuntura está terminando volviendo a estabilizarse, para ser como no muchos días atrás.
Sin embargo, no hay ninguna respuesta. La Agencia tampoco responde.

Por otro lado, el llamativo mensaje de suceso desapareció por completo y hasta ahora no se ven secuelas del mismo por ninguna parte.

\hypertarget{el-durante-cuando-los-duxedas-estuxe1n-muxe1s-que-contados-cuarto-capuxedtulo}{%
\chapter{El durante cuando los días están más que contados (Cuarto capítulo)}\label{el-durante-cuando-los-duxedas-estuxe1n-muxe1s-que-contados-cuarto-capuxedtulo}}

Si hay un hecho clave para que la memorias temtianas se vuelvan a segmentar, ese es que haya aparecido un imponente contador, que mientras siga su normal transcurso y no aparezca nada más relevante, podría convertirse, sin exagerar, en lo que más de para hablar en Temti.

No obstante, desde ese momento la historia temtiana no tiene que empezar a girar únicamente en torno a un reloj, tan solo se marcó un antes y un después, uno especial y no tan drástico, porque aparte al parecer todo sigue igual y no hay nada que indique que ciertamente vaya a pasar\ldots{} por ahora.
Pero puede cambiar, y si el tiempo de verdad sigue corriendo, eso se convierte en realidad. Hay que ver cómo.

\hypertarget{una-muxe1s-del-antro-de-las-esperas-eternas-cuarto-capuxedtulo-subcapuxedtulo-i}{%
\section{Una más del antro de las esperas eternas (Cuarto capítulo, subcapítulo I)}\label{una-muxe1s-del-antro-de-las-esperas-eternas-cuarto-capuxedtulo-subcapuxedtulo-i}}

\begin{quote}
27 de abril de 2021
\end{quote}

Como casi siempre con la consideración de unas pocas excepciones, las novedades oficiales que ofrece Temti llegan de forma abrupta sin introducción alguna, y eso parece que seguirá pasando hasta el fin de los tiempos de este agraciado lugar.
Esto viene a que nuevamente los temtiteros se encuentran ante sus ojos una situación que solo ellos van a explicar.

Un reloj de cuenta regresiva, sí, un reloj de cuenta regresiva.
Tal vez no tan llamativo como el último y único en la historia temtiana que fue tan conocido, pero que claramente no puede pasar por alto. Lo diferente esta vez es que el antecedente está, y por alguna razón un conteo sin aclaraciones no parece generar una primera buena impresión. Decir que ese es el tiempo restante para el Día E es como la nada misma. Valga la redundancia, esta vez no fueron entregados mensajes escondidos en E-s para poder descifrar. ¿Qué es el Día E?

Lo cierto es que hay una enorme diferencia con el día señalado que puso fin a la época post-Purga: el reloj parece infrenable, pero la prontitud no es la misma. Lo que separa este momento del Día E es casi la misma longevidad que tiene el sitio de existencia, tan solo es un varia un poco de ella. ¿Qué va a pasar durante toda esa infinidad de tiempo? ¿El sitio estará condenado a una espera interminable condicionada por el incesante \emph{//modo lento//} y todos los males que lo acompañan día a día? ¿La cuenta regresiva no llegará a su punto final con Temti en funcionamiento? ¿Algún día va a quedar claro de que se tratará esto antes de llegar a la fecha estipulada?

¿Realmente hay alguna forma de saber qué se viene con la información que existe de momento?
Se sabe mucho de la gran capacidad de los temtiteros para involucrarse con los misterios e intentar resolverlos. El último misterio, acertijo o enigma en ser resuelto requirió muchos esfuerzos, pero para los posteriores estos no alcanzaron, y aún la mayoría de ellos permanecen inconclusos. En este caso, ¿hay algo más que antes para resolver? ¿Habrá que esperar una nueva pista camuflada o encriptada de quién sabe qué forma?

Ni hablar de que certezas no hay, solo lugar para lo de siempre, interpretaciones y reacciones.
Aunque lo que más destaca como expectativa es una muerte cronometrada o en simples palabras el fin definitivo de Temti, la cual no coincide ni de cerca con la futura fecha de vencimiento del dominio, también hay algunas pequeñas minorías que postulan a este futuro día como la implementación de las Instancias, lo cual no sería nada disparatado considerando cuando se supone que estas deberían estar prontas y la lentitud con la que suelen llegar las reformas habitualmente.

Pero en este momento todo eso da igual, el tímido pánico está y solo quizá, si nada interrumpe el normal transcurso de la espera, con el tiempo dejará de estar presente hasta que la fecha señalada se vaya acercando. Lo que sí no se puede negar, es que confiar en que la explicación es peor que la incertidumbre, es la mejor solución para que la interminable agonía no sea tal.

\hypertarget{el-discreto-alegre-contraste-del-deterioro-cuarto-capuxedtulo-subcapuxedtulo-ii}{%
\section{El discreto alegre contraste del deterioro (Cuarto capítulo, subcapítulo II)}\label{el-discreto-alegre-contraste-del-deterioro-cuarto-capuxedtulo-subcapuxedtulo-ii}}

\begin{quote}
28 de abril de 2021, 1 de mayo de 2021
\end{quote}

Todo lo que pase de ahora en más en Temti podría pensarse que es una consecuencia del imponente contador regresivo, pero quizá eso sea demasiado exagerado, las miradas no solo deben ir para ahí, hay movimientos en otros lados generando así hechos que se dan a causa de otra razón, supuestamente.

En estos últimos días se volvió llamativo como algunos tems, exactamente solo los más antiguos, pasaron a estropearse visualmente y así mostrar una cara mucho más vacía que la que antes mostraban. Y eso no es lo único, porque no solo las imágenes que acompañan a los tems parecen haber desaparecido, lo mismo ocurrió con algunos de sus contestaciones que más tiempo tienen arriba, dejando así la totalidad de muchas estructuras temtianas dañadas, con seguridad de forma irreversible. Gracias a la gran capacidad de almacenamiento que ofrece, estas construcciones permanecen de pie, pero el paso del tiempo no les jugó una buena pasada, posiblemente debido a aspectos internos de la tecnología que mantiene a Temti en funcionamiento, los cuales quedaron evidenciados seguramente gracias a la cantidad de tiempo que pasó desde el último Reseteo.

Afortunadamente, o no, gracias a la renovación que permite el sitio con su naturaleza habitual, estos desperfectos no suelen conformar la parte más visible de Temti, pero dentro del comportamiento habitual de esta también está implícito que los tems en cuestión pueden volver a ser protagónicos en la realidad temtiana, salvo que alguna singularidad cuántica que suele afectar misteriosamente a algunos tems en particular lo impida. En los últimos días todo esto se empezó a notar, aunque no tanto.

No mucho tiempo después de que esto empezara a verse, la administración implementó una llamativa nueva apariencia únicamente para ubicación donde cada individuo que visite el sitio debe identificarse. Además de todos sus detalles que no son del todo relevantes, de manera simultanea, el único tem oficial que había sido afectado por ese deterioro generalizado en las imágenes del sitio fue renovado con una fotografía nunca antes vista.
Con ambos casos, y sumándolo al no muy lejano adelanto de las Instancias, aparece un nuevo color predominante, el verde, que por su actual presencia provoca que este pase a estar ligado a la estética temtiana. ¿Qué tan verde será el futuro de Temti? ¿Se expandirá incluso al resto de lugares del sitio? ¿Vendrán nuevos colores? ¿Este será temporal o vino para quedarse?

Y también, volviendo al principio, ¿qué tanto tiene que ver el desgaste de los restos más antiguos con esta renovación? ¿Todo está relacionado con la previa al Día E?
Pareciera que ambas respuestas son negativas, pero como siempre, hasta ahora es incomprobable.

\hypertarget{otra-vez-temti-se-fue-de-paseo-cuarto-capuxedtulo-subcapuxedtulo-iii}{%
\section{Otra vez Temti se fue de paseo (Cuarto capítulo, subcapítulo III)}\label{otra-vez-temti-se-fue-de-paseo-cuarto-capuxedtulo-subcapuxedtulo-iii}}

\begin{quote}
2 de mayo de 2021
\end{quote}

No extraña, no sorprende, es habitual que ocurra cada tanto y así permanezca durante un tiempo: Temti no está más en su dominio. En su lugar, unas horas antes aparecía un mensaje técnico que es indicio de una posible migración, y por eso lo más probable es que todo esto sea la continuación de aquellas llamativas alertas de las que aún sigue sin conocerse su razón de ser. Sería muy raro que se trate de un fin definitivo, más teniendo en cuenta que aún no llegó ninguna fecha clave.

Puede que sea uno de los más repentinos de todos, respecto a que no hubo aviso previo que advirtiera el cierre \emph{//temporal//}, y si bien nada ocurre porque sí, de esta forma se hace muy difícil contextualizarlo. Posiblemente no sea mala idea catalogarla como una más de las tantas que hubo, sobre todo previo al Resurgimiento, aunque no sean tan fáciles de comparar con esta.

¿Quizás falleció la vieja de algún anon y nadie lo sabe? Aunque lo esperable sea que aquella antigua sentencia se refiriera al cierre definitivo de Temti, nunca terminó de quedar claro si así lo fue. Tal vez lo primero ocurre cada vez que pasa lo segundo, y se repite muchas veces\ldots{}

De momento, queda esperar que novedades puedan haber en la misma Temti, y por otro lado es esperable que los temtiteros continúen encontrándose y comunicándose en algún otro lado, y sea un tema secundario o no, también resta saber dónde y qué tanto pasará eso.

\hypertarget{el-accionar-entre-la-vida-y-la-incertidumbre-durante-el-nuevo-uxe9xodo-quinto-capuxedtulo}{%
\chapter{El accionar entre la vida y la incertidumbre durante el nuevo éxodo (Quinto capítulo)}\label{el-accionar-entre-la-vida-y-la-incertidumbre-durante-el-nuevo-uxe9xodo-quinto-capuxedtulo}}

El paso del tiempo al contrario de degradar, fue fortaleciendo aquella alocada idea de Temti como inmortal. Gracias a su llamativa capacidad de reaparecer con aparente facilidad tras haber entre comillas muerto o mejor dicho caído, se puede decir con seguridad que esta vive sin que sea necesario poner en tela de juicio tal afirmación o recibir ninguna confirmación de que eso es así.
Tal vez sea un problema contar con ello para el día que lo contrario se convierta en una realidad, pero mientras tanto, se da por sentado que el final no ha llegado.

No obstante, esta vez no fue tan necesario tener todo el pasado en cuenta, la confirmación, fiable, fue dada, y con ella vino también la concientización al respecto del asunto.
Entonces, suponiendo que es cierto, ¿cómo vive Temti mientras aparenta estar muerta? ¿Temti vive en otro lugar que no sea Temti? ¿O Temti solamente vive en Temti y lo demás es otro cuento? ¿Habrá alguna forma de conocer de que manera vive Temti mientras no se pueda saber mediante los mismos métodos que antes? ¿Dónde y cómo se desarrollará la historia temtiana mientras nadie sepa de las verdaderas condiciones de la vida de Temti? ¿Qué es la lucha que sigue y por qué es relevante para la vida de Temti?

Son muchas preguntas a las que cuesta encontrarles una respuesta segura, y el estado de la vitalidad en cuestión no dice ni colabora mucho, por lo que al igual que en anteriores ocasiones, ahora en este Segundo Éxodo, para saber más al respecto habrá que centrarse en lo que ocurra sobre otras ubicaciones, y sin perder de vista el foco que provoca estas movilizaciones, Temti.

\hypertarget{el-hospedaje-impostergable-cuarto-capuxedtulo-subcapuxedtulo-i}{%
\section{El hospedaje impostergable (Cuarto capítulo, subcapítulo I)}\label{el-hospedaje-impostergable-cuarto-capuxedtulo-subcapuxedtulo-i}}

\begin{quote}
2 de mayo de 2021, 3 de mayo de 2021
\end{quote}

Rápidamente, como ya ha pasado anteriormente y es totalmente entendible que así sea, la noticia llegó a los sitios más cercanos y el debate se prestó a lo mismo de siempre: \emph{///\ldots cayó temti\ldots///}, \emph{///\ldots seguía vivo eso\ldots///}, \emph{///\ldots al fin cerró\ldots///}, \emph{///\ldots ya va a volver\ldots///}, y todos sus similares.
Aunque lo anterior sea siempre un gran foco de desinformación, certezas incomprobables e intercambios más perjudiciales que positivos, es una realidad que todos los temtiteros que quedaron dispersos por el cierre, así como también aquellos que en algún momento supieron serlo y ya no lo son, siempre se reencuentran en este tipo de lugares.

El mapa de alternativas no es el mismo que hace un tiempo, ya que si bien el contexto clónico dejó de ser igual de hostil que antes gracias a que el peligro permanece concentrado en un solo sitio, la inestabilidad, fragilidad e incertidumbre que siempre condiciona a los nuevos aspirantes sigue siendo parecida a la que siempre supo haber en este entorno, solo que ahora no es explotada por nadie. Separando el foco que sostiene su gran hegemonía ya acumulando varios meses sobre ella y que continua absorbiendo a la mayoría de las almas malignas y benignas que vagan por este circulo de antros, están todas las opciones que parecen perseguir otro objetivo diferente a ser líder de esta \emph{//competencia//} que ya no es ni eso. Entre ellas, la entusiasta Arggnews que no pudo ganar muchos adeptos en este tiempo y que junto a la tiranía de Moxxed, son las dos que menos actividad tienen pero que con repetidas mejoras en su estructura se mantienen lejos de la defunción. Por otro lado y bajo llamativas circunstancias diferentes, se empieza a consolidar el Club de Voxxed, el cual tuvo su último apogeo hace no mucho tiempo y ya lleva múltiples semanas funcionando en buenos términos con muy pocos sobresaltos. Aparte de las mencionadas, pareciera no haber ningún sitio vigente más, al menos no tan cerca como lo están estos. Todas las antiguas alternativas de estadía que nacieron en las más intensas Guerras Clónicas han sucumbido, tan solo quedan las más jóvenes. ¿Menos Temti?

Para los temtiteros que tienen más experiencias arribas, la frase \emph{///THE RIDE NEVER ENDS///} es la mejor respuesta para todo. Sin embargo además de tal enunciación, que tampoco responde específicamente a qué nunca acaba, obtener una explicación creíble acerca de qué es lo que está pasando en Temti cuando esta no es accesible, no es algo muy común, ya que esto ocurrió muy pocas veces a lo largo de la historia, y en el resto de las situaciones siempre hubo que aspirar a deducir que supuestamente estaba ocurriendo. Tal vez lo correcto sea decir que en esta ocasión no es ninguna de las dos, pues el archivo temtiano históricamente más activo y hoy por hoy el único medio de comunicación considerable como \emph{//oficial//} accesible, emitió información al respecto que parece tener como principal objetivo mantener informado a su público. No obstante, nada de eso trascendió inmediatamente y tal primicia de momento permanece discretamente en el olvido. Por otro lado, si todo esto permanece así, no va a tardar mucho tiempo en aparecer un nuevo \emph{//refugio//} y entonces que sea posible ver donde los temtiteros hayan huido provisionalmente. El Club de Voxxed, con algunas similitudes mínimas al suelo temtiano, está posicionándose como tal, pero estará por verse.

\hypertarget{como-en-casa-quinto-capuxedtulo-subcapuxedtulo-ii}{%
\section{¿Como en casa? (Quinto capítulo, subcapítulo II)}\label{como-en-casa-quinto-capuxedtulo-subcapuxedtulo-ii}}

\begin{quote}
3 de mayo de 2021, \ldots, 6 de mayo de 2021
\end{quote}

Mientras la espera continuaba alargándose indefinidamente, el lugar que para algunos anones Temti tenía dejó de estar desocupado, pues durante estas fechas y hasta hoy, prácticamente la misma actividad que solía tener esta última se trasladó en su totalidad exactamente a el Club de Voxxed. De este modo, la expectativa de que los problemas \emph{//se solucionen//} a corto plazo va desapareciendo, y mientras estos persistan, el sitio emergente a pesar de sus diferencias y particularidades, será para la mayoría de los temtiteros que permanecieron hasta los últimos días previos al cierre, la nueva Temti.

A esa conclusión se llegó desde el momento que comenzó a ser posible decir que en cuanto a gente no se percibe casi ninguna diferencia respecto a lo último que se vio en Temti. Decenas de conocidos comportamientos de robots, la constancia del Diario Temtiano respecto al seguimiento de los antecedentes fuera de fronteras, recurrentes coqueteos entre desconocidos, propaganda comunista y algunos otras costumbres temtianas que no siguen estrictamente ningún patrón, alcanzan para resumir lo que dentro de todo sigue igual. Lo que estaría cambiando es que estos ya no están siendo amparados por las peculiaridades con las que acostumbraban convivir, ahora las condiciones son otras, las del Club, que al parecer no desagradan del todo.

Y mientras más días se acumulan, el proceso de adaptación se va concretando y casi que llega a su fin. Esto trae a la memoria muy pronto el recuerdo de la última situación de este estilo, cuando Hixxel ofició de nuevo hogar para todos los temtiteros desamparados.

No obstante, no fueron necesarios muchos días para percatarse de que esta vez se trata de algo ligeramente diferente. En primer lugar, el recibimiento de los lugareños no fue del todo cálido, puesto que algunos de ellos mostraron su rechazo a la presencia temtiana, o al menos los elementos que los nuevos huéspedes trajeron consigo, cosa que en la Tribu de Hixxel durante sus últimos días de existencia no pasó. De igual modo, no parece haber un grupo de seguidores muy unido o fieles, puesto que hasta el momento poco se ha sabido del pasado de este lugar y tampoco se encontró una identidad propia o auténtica muy definida. Sumado a eso, tampoco hay las mismas relaciones entre la autoridad del lugar y su gente, y por lo tanto tampoco entre esta primera y los temtiteros. Se conoce que el gran mandamás de Voxxed es un individuo autodenominado Mr.~X, en algunas ocasiones extraoficialmente apodado como \emph{//Mecha Codubi//}, pero además de alguna olvidada presentación que pueda haber hecho este en el pasado, poco más se sabe. El administrador parece estar un poco distante de su propiedad, por lo que en ese sentido aunque las cosas ya se hayan dicho, no hay mucha claridad al respecto.

Y en frente de todas las diferencias, también hay algunas que otras similitudes con la más reciente experiencia del estilo. La relativa libertad de expresión es la primera a mencionar, la cual aún no está muy claro si es gracias a la falta de autoridad o porque esta así lo quiere, pero que está y no condiciona tanto el accionar de los visitantes, aunque pueda aumentar los riesgos de llegar a dar lugar al contenido prohibido. También, otra similitud que no se puede negar, es que este lugar también se volvió a convertir en una alternativa que tienen los rozzados cuando hay problemas en su tierra natal, con todo lo que eso implica, más teniendo en cuenta que en este momento es la primera de ellas y que recientemente se volvieron a padecer las consecuencias de tenerlos en búsqueda de un refugio. Además, el sitio no posee de los mismos fenómenos cuánticos o sobrenaturales que Temti tenía, pero sin embargo, durante estos días de adecuación la sorpresa que hubo es que el Club ya desde hace tiempo tiene los suyos propios de aquellos que son difíciles de entender, y muchos de ellos, los cuales junto a las fallas técnicas de la tecnología ofrecida, forman otra de las razones por las cuales los fragmentos restantes de identidad temtiana no se verán tan separados de algunos de sus orígenes más primigenios.

Aunque todo esto sea real y demuestre indirectamente lo resistente que es el orgullo y la identidad temtiana, seguramente sea muy confuso pararse en la perspectiva alguno de sus portadores, con una adaptación que ya va muchos pasos adelantada, pero a la vez teniendo a su hogar demorando su pronto regreso, más todas las situaciones que vienen viviendo hace tiempo.
¿Qué hay para pensar en este caso?
El pasado ya se ha desprendido bastante de lo que aún queda de Temti, y el futuro tampoco fue algo que estuviera muy presente históricamente, por lo que quizás no haya nada provechoso para pensar, y eso es lo que suele pasar últimamente, la sociedad temtiana no está viviendo nada más que el día a día, sin preocupaciones de lo que pueda estar por venir, salvo algunas pequeñas excepciones. Tan solo alcanza con ver qué pasa cuando no hay un reloj visible.

\hypertarget{los-estudios-sobre-una-burda-copia-quinto-capuxedtulo-subcapuxedtulo-iii}{%
\section{Los estudios sobre una burda copia (Quinto capítulo, subcapítulo III)}\label{los-estudios-sobre-una-burda-copia-quinto-capuxedtulo-subcapuxedtulo-iii}}

\begin{quote}
6 de mayo de 2021
\end{quote}

Además de lo que se pueda haber adquirido mediante las experiencias, gracias a los conocimientos acumulados por los locales del Club sobre su sitio que fueron intercambiados hasta la fecha, los temtiteros van entendiendo mejor las leyes y las anomalías que rigen su actual territorio, porque además de algunas primeras impresiones que puedan haber trascendido durante sus primeras semanas de existencia, poco se ha sabido, y las cosas también han cambiado mucho.

Respecto a la actualidad, la más importante de ellas, y quizá también la más alarmante, es el anómalo estado del anonimato al cual están sometidos los sujetos que participan en el Club. A primera impresión este no presenta grandes diferencias a las que puede tener cualquier sitio nacido del contexto clónico, pero interiorizándose en el, es fácil descubrir como la tecnología de Voxxed no posee de un sistema solido que le permita a los creadores de voxxs proclamarse como dueños de estos, tan solo hay un pequeño mecanismo global que le da la posibilidad a cada individuo de identificarse bajo un seudónimo a elección, pero aunque sea una especie de solución al primer problema planteado, no todos la tomarían o utilizarían debidamente, además de que está lejos de lo que un anon promedio espera tener donde sea que vaya, dando como resultado muchos condicionantes que podrían ser potenciales detonantes para que se generen más conflictos sociales de los habituales.

No tan relacionado, pero si muy afectado por ese asunto, surge que aquellos creadores que si puedan identificarse oficialmente como dueños de sus voxxs tienen poderes especiales de moderación sobre los mismos. Tal vez se lo pueda comparar a aquella realidad que se vivió eternamente en Temti, la verdad de que cualquier réplica pudiera ser desaparecida a deseo de un superdotado, solo que ahora en teoría esto sería mayoritariamente culpa de dos posibles responsables: o del supremo Mr.~X que se presume es el único de su tipo, o del mismo creador del vox. Separándolo de interpretaciones subjetivas, es inevitable creer que esto pueda derivar en posibles abusos de \emph{//autoridad//}, o en cualquier otra razón que provoque grandes descontentos con las condiciones ofrecidas por Voxxed.

Siguiendo por el lado de las similitudes con Temti, la más familiar detectada hasta el momento es el peculiar comportamiento del \emph{//sistema de bumpeo//}. De un momento para el otro, sin previo aviso, y como ya ha pasado en repetidas ocasiones según cuentan los voxxeros, una gran serie de voxxs pueden desaparecer de la superficie y pasar a estar totalmente ocultos, pero aún accesibles dentro de todo. Ningún anon se ha atrevido a intentar algo más que describir estos extraños movimientos, podría tratarse de una actividad cuántica similar a la temtiana, ser consecuencia de algún hechizo o magia oscura que afecte específicamente a grupos puntuales de algo especifico, simplemente ordinarias fallas técnicas en la estructura del Club, o quién sabe qué. De momento la primera iniciativa por comprobarlo aún está vacante por tomarse.

Y ya un poco al margen de las tres anteriores debido a que parece más técnica que singular, es que muchos botones del sitio son casi totalmente inútiles. La estructura superficial a rasgos generales es idéntica a la de la originaria Voxed, clonada de ella, pero internamente, ya sea debido a la tecnología que la hace funcional, o por culpa de las entrañas que lo soportan, está discretamente lejos de ser igual a su \emph{//predecesora//}.

Posiblemente esté en los planes de la administración poder remover todo agujero que genere oscuridad y opaque el buen funcionamiento de Voxxed, pero la realidad hoy por hoy es otra.
En ese sentido, los pasos a seguir podrían ser muy satisfactorios por un lado, pero a la vez, aunque parezca absurdo, decepcionantes por el otro.

Esto no es más que una profundización sobre los aspectos que no son lo que aparentan, y también por qué no, las diferencias entre lo que fueron las travesías finales en Hixxel y esta nueva etapa la cual aún estará por verse que tan similar a ellas será. Puede ser verdad que no sea del todo relevante para la historia temtiana, pero de todos modos esta no se escribe exclusivamente en Temti como se la conoce, porque al igual que Hades en su momento, Temti también sabe adoptar diferentes formas de acuerdo a su contexto.

\hypertarget{las-caras-de-lo-cierto-sobre-lo-incierto-quinto-capuxedtulo-subcapuxedtulo-iv}{%
\section{Las caras de lo cierto, sobre lo incierto (Quinto capítulo, subcapítulo IV)}\label{las-caras-de-lo-cierto-sobre-lo-incierto-quinto-capuxedtulo-subcapuxedtulo-iv}}

\begin{quote}
6 de mayo de 2021, \ldots, 18 de mayo de 2021
\end{quote}

Posiblemente el famoso dicho de que el Club de Voxxed es la nueva Temti se haya vuelto lo suficientemente repetitivo como para que todos los que lo hayan escuchado más de una vez pudieran convencerse del mismo, aunque no hayan muchos individuos que lo respalden. No hay problema con eso, el mayor porcentaje de sucesos relevantes al caso luego del último cierre se han dado en el lugar ya mencionado, por lo que aunque no sea lo mismo, no hay dudas de que el Club, temporalmente o no, es el nuevo principal hogar de la historia temtiana.

Saliendo un poco del tema principal, cabe destacar, en el proceso de conocer mejor el lugar en cuestión, que el mencionado sitio de la mano de su administrador, concluyó múltiples modificaciones que apuntan a corregir algunas anomalías, que finalmente no eran más que fallas técnicas. Con esto, se logra un estado mucho más predecible, fácil de entender y poco peculiar, aunque aún no haya logrado del todo dotarse de dichas características. No parecen haber afectando negativamente a la comodidad de los temtiteros con ellas, sino al contrario, y aunque pocos se percaten, de esta forma ellos se comienzan a alejar de una de sus raíces más representativas.

Pero ahora si, con lo que más hace pensar sobre el presente. \emph{///Temti vive, la lucha sigue///}: eso es lo que se dice, y nadie puede afirmar lo contrario, solo que sus condiciones son muy relativas. Objetivamente esta no se puede describir de ningún modo, pero desde el silencio y también escandalosamente, realmente sigue.

Sencillo es desorientarse en esta lucha, pues los propósitos o los objetivos de la misma no están explícitos en ningún lado. La que pueda estar viviendo A. puede ser totalmente diferente a la que esté enfrentando un temtitero común y corriente, al punto de que no suena descabellado pensar que cada individuo está desarrollando su propia lucha, solo que decírselo a un anon que pueda estar comprometido con la misma puede derivar en la negación, se supone que esta tendría que ser colectiva y que unos se ayuden con los otros, pero en ese sentido la comunicación y la colaboración no está siendo muy buena.

Una vez quedó claro cuál iba a ser el nuevo sitio de mayor concentración temtitera, las costumbres típicas que predominaban en la actualidad de su identidad no tardaron en aparecer, como el constante ida y vuelta en los romances temtianos, la promoción de la ideología comunista y los ya mencionados comportamientos de robot que se trasladaron con bastante exactitud, destacando las vigentes exportaciones compulsivas de las tierras rozzadas, así como también las repetitivas respuestas que salen de la boca de los anones una y otra vez sin importar la situación. Pero acompañados de ellos nació una nueva costumbre, o tal vez resurgió de una forma diferente un viejo propósito, y por eso se volvió reconocido el flamante nuevo patrón estelar de esta época especial, el de reproducir antiguos tems que fueron creados y posteriormente destruidos en el pasado, para traerlos devuelta a la vida en el Club. Recuerda a algo similar a lo que ocurrió históricamente luego de cada comienzo de cero que tuvo que tener Temti, pero esta vez es de una forma mucho más primitiva y poco innovadora que en esos entonces, pues nada se promueve el surgimiento de nuevos elementos. Pese a que esta secuencia solo sea repetida por una ínfima cantidad de anones, debido a su constancia y numerosidad, se ha convertido en uno de los fenómenos más relevantes en cuanto a actualidad se refiere.

Y eso también vuelve a confirmar que los más comprometidos en prolongar la existencia de la identidad temtiana, son los anones que se comportan como robots y que posiblemente lo sean, ellos no dan respiro en su lucha y siguen repitiendo sus patrones sin cesar, muy lejos de dar señales de futuras rendiciones. Para bien o para mal, lograron que las últimas fechas de Temti hayan tenido algo más que solo estatismo, y también están provocando que dicho nombre vuelva a estar presente fuera de fronteras, más que antes.

Siendo pocos o muchos, todo ese batallar por hacerse presente está dando sus resultados: el dominio ya es evidente, tanto los locales del Club como las costumbres voxxeras, están siendo opacados por la insistencia y participación temtitera, por bastante diferencia.
No se trata de una conquista agresiva, de hecho no han habido grandes conflictos sociales como para decir que hubieron hostilidades, por lo que ambos, así como también los pueblos o corrientes que no estén comprendidas en esos dos grandes grupos, están sabiendo coexistir, solo que la superioridad en cuanto a orígenes de la actividad es clara.
Puede ser una buena oportunidad para expandir lo que quede de cultura temtiana, ya que varias poblaciones ajenas están permaneciendo expuestas a lo que se repite día a día, y claro está que aunque no sea una conquista radical como la palabra lo impone, el nombre Temti hará eco en sus cabezas, lo quieran o no.

Pero a todo esto, también las hay de aquellas que no están siendo continuas, y no solamente la de John Cena junto a los temtiteros, sino muchas más. Dejando de lado todas las que en ellos se han entregado, quizá la lucha originaria que dio tanto para hablar, la de los supuestos problemas que siguen sin solucionarse, ahora es una de las que más quedó en el olvido. No parece estar perdida, y seguramente no está ganada. Tal vez sus combatientes tan solo estén durmiendo una siesta, de la que quién sabe cuándo despertarán.

\hypertarget{disputas-territoriales-quinto-capuxedtulo-subcapuxedtulo-v}{%
\section{¿Disputas territoriales? (Quinto capítulo, subcapítulo V)}\label{disputas-territoriales-quinto-capuxedtulo-subcapuxedtulo-v}}

\begin{quote}
18 de mayo de 2021, \ldots, 20 de mayo de 2021
\end{quote}

Paralelamente al mismo desarrollo de todo lo que venía ocurriendo hasta hace poco, fueron apareciendo algunos cambios y así la conquista temtiana empezó a quedar levemente bajo dudas.

Los anones conocidos y llamados como amarillos en algunas de sus vertientes más radicales, y otras corrientes diferentes de menor reconocimiento volvieron a aparecer para mostrar fuertemente su presencia en este territorio, y de tal forma causaron impresiones diferentes a la de los previos días. Más allá de que las primeras estén públicamente asociadas a la identidad temtiana en general, en los últimos tiempos salvo algunas excepciones muy recónditas, no habían tenido grandes vinculaciones, marcando así una tendencia a alejarse de dichas corrientes, hasta ahora. Nuevamente, sea desde la interna de la misma o no, los temtiteros están involucrados con las andanzas de las peligrosas agrupaciones difusoras de contenido prohibido por casi todos los sitios del contexto clónico.

De esta forma, también, el interior del Club está dejando de ser únicamente dominado por las habladurías nacionalistas de los temtiteros y sus preponderantes intereses secundarios. Tal vez este aún esté lleno de ellos, o quizás no, pero el chusmerío ya no está yendo únicamente por ese lado.

¿Alcanzarán las baterías de los robots para continuar expandiendo los rasgos de la nación nómada? ¿Algún día estos anones verán reprogramados sus comportamientos para comenzar a enfocarse en cosas realmente nuevas? ¿Lo tan poco de identidad temtiana que queda comenzará a perderse de a poco?

Aunque parte de lo único que los diferenció por tanto tiempo se pueda llegar a perder, los temtiteros tendrían que ser capaces de prosperar en un futuro sin hablar de Temti, pero históricamente nunca fue así, aparte de que tampoco prosperaron siempre cuando si lo hicieron.
Y cuando sea así, mientras su hogar natal siga inaccesible, ¿qué los distinguirá como temtiteros?

\hypertarget{el-supuesto-mito-de-la-mayor-fama-volviuxe9ndose-real-quinto-capuxedtulo-subcapuxedtulo-vi}{%
\section{El supuesto mito de la mayor fama, volviéndose real (Quinto capítulo, subcapítulo VI)}\label{el-supuesto-mito-de-la-mayor-fama-volviuxe9ndose-real-quinto-capuxedtulo-subcapuxedtulo-vi}}

\begin{quote}
20 de mayo de 2021
\end{quote}

A pesar de que en algún momento hayan estado las intenciones de separarse por completo de lo que tuviera que ver con Rozzed y todo lo que saliera de ella, con el tiempo junto al mismo destino del antiguo Plan, debido a diferentes factores dicha idea no solo que se fue tornando inviable sino que también poco se volvió a pensar en ella. Su objetivo, en este contexto clónico, sea el momento que sea parece algo muy difícil de lograr, y más imposible es si no hay compromiso alguno con la causa. No se puede decir que se haya intentado revivirla, nada de eso ocurrió últimamente.

Pero ajeno a lo anterior, algo que sorprendió a propios y extraños, es la aparición de un pequeño nicho en el enorme Imperio de Rouzzed, uno dedicado a Temti, o por lo menos que la homenajea en su nombre. Este siempre existió, pero nunca tuvo tal denominación que tanto sorprende, y ahora tampoco lo tiene, solamente fue por unos minutos pues a la brevedad todo volvió a como estaba antes. Dicho lugar está dedicado para contenidos muy específicos, más algunos también que rozan lo prohibido, y algo que ocurre es que es un sitio donde se manifiestan anones de una buena diversidad de procedencias, de las cuales entre ellas se ven incluidos muchos antiguos temtiteros, conviviendo con otros asociables a las ya nombradas corrientes amarillas. Todos ellos en este pequeño apartado perteneciente al gran antro de los rozzados, habitualmente conviven sin cometer violaciones a las normas del Imperio, pero cuando se está tan cerca de cruzar el limite, este suele romperse. Tiene relación sí, algo importa, aunque tampoco viene mucho al caso, pero si en algún momento todo eso tuvo el nombre que supo tener, más profundización merece.

¿El motivo de esto? La reputación especifica que tiene Temti, nadie tiene duda de que eso fue lo fundamental. Ahora, ¿realmente pasó algo importante como para que esto ocurriera? ¿Algún gusto o capricho que se dieron los administradores del gran sitio? ¿Fue algo inevitable que no se vio afectado por la actualidad de la realidad temtiana? ¿O viene de la mano con sus vigentes influencias?

Las autoridades rozzadas se declararon no responsables de tal situación, justificándola como una ilusión de algo que realmente no ocurrió ni existió. Sin embargo, hay pruebas materiales al respecto y su negación no pudo limpiar su imagen, además de que las repercusiones para el poco tiempo que perduró la situación fueron muy grandes. Pero mientras no respondan con honestidad lo que ocurrió, nadie sabrá el porqué.

Basándose en los antiguos principios que llevaron a este nombre a su cúspide en muchos sentidos, Temti está permaneciendo muchísimo tiempo siendo su propio enemigo. Pero finalmente, ya sea de una forma paródica o no, pasó a estar inserto dentro de su verdadero enemigo. Posiblemente no haya sido ninguna simbología difícil de entender, realmente lo que queda vigente de Temti son importaciones o exportaciones provenientes del exterior, en este caso Rouzzed, costumbres de las corrientes amarillas, algo de actividad historiadora, y poco más. Gracias a este antecedente, lo primero se pudo ver con muchísima claridad.

\hypertarget{reivindicando-la-pulseada-de-la-verdad-quinto-capuxedtulo-subcapuxedtulo-vii}{%
\section{Reivindicando la pulseada de la verdad (Quinto capítulo, subcapítulo VII)}\label{reivindicando-la-pulseada-de-la-verdad-quinto-capuxedtulo-subcapuxedtulo-vii}}

\begin{quote}
20 de mayo de 2021, 21 de mayo de 2021
\end{quote}

La historia temtiana de momento no dejó de desarrollarse principalmente donde lo estaba haciendo, pero vale recordar que los ojos que la siguen de cerca continúan puestos mayoritariamente en los mismos dominios, y en este momento eso está pagando gratamente.

La inestabilidad y el panorama variante pero a la vez aburrido pronto en el tiempo fue totalmente opacado por el radical cambio en los registros del Club. Aparentemente, a causa de un pedido de un temtitero que fue cumplido, el Club de Voxxed dejó de ser tal, para pasar a ser el Club de Temti. Por supuesto que entrando al lugar la impresión es un poco extraña, pero en este momento lo que valen son las palabras y no los recuerdos.

Con seguridad se podría decir que tal determinación tomada por el gobernante Mr.~X fue a raíz de la incesante cantidad de proclamaciones hechas por los temtiteros, que desde el cierre de su sitio originario, no han dejado de marcar huella en su nuevo alojamiento.
Tal es el enorme porcentaje de voxxs que no se escapan de las palabras más escuchadas en estas semanas, que es bastante más justo decir en cuanto a la gente que lo mantiene con viva, que el Club pertenece a Temti, y no a Voxxed. Eso ahora es realidad.

¿El administrador cedió? ¿Se convenció de que es así como ellos dicen?

Sea como sea, esto promete que habrá más anones contentos, pero claro está que también habrá de los otros. Si en todo este tiempo poco y nada se vio de la identidad de los voxxeros, de lo que haya podido ser rescatado de la difunta Voxed, o de cualquier otros, seguramente los infelices sean minoría. Aunque habrá que ver, el mapa las últimas fechas estuvo cambiando un poco. Por eso, no está mal decir que el pueblo ha hablado, y el mandamás ha escuchado y considerado, pero, ¿quién es el pueblo hoy día? ¿Y qué les debe su supremo a ellos?

Al respecto de la segunda pregunta, Mr.~X en estos últimos tiempos ha estado muy presente moderando su antro y parece ser el único que lo hace, y si bien ha tenido muy buena atención con los anones que le presentan dudas y parece deberse a ellos, poco ha dicho sobre su aceptación a las sobras que absorbió o \emph{//heredó//} de la desaparecida Temti, y tampoco sobre su relación con el amo y señor de dicho sitio. Y si se ahonda sobre eso, aunque no sea la gran interrogante de esta novedad, es ineludible suscitar algo que ha inquietado mucho constantemente y que mientras más tiempo los temtiteros acumulan aquí más se fortalece toda esta cuestión, la de qué tiene que ver ese desconocido individuo con Temti.
Siempre ha estado la duda, antes no tanto, y ahora mucho más pensando que se está en la nueva Temti, sobre si Mr.~X es A.. Aunque suene contradictorio, la olvidada ausencia y desconocida situación de este segundo está haciendo ruido, pocos han contemplado que este pueda estar en el Club y nadie sabe si realmente es así, y por otro lado, también algunos anones pudieron notar por varias similitudes difíciles de percibir en los comportamientos de ambos y la forma de comandar sus sitios, que posiblemente sean el mismo individuo con diferentes identidades, lo cual ya fue manifestado en varias ocasiones, pero nunca respondido.
La situación aparenta ser un poco diferente a la que ocurrió con \_Iuri tiempo atrás, en este caso hay más evidencia a favor. Pese a que nadie confirme que esto es así, casi imposible sería comprobar lo contrario, pareciera que ninguno de los dos lo va a hacer, así que solo quedará prestar atención y tomar con pinzas sus palabras.

Dentro de algunas cabezas puede caber que el hecho principal de las últimas horas no va a permanecer así por mucho, sacándolo por contexto: porque podría estar yendo en contra del propósito inicial de la existencia de este lugar, porque no tiene sentido, porque carece de seriedad y muestra debilidad, o porque parece un chiste. Solo que si de verdad se busca encontrar respuestas en el contexto, se va a encontrar más de lo que ya se vio tanto.
Tal vez no evolucione de la misma forma, pero si al Club lo hacen sus integrantes y no solo su dueño, hasta ahora esto es lo que es, el Club de Temti.

\hypertarget{la-cauxedda-a-la-dura-lucha-terrenal-quinto-capuxedtulo-subcapuxedtulo-viii}{%
\section{La caída a la dura lucha terrenal (Quinto capítulo, subcapítulo VIII)}\label{la-cauxedda-a-la-dura-lucha-terrenal-quinto-capuxedtulo-subcapuxedtulo-viii}}

\begin{quote}
22 de mayo de 2021, 23 de mayo de 2021
\end{quote}

Los temtiteros celebraron la \emph{//conquista//}, estuvieron felices con ella y revindicaron lo que tanto ansiaron, el vencer en una de sus luchas.
Los individuos molestos con esto fueron pocos, aunque los hubieron, y también se habló de más cosas durante el tiempo que eso fue así, confirmando que el espacio está siendo compartido más equitativamente, pero lo más llamativo fue esto: el Club entero sometido al nombre Temti. Tal como expresan esas palabras, no duró mucho, el Club volvió a tener el nombre de Voxxed poco tiempo después de que dejara de ser así, concretándose con el hecho clave del pedido de un anon, nuevamente.

Por eso se hace fácil predecir cual sería la reacción de los anones, negativa mayoritariamente y positiva en un segundo plano. El panorama está claro, los defensores de la identidad temtiana contra todo el resto, la misma contienda o lucha de antes, ahora con mucha más historia sobre sus hombros, victorias y derrotas parciales, y por eso esta sigue, no hay rendición por ningún lado.
Como ya se ha dicho, el Club parece estar más poblado, con mayor diversidad de anones que antes, y aunque se sigue sin identificar a los voxxeros, ya no solo se habla de Temti, por lo que la batalla constante, que aún no queda claro contra quién es, quizá se desarrolle en un segundo plano. No quita que esta vaya a ser bastante movida: conflictos y disputas sociales parecen avecinarse si se mantiene la misma intensidad respecto a la situación, el ida y vuelta está y el ciclo no aparenta estar cerca de terminarse.

Aunque de la impresión de que un antecedente histórico como la consagración antes vista no se repetirá, esa no es la única manera de vencer, pero en su defecto, los métodos, los caminos y las prácticas llevadas a cabo para mantenerse en combate de parte de los temtiteros se han vuelto repetitivas, poco innovadoras y casi nada transmiten. Tal vez esta lucha gracias a ello incremente su mala fama, en un contexto que de igual manera, poco bueno hay.

Ahora, la misma pregunta de siempre, ¿cuál es el motor que alimenta a los defensores de todo esto, de seguir prolongando la existencia de una extremadamente decadente identidad temtiana?
Históricamente siempre supo haber mucho atrás de Temti, pero en el presente lo que queda de aquello no tiene ni de cerca la misma fuerza, hoy día lo único que se puede asemejar a eso es el nombre en sí, y no mucho más. ¿Qué tanto vale eso actualmente?

La causa parece estar muy vacía, e inevitablemente, con muy pocos comprometidos, muchos menos que antes. Sin embargo, una vez más repetirlo, esta sigue.

\hypertarget{el-fin-impluxedcito-quinto-capuxedtulo-subcapuxedtulo-ix}{%
\section{El fin implícito (Quinto capítulo, subcapítulo IX)}\label{el-fin-impluxedcito-quinto-capuxedtulo-subcapuxedtulo-ix}}

\begin{quote}
24 de mayo de 2021, \ldots, 28 de mayo de 2021
\end{quote}

Tan pocos cambios hacen que haya muy poco para decir, la realidad del Club fue siguiendo la misma tendencia que tenía y de esa manera la importancia de lo que ocurriera ahí.

No obstante, seguramente con más razón nada de eso sea recordado, dicho sitio cerró sus puertas sin previo aviso y con ello la reconocida concentración de temtiteros identificados se volvió a dispersar.

Al parecer, según lo que se está hablando en el medio, el repentino cierre se debe a una gran coincidencia de sucesos, pero más que nada estuvo provocado por dos factores fundamentales: la cantidad de contenidos prohibidos que empezaron a desestabilizar seriamente el sitio, y los desafortunados sucesos que ocurrieron en el Imperio de Rouzzed y que repercutieron indirectamente en el Club.

Es de público conocimiento que el antro rozzado tuvo grandes conflictos con fuerzas extranjeras de otro contexto totalmente diferente el clónico, que desde hace tiempo vienen ingresando cada tanto al mismo y que en una nueva ocasión volvieron a generar grandes revueltos, esta vez en la estabilidad de Rouzzed. No obstante, debido a una particular y efectiva estrategia tomada por sus autoridades, viene superando dicha adversidad con mucha facilidad, ya que gracias a ella la frontera pasó a estar muchísimo menos expuesta y vulnerable que antes, y de esa forma se libró de un gran porcentaje de las amenazas, al menos a corto plazo. Sin embargo, dentro de dicha táctica, también se incluyó un punto que terminó influyendo seriamente en la realidad de un segundo sitio, y esto fue gracias a una especie de indicación, que repercutió guiando a parte de dichas fuerzas exteriores, directamente a este, al Club.

Así no todo el peso de los inconvenientes cayeron en allí, pero si derivó en otros, porque resultó catastrófico sumado a la fragilidad que tenía y los infortunios que ya venía padeciendo.
Todo venía desarrollándose con normalidad en el de acuerdo a lo que ya se venía viendo, hasta incluso en el día anterior había implementado una interesante novedad a su estructura, una similar al conocido Contador de gordos de Temti. Pero esa supuesta normalidad, fue interrumpida.

Sin haber quedado claro que tan importante fue cada componente, y si la determinación final ocurrió a modo de autoprotección o porque realmente cualquiera de los problemas del exterior se trasladaron oficialmente a los interiores del Club, lo cierto es que de forma paralela a todo eso, Voxxed cerró sus puertas desapareciendo todas sus estructuras y el contexto clónico entendió que cayó definitivamente.

Todo apunta a que no fue una maniobra únicamente destinada a alejarse de posibles invasores indeseados sino que también tenía la posiblemente maléfica intención de desviar el ojo de la tormenta hacia una ubicación limítrofe. Aunque\ldots{} ¿por qué lo harían?
Muchas veces se ha hablado de Voxxed como un refugio alternativo ante posibles percances que pueda llegar a tener el gran antro que se lleva todas las luces permanentemente, y así ofició en múltiples ocasiones, pero al parecer no estuvo a la altura de las circunstancias, pues no logró superar una no muy grande adversidad, no en el sentido de su supervivencia.

¿O quizás nada de esto estuvo en los planes y fue nada más que una consecuencia no deseada dentro de la sorprendente táctica?
Parece un pensamiento bastante ingenuo, pero no se lo puede descartar, los responsables no se van a pronunciar al respecto, solo queda suponer y para ello hay que contemplar todas las variantes. Y sin ir más lejos, nadie sabe de como puedan ser las relaciones de Mr.~X con los demás poderosos del contexto, pero aunque esto pueda haber sentado la base de una gran enemistad, parece que ese mandamás no volverá a aparecer ya que algunos individuos confirmaron haberlo visto huir y escapar a una tierra vecina y dar sus últimas palabras, al menos bajo ese seudónimo como todos lo conocen.
Respecto al resto de jerarcas, no se sabe de mayores disputas o conflictos.

Habrá que prestar atención a los próximos movimientos, ya que últimamente Rouzzed no está manteniendo vínculos del todo neutrales. No es ni de cerca la misma hostilidad que se exhalaba en las Guerras Clónicas, sobre todo teniendo en cuenta que no hay competencia ni se persigue el mismo objetivo, pero la realidad es que hay un enorme desequilibrio de poder y como siempre este puede salpicar al resto.

De cualquier forma, ya no habrá una sitio de escape fijo, pues todos los disponibles de momento tienen condiciones indeseadas que hacen que estén lejos de convencer, sobre todo los rozzados.
Y referido a tal asunto, vale notar lo siguiente: antes de conocer dicha consecuencia que dejara a muchos individuos desamparados, entre los anones se había comenzado a difundir una noticia altamente llamativa, el retorno de Tzzubit a la actividad. El antiguo sitio tras tanto tiempo inactivo, volvió a permitir la entrada al público común y corriente.

Si bien la situación podría haber evolucionado con los meses, y también el sitio presenta muchas condiciones que no comparte con el resto de los del contexto clónico, siempre de igual forma estuvo inserto en el y se sabe que los temtiteros fueron cercanos a dicho lugar, por lo tanto es probable que varios de los que estaban en el Club, así como también muchos que abandonaron el seguimiento de los pasos de Temti, se hayan trasladado allí. Incluso, para sorpresa de propios y extraños, este es el lugar donde el mismo Mr.~X dio su última señal de existencia declarando la intención de desaparecer del mapa. Sin embargo, casi ninguno ha hablado de Temti, y parece que eso no va a pasar.

De este modo, ya no hay ni Temti, ni nueva Temti, ni se habla de Temti fuera de Temti.
No ha aparecido plataforma o lugar que soporte a los robots que aún hacían posible la supervivencia de dicho nombre, o quizá estos dejaron de destinar energías a eso, hoy no se sabe donde se encuentran. Si aquello era lo único que quedaba, está más o menos claro lo que puede evidenciar su ausencia. ¿Qué puede haber durante la total hibernación?

Pequeños destellos ocasionales y descontextualizados van a confirmar que las memorias, el orgullo, y la identidad temtiana siguen vivas y dispersas dentro de su antiguo contexto, pero mientras no haya un espacio apropiado para ellas, van a seguir siendo muy minúsculas e insignificantes. No van a dar mucho para hablar.

\hypertarget{un-paso-muxe1s-cerca-para-descubrir-lo-lejos-que-se-estuxe1-quinto-capuxedtulo-subcapuxedtulo-x}{%
\section{Un paso más cerca para descubrir lo lejos que se está (Quinto capítulo, subcapítulo X)}\label{un-paso-muxe1s-cerca-para-descubrir-lo-lejos-que-se-estuxe1-quinto-capuxedtulo-subcapuxedtulo-x}}

\begin{quote}
29 de mayo de 2021, 30 de mayo de 2021
\end{quote}

Para poner fin al silencio, que no duró mucho realmente, Temti anunció novedades. Aquella olvidada fecha del llamado Día E, está resultando ser la señalada donde supuestamente se le dará apertura nuevamente al antiguo y aún vivo sitio. La ubicación conocida del mismo lo único que contiene, o lo único que muestra y permite ver, es un pequeño nicho casi idéntico a uno antes visto, pero con dicha cuenta regresiva que como ya se dijo, sigue la misma cronología de la ya conocida y pendiente por terminar.

No obstante, ya no está llamada al igual que antes, aunque nada contradiga que siga siendo el Día E, ahora refiere a un \emph{///Reingreso en orbita///}, lo que invita a pensar y da alguna pista de lo que pueda haber pasado todo este tiempo y lo que pasará de ahora en más.

Además, al igual que en la entrada que se vio durante los últimos días de Temti como se la conoce, se encuentran diez pinturas dispersas por el mismo, pero no son las mismas que estaban en aquellas ocasiones, y suponiendo que el único que estuvo allí dentro durante todo este tiempo fue A., estas no serían al azar ni producto de factores ajenos a el, sino que puestas intencionalmente por su cuenta.
Y por último para notar, dentro del recoveco que oficia de \emph{//sala de espera//}, también hay una pegadiza melodía que ambienta el lugar, una íntegramente vinculada a una figura ya conocida en la identidad temtiana y que justo aparece entre las diez mencionadas, tal vez para dejar un sencillo mensaje y anticipar lo que va a ocurrir durante la larga espera, o tan solo para contrarrestar el silencio, o ambas.

A diferencia de todos los episodios anteriores de esta misma escena, de tener enigmas o misterios sin aclarar, y no tener la posibilidad de establecer la sede de investigación en el lugar de los hechos, en esta ocasión la iniciativa por avanzar en la causa no progresó en lo absoluto. Al parecer, ya no hay individuos dispuestos a ni interés suficiente en intentar sacarle todo el jugo posible a las señales de A., o quizá se interpreta que no hay nada más para analizar y que dicha fecha es simplemente el retorno de Temti. Y no llama la atención, ya se sabe que todas las organizaciones inmersas en la trama desaparecieron y que los lacayos temtiteros continúan estando muy cerca de ser cero, solo algunos entusiastas incondicionales sostienen que a estos extraños movimientos se les siga prestando atención.

Lo que si se puede decir al respecto, es que las pinturas mencionadas no son nada del otro mundo, prácticamente todas fueron antes vistas incluso en las ubicaciones más cercanas en el contexto clónico, pero quizá escondan alguna secuencia en clave, hagan referencia a ciertos individuos o elementos, y nadie se está percatando puntualmente que quieren decir.

Por otro lado, dentro de lo poco que se habló sobre el asunto en los sitios vecinos, fue que si se recordó algo de aquella misteriosa sucesión de imágenes en la que no se volvió a avanzar, y es de un par de elementos que en su entonces no terminaban de cerrar a que referían, esos eran un botón rojo y un cohete al espacio. ¿Alguien accidental o intencionalmente presionó el botón y provocó que Temti despegara cual cohete, produciendo así los supuestos problemas que la alejaron de sus coordenadas habituales?

Y bien hace cuestionar también que representa ese Reingreso en orbita, porque no necesariamente tiene que significar que cuando eso ocurra el sitio original vuelva a ser accesible. Tal vez solo se acerque un poquito más\ldots{} Y, ¿en otras condiciones?

No obstante, antes de que eso pase, salvo alguna sorpresa, seguramente el silencio se siga prolongando, y este gran anuncio no sea más que otra pequeña pausa pero relevante en esta extensa hibernación temtiana, ya que poco cambia la nueva realidad posterior al cierre del Club.
Y ahora que se lo menciona, ¿de verdad habrá alguna mente maestra moviendo los hilos atrás de todo esto? No solo se ha pensado que A. pueda ser Mr.~X, también, y en gran mayor medida, hace mucho que se viene hablando de que el gran artífice que comanda muchos de los sitios del contexto clónico es un mismo individuo, y ver que las novedades de Temti suceden al cierre de Voxxed y a la reapertura de Tzzubit, no hacen más que suscitar esa interesante teoría, la cual a esta altura ya no es para nada alocada. Como nada va a permanecer estático, el tiempo seguirá tirando señales que permitan sumar datos al asunto, aunque habrá que ver que tanto afecten a la historia temtiana, la cual parece que gracias a Temti en si no se extenderá mucho, provisionalmente.

\hypertarget{sensaciones-de-descanso-eterno-quinto-capuxedtulo-subcapuxedtulo-xi}{%
\section{Sensaciones de descanso eterno (Quinto capítulo, subcapítulo XI)}\label{sensaciones-de-descanso-eterno-quinto-capuxedtulo-subcapuxedtulo-xi}}

\begin{quote}
31 de mayo de 2021, \ldots, 13 de agosto de 2021
\end{quote}

El silencio es estrepitosamente fuerte, cientos de veces más intenso y prolongado que aquellas veces cuando pasaban horas sin un solo movimiento. ¿Qué podría haber para comentar en un momento como este, si en tanto tiempo no pasó nada?
Las particularidades de esta extensa circunstancia tan especial en lo variada y diversa que es la historia temtiana, ahora y en el futuro la van a situar por encima de muchas otras que quedaron grabadas en la misma sin pena ni gloria. Se trata de un antecedente sin precedentes, donde además de destacar el estado de Temti en sí, también resalta la rotunda ausencia de menciones en el contexto clónico y el total desinterés. Claramente la distancia condiciona, pero no siempre se va a estar tan lejos, y ahí la esperanza de que en algún momento la situación de un giro por más pequeño que sea.

Porque tarde o temprano las ocho frías cifras cambiantes del luminoso contador se convertirán en algo que inspire más cercanía, y porque el resto de los factores que condicionan a los anones no van a permanecer invariables, por eso vale cuestionarse qué tanto pueda llegar a cambiar el panorama desde el inicio hasta el fin de la espera planteada.
Tal vez la abstinencia llegado un punto logre tener un efecto sorprendente, atrayendo a pocos o muchos desencantados y prendiéndolos instante a instante con la espera, o tal vez el Reingreso en orbita no logre generar nada antes de su acontecer.
Puntos intermedios al respecto o alguna notable inclinación hacia alguno de los dos extremos\ldots{} todo se puede resumir, la pregunta que más intriga antes de saber que pueda venir después es si podrá el silencio llegar a romperse antes del día señalado.

\hypertarget{buscando-esa-dulce-compauxf1uxeda-quinto-capuxedtulo-subcapuxedtulo-xii}{%
\section{Buscando esa dulce compañía (Quinto capítulo, subcapítulo XII)}\label{buscando-esa-dulce-compauxf1uxeda-quinto-capuxedtulo-subcapuxedtulo-xii}}

\begin{quote}
13 de agosto de 2021, \ldots, 1 de setiembre de 2021
\end{quote}

El caos y la inestabilidad de los antros rozzados una vez más repercuten a Temti, incluso cuando no está activa o accesible como siempre se la conoció.

Como mucho tiempo atrás supo ocurrir, aunque debido a esta extensa pausa vigente parece que hubiera sido de los hechos más frescos, Rouzzed gracias a peligrosas actividades en su interior atrae las amenazantes miradas del exterior, y ahí comienza la parte más significativa de la cadena donde se vuelve a suscitar el interés por Temti entre ellos, pero\ldots{} estando esta fuera de Orbita.

Dicho punto de partida fue sucedido de una enredada y larga lista de eventos que movilizaron al contexto clónico. Las extrañas mudanzas entre la difunta Rouzzed y la cerrada Dibujazzo, que desembocaron en que la mayoría de los rozzados se estabilizaran en una flamante Rouzzer sumaron mucha incertidumbre al asunto. Pero esto se agitó aún más, cuando se consumó el hecho de que esta última rápidamente después de su nacimiento cerrara sus fronteras, dejando a muchos por fuera sin la posibilidad de ingresar debido a que no poseían las credenciales exigidas en los controles fronterizos.

Por todo esto anterior, muchos fueron a parar temporalmente a diferentes alternativas, entre ellas un nuevo antro que utilizó las viejas estructuras de Rozzed y también su mismo nombre, la ya no tan joven Arggnews que oportunamente obtuvo sus primeros días de fama, y otros territorios algo más alejados del contexto de los clones, puesto que esos dos resultaron ser los únicos con vida y que se adaptaban a sus pretensiones a la fecha. En ese lugar, es que habría estado Temti, gracias a su prestigio y reconocimiento posiblemente por encima de todas las otras opciones, sin embargo quienes no sabían de la situación que condiciona la realidad temtiana, no se encontraron con lo esperado.

Con pasar de los días la efervescencia fue bajando enormemente, sumándole algunos factores que ayudaron a disminuir más aún la presión, hasta que finalmente Rouzzer tiró abajo el muro que prohibía el ingreso de anones que no conservaran sus identificadores, y de esa forma, Temti dejó de ser buscada por los pocos de ellos que lo hacían.
Sin embargo, coincidió en un momento que la cuenta regresiva ya estaba llegando no a su final, pero si acercándose a el, y sin ser algo que se note mucho, discretamente, comenzó a revelarse el interés de algunos temtiteros en tierras rozzadas que tenían su interés dormido, y no se volvió a apagar desde entonces.

Definitivamente aproximarse temporalmente ayudó para poder conectarse mejor con la frase clave de la canción que disimula el silencio, \emph{///You gonna miss me when Im gone\ldots///}. Eso se convirtió en una realidad para unos cuantos anones.

Certezas no hubo casi nunca, no las va a haber ahora que el paradero de A. es totalmente desconocido, ni menos con dicho ser al frente, pero ya mucho no importa eso, las condiciones tampoco, el Reingreso en orbita tendrá sus espectadores: distantes, entusiasmados, curiosos, desinteresados, y un largo etcétera\ldots{} pero los tendrá.

Ahora pronto será turno de la Orbita, esta debería enseñar qué es lo que va a ingresar.

\hypertarget{en-la-atmuxf3sfera-de-la-misma-orbita-quinto-capuxedtulo-subcapuxedtulo-xiii}{%
\section{En la atmósfera de la misma Orbita (Quinto capítulo, subcapítulo XIII)}\label{en-la-atmuxf3sfera-de-la-misma-orbita-quinto-capuxedtulo-subcapuxedtulo-xiii}}

\begin{quote}
1 de setiembre de 2021
\end{quote}

Entrando en la recta final\ldots{} nada salió de lo esperado.
Los anones, de distintas procedencias, en su mayoría provenientes de Rouzzer, están instalados en territorio temtiano para observar como continúa esta especial parte de una muy dilatada historia.

En ellos la incertidumbre no es tanta realmente, la explicación literalmente no existe ni en este momento ni desde hace mucho tiempo, pero la curiosidad sí que está, incluso para los individuos ajenos a todo esto. También se puede encontrar intriga, impaciencia, ansiedad, y más escalando en esos términos. Definitivamente hay expectativas, pero expectativas no tanto de Reingreso en orbita como tal, sino más bien de impacto o aterrizaje, las cuales están bastante justificadas.

Lo que Temti genera, propulsado por lo que una cuenta regresiva llegando a su fin después de un largo camino promueve, siendo recibido por una población que malinterpreta y que sin exagerar tiende a prender las alarmas por las situaciones más insignificantes, esa es la bomba de tiempo que se fabricó y que se está llevando la mayor parte de las miradas en el Imperio rozzado, la cual en instantes, estará estallando.

\hypertarget{la-uxe9poca-de-las-infinitas-pruxf3rrogas-disfuncionales-sexto-capuxedtulo}{%
\chapter{La época de las infinitas prórrogas disfuncionales (Sexto capítulo)}\label{la-uxe9poca-de-las-infinitas-pruxf3rrogas-disfuncionales-sexto-capuxedtulo}}

Con la leyenda de \emph{///THE RIDE NEVER ENDS///} más presente que nunca, el mayor enlentecimiento jamás visto antes se redimensiona y arranca a darle un sentido bastante diferente a uno de los fundamentos que acompañan a la realidad temtiana de todas formas y colores.

La llegada de la anunciada fecha del Reingreso en orbita, haya ocurrido algo oficialmente o no, ineludiblemente marca un antes y un después, el cual hará que aquello que nunca termina pase a ser primordialmente la espera, que cambiará la reputación del antro irreversiblemente, que hace que el recorrido de la simbólica caída libre sea mucho más amplio aún, entre otras cosas, incluyendo las que no se saben con exactitud.

Mientras tanto, el Segundo Éxodo tampoco llegó a su fin, y lo que reste de identidad temtiana tendrá que seguir moviéndose por otros sitios, atada de pies y manos a la realidad que le impide a Temti acelerar sus pasos\ldots{} según su pueblo siempre supo interpretar, una de las tantas luchas donde se perdió abundante terreno, la lucha contra la indiferencia.

Pese a tenerla imponiéndose, rigiendo en gran y casi máximo esplendor, no deja de haber vida, ni aunque sea en su mínima expresión, y por muy irrisorio que sea el grupo que sigue conformando la diáspora temtiana, su resistencia persiste, no hay error que los ahuyente ni demora que los desvíe de su gran anhelo, ese que se cree algún día se cumplirá: regresar a Temti y habitarla cuando esta esté pronta para poder hacerlo.

\hypertarget{el-histuxf3rico-fracaso-de-los-seres-terrenales-sexto-capuxedtulo-subcapuxedtulo-i}{%
\section{El histórico fracaso de los seres terrenales (Sexto capítulo, subcapítulo I)}\label{el-histuxf3rico-fracaso-de-los-seres-terrenales-sexto-capuxedtulo-subcapuxedtulo-i}}

\begin{quote}
1 de setiembre de 2021
\end{quote}

¿Y?

Se supone que llegó el Día E, el día del Reingreso en orbita, pero la gran expectativa de la multitud no se ha cumplido, nadie esperaba fallar al \emph{//Comenzar!//}.
Así es, el resultado de intentarlo es obtener un mensaje de error de carácter técnico, el que índica que algo no ha salido bien, lo que en jerga rozzada sería chocar, el único impacto que al parecer se cumplió y se repite cada vez que se insiste.

Confiando en la seriedad del antro de las esperas eternas, la cual tal vez perezca una vez estas acaban, tiene sentido creer en que nadie conoce bajo qué calendario se rigen sus fechas y sus horas, y que al parecer la mayoría se desentendieron respecto a eso, que Temti no sigue el mismo que ellos, que lo que alguna vez iba a llegar era el \emph{//Día del Error 405//}, que el Reingreso en orbita ocurrió en una distante a la de los espectadores, que fue momento de algún fenómeno sobre natural, cuántico, o espacio-temporal, o que los que están mal son quienes tratan de ingresar y no el antro en sí, o cualquier otra excusa. Siempre de alguna forma se puede justificar, pero a esta altura no parece ser el camino rebuscado correcto.

Pero sí, no es como que nada haya ocurrido, la bomba de tiempo sí estalló, y todo el Imperio de Rouzzer se hizo de la noticia de que la cuenta regresiva fue relativamente en vano, para nada más que ridiculizar a Temti y/o para cometer un canallesco \emph{//baiteo//}. A todo esto viene la gran pregunta de siempre que podría no valer la pena repetir, ¿por qué?

¿Esto estaba en los planes desde aquel momento en que la cuenta regresiva empezó? ¿Qué tanto se desvirtuó el destino de todo esto con el tiempo?

¿A.?
Lo predecible es que este se encontrara desde las alturas observando y moviendo los hilos, como siempre supo hacer, pero otro matiz que resultó contra pronostico es que parece que no estuviera, y que su última aproximación a su creación hubiera sido cerca de cuando el Club de Voxxed murió, puesto que realmente nada ha cambiado desde ese entonces.
La cuenta regresiva desde el minuto cero estaba programada para desaparecer y dar lugar al enlace que todos pensaron permitiría acceder o al menos acercarse a aquello que iba a protagonizar el Reingreso en orbita, y a la vista está que el \emph{//pase//}\ldots{} ¿no ha sido habilitado?

Algunos anones se han empecinado en que lo que repite la gente son disparates y esto un malentendido, y que si se puede acceder, pero que se está cometiendo un importante error en el proceder. No obstante, solo se menciona que como en los viejos tiempos un código válido es requerido para ingresar, pero ninguno de ellos revela cómo obtenerlos o el verdadero método permitido con detalle, y ninguna realidad legítima de la súper exclusiva Temti se filtró entre la muchedumbre, sus dichos parecen más y más humo del que ya generó este Reingreso en orbita.

Las posibilidades no son tantas: efectivamente resultó ser una vil mentira, volvió a surgir un problema quién sabe de qué tipo, o Temti fue abandonada por su mandamás. La mayoría presente en el evento como era de esperar se inclinaron por las dos primeras opciones, pero de igual forma con especulaciones o sin ellas, sigue siendo extremadamente insólito y muy difícil de lograr entender correctamente. Incluso hasta los temtiteros más experimentados están confundidos y no entienden lo que ocurrió, pero claro, \emph{///la explicación podría ser peor que la incertidumbre///}, y a la vista está la evidencia de que semejante cita en su momento logró tirar la posta. Si quedaba alguna duda, los hechos de público conocimiento vuelven a demostrar que los mortales no suelen estar preparados para la realidad, mucho menos la temtiana.

Ya podría considerarse obvio que será posible regresar atrás en el tiempo y ese tipo de cosas, pero que esto no se podrá cambiar ni disimular, los hechos ya están consumados y ahora sí que sí se está ante algo que difícilmente se vaya a olvidar, seguramente pase a integrar la retina de cada individuo que intente ubicar a Temti con su particular reputación y fama en el contexto clónico. Sin embargo, también es importante aclarar que nada de esto cierra las puertas a posibles cambios en un futuro, de hecho seguir adelante con un antecedente como este podría resultar interesante en varios sentidos, especialmente para los temtiteros como una útil y rigurosa prueba para demostrar su fidelidad. Ni hablar que más tiempo será necesario para esclarecer las turbulentas nieblas que esconden el motivo de este antes y después.

Pero al parecer es muy pronto para lo anterior, puesto que luego de algo de tiempo después del acontecimiento único, solo la conmoción disminuyó un poco, mientras el resto sigue igual.
La Orbita aún se muestra incambiada respecto a lo mismo que era hace unos cuantos meses, A.
aunque pueda estar entre la gente sigue sin dar indicios de vida, y el \emph{///¿Y?///} se sigue escuchando tanto en suelos temtianos como rozzados, más fuerte que la canción When Im Gone, lo cual no es nada desorbitado, todos se preguntan lo mismo.

\hypertarget{ruxe9cord-de-desoriente-sexto-capuxedtulo-subcapuxedtulo-ii}{%
\section{Récord de desoriente (Sexto capítulo, subcapítulo II)}\label{ruxe9cord-de-desoriente-sexto-capuxedtulo-subcapuxedtulo-ii}}

\begin{quote}
1 de setiembre de 2021
\end{quote}

La jornada del Día E no concluyó con el inaudito episodio de las expectativas frustradas y la burla descomunal. Horas pasado tal descalabro, Temti vuelve a confirmar que vive, y que no muy lejos hay una pequeña cuota de esperanza de que después de tanto tiempo vuelva a estar en condiciones mínimamente aceptables.

\emph{///Un solo click///} son las llamativas y prometedoras palabras con las que se recibe a los visitantes en el sitio, ya no hay referencias al Reingreso en orbita. El anon promedio, sobre todo el menos experimentado, pese a los decepcionantes antecedentes, imaginaría que esta es la definitiva y que dirigirse donde las referencias indican que se continúa no podría salir mal.

Estarían cometido un error, y precisamente no el número 405. También parecen estar equivocándose aquellos individuos que efectivamente siguen ese mismo camino que en teoría debería funcionar.
Actualmente el sentido común no es suficiente para encontrar los caminos correctos en Temti, tampoco lo es la experiencia, y tampoco el conocer los inconfirmables mitos que disimulan o al menos intentan acercarse a la verdad de esto, nada estaría alcanzando. Definitivamente los caminos no están revelados, ni son fáciles de revelar, si es que realmente los hay\ldots{}

Además de eso, cabe destacar que el tono que abunda en territorios temtianos ya no es el mismo, el verde de las últimas épocas se ha ido y en su lugar llegó el rojo, prolongando así el extenso historial del sitio vistiendo una enorme diversidad de colores, en una oportunidad más. Sin muchos fundamentos para poder interpretar tal variación, quizá esto haga alusión a un cambio respecto a la proximidad de lo que pueda estar por venir, ya sean futuras novedades, aquello que se supone que ya ingresó en Orbita, o el retorno Temti a su normal funcionamiento. Sea lo que sea, algo ha cambiado, aunque más allá de las apariencias, es complejo percibirlo.

Pero saliendo de la difícilmente descifrable simbología que pueda llegar a haber atrás de ese preciso color o de esa engañosa frase, lo que destaca por sobre medida son las claras evidencias de que una vez más el abandono no es tal y que hubo algún movimiento oficial del antro, por más mínimo que sea. La esperanza de poder vivir junto a Temti en un futuro más deseado regresa a circular en la superficie con un tanto más de razón, sin importarle lo distante que esté de su realidad deseada.

\hypertarget{la-calma-de-la-fruxeda-lejanuxeda-sexto-capuxedtulo-subcapuxedtulo-iii}{%
\section{La calma de la fría lejanía (Sexto capítulo, subcapítulo III)}\label{la-calma-de-la-fruxeda-lejanuxeda-sexto-capuxedtulo-subcapuxedtulo-iii}}

\begin{quote}
1 de setiembre de 2021, \ldots, 6 de setiembre de 2021
\end{quote}

Al demérito de los cambios oficiales en territorio temtiano, la expectativa acumulada no volvió a ser ni la sombra de lo que había generado el Reingreso en orbita.
Como en los tiempos más decadentes, nuevamente los únicos interesados en el presente y el futuro de su antro en los papeles predilecto, son algunos pocos leales que incondicionalmente pase lo que pase regresarán, y poco más de eso.

Sin certezas algunas sobre esta realidad contradictoria, se está volviendo de a poco a una coyuntura similar a la padecida en la prolongada hibernación, con la diferencia de la gran incertidumbre y la prontitud de lo que ocurrió más lo que pueda estar por cambiar.

Eso es lo que al parecer mantiene expectante a la minúscula resistencia, la visible mayoritariamente sí orbitando alrededor de Rouzzer, la de algunos referentes, personajes y bots ya reconocidos en el contexto temtiano, la que es apoyada por el seguimiento de Diario Temtiano y las recopilaciones del archivo oficial del sitio, toda esa que en conjunto o no aún lucha por no cerrar los ojos o desviar la mirada totalmente hacia otro lado, esperando que Temti se acerque un poco más.

¿Y?
¿Hasta cuándo?

\hypertarget{en-vuxedsperas-de-aterrizaje-sexto-capuxedtulo-subcapuxedtulo-iv}{%
\section{En vísperas de aterrizaje (Sexto capítulo, subcapítulo IV)}\label{en-vuxedsperas-de-aterrizaje-sexto-capuxedtulo-subcapuxedtulo-iv}}

\begin{quote}
6 de setiembre de 2021, 7 de setiembre de 2021
\end{quote}

Del rojo al azul oscuro, de Virguero al Papa Francisco, de un gato con sandías a la Argentina con la copa, y de temti a\ldots{} ¿servidor instancia temti?

Si el verde era estar fuera de Orbita, y el rojo estar dentro de ella, ¿qué es ese azul?

Nuevamente ante la misma situación, las incomprendidas pequeñas variaciones en los territorios temtianos, pero al fin y al cabo nada de interés general parece haber cambiado. No obstante, el sitio se está comunicando, y el mensaje está quedando en el aire. Quizá no sea lo único.

Asimismo, dentro de la dura tarea de intentar analizar las pocas y crípticas pistas, no se puede ignorar que haya aparecido ese característico símbolo ya antes visto en el sitio, el cual a partir de cierto entonces siempre representó al \emph{//servidor//} que alojaba a la Instancia principal, en la que siempre se encontraron todos los tems visibles. ¿Este está de regreso? ¿Ya está disponible para o cerca de poder dar lugar a las construcciones que siempre formaron al interior del antro?

Tal como fue adelantado en las épocas previas al Segundo Éxodo, el sistema de Instancias estaba en camino, y entre algunos anones además de las proyecciones de otro final, circuló la idea de que el Día E iba a ser ese lejano momento en el cual llegara dicha particular estructura.
Mucho tiempo después, un elemento de la no tan vieja conocida organización de Temti, conectado con toda esa ambiciosa conformación que contenía a la \emph{//instancia temti//}, vuelve a aparecer, lo que inevitablemente trae aparejada la idea de que todo ese \emph{//cuento//} no haya sido tal y que de verdad se esté acercando una innovadora y explorable reforma.

Gracias a todo lo anterior, pese a que el panorama literalmente se oscureció, el mismo se ve más prometedor, aunque sin quitar que sigue siendo ampliamente impredecible y que el ritmo de esta realidad temtiana todavía está en valores bajísimos.

\hypertarget{colisiuxf3n-de-ilusiones-sexto-capuxedtulo-subcapuxedtulo-v}{%
\section{Colisión de ilusiones (Sexto capítulo, subcapítulo V)}\label{colisiuxf3n-de-ilusiones-sexto-capuxedtulo-subcapuxedtulo-v}}

\begin{quote}
7 de setiembre de 2021
\end{quote}

Los minúsculos avances de las anteriores jornadas fueron proseguidos de algo que le dio bastante más sentido a lo anterior. Temti de regreso en la superficie y disponible para que los mortales puedan acceder en ella, con grandes novedades, transformándose en una inquietante señal que invita a abrir bien los ojos y explorar, donde ya no parece hacer falta escuchar.

Sin intenciones de ironizar, se podría decir que es demasiado bueno para ser verdad, pero efectivamente es tal cual como lo explicita la oración. Las prestaciones que ofrece el antro adoptaron una forma mucho más convincente, donde las ambiciosas palabras del pasado ya no se imaginan cual utopías sino que ahora se pueden palpar y casi que sentir como una realidad, pero aún sin llegar a serlo, porque casi nada de eso funciona.

Pocas veces pulsar botones fue tan en vano, la tecnología está completamente rota, con la excepción de un pequeño recoveco en el cual se pueden generar códigos, llamados \emph{//tokens//}, ilimitadamente, que en este momento no parecen ser útiles en lo absoluto pero que en un futuro quizá lo sean, más con la diluida incógnita de Temti Premium aún misteriosamente escondida. Pero salvo eso, todo el resto es como un simple telón sin nada detrás de sí, un relativamente atractivo muro impenetrable que no permite ir más allá.

Toda la genialidad de las apariencias y del ingenio se contrastan con la ausencia de su complemento más esencial, haciendo que hasta una calavera que podría pasar desapercibida como un elemento temático más, comience a dar una fuerte mala espina como si de la vitalidad del sitio se tratara. Ni la energía cuántica es capaz de disimularlo, parece que es otra de las maravillas que hacían único a este antro que se perdieron en el espacio exterior o en otro rincón de algún universo no muy cercano. Y nada ha llegado a suplantarla o cumplir su rol, todo demuestra ser muy vacío, carente de contenido.

Aunque cueste creerlo y procesarlo, es una falsa materialidad, una vez más. La tendencia se vuelve a confirmar y Temti como tal continúa ausente. ¿O esta vez es diferente? ¿Cuándo se le pondrá fin a esta etapa de seducciones sin retorno?

Solo A. debería saberlo, y en estos momentos no es posible develar si este quiso convidar al público con una pizca de su fantasía de las Instancias no cumplida antes de hacerla realidad por completo, si se trata de otro problema que inhibió a la flamante transformación y que tardará una nueva infinidad en solucionarse, o qué.
Lo que sea, habrá que interpretar, y vale la pena desconfiar, sobre todo en estos últimos tiempos que tanta mitomanía abunda en la distorsionada realidad temtiana, a la que el verdadero temtitero promedio seguramente sepa responder, pues de la mano de sus robots ha aprendido a discrepar de lo que ve, y a la vista está como quienes más recaen en decepciones son los extranjeros.

Y antes de reiterar el error de creer o confiar que Temti no puede continuar permaneciendo así mucho rato más, mejor recordar que para lograr medianamente entender al sitio siempre fue necesario contextualizarse, ubicarse, interiorizarse, etcétera etcétera. Esta historia en su actualidad indica que lentamente pasito a pasito se va, y que no todo lo que se ve resulta ser como aparenta, pero que los cambios, sean los deseados o no, llegan. Así que la paciencia, la calma y la cautela son los mejores remedios para cualquier perspectiva que apunte a suelos temtianos.

\hypertarget{sobre-el-insondable-luxedmite-sexto-capuxedtulo-subcapuxedtulo-vi}{%
\section{Sobre el insondable límite (Sexto capítulo, subcapítulo VI)}\label{sobre-el-insondable-luxedmite-sexto-capuxedtulo-subcapuxedtulo-vi}}

\begin{quote}
7 de setiembre de 2021, 8 de setiembre de 2021, 9 de setiembre de 2021
\end{quote}

Fue un buen rato, el suficiente como para que el selecto reducido público pendiente pudiera encandilarse con la ilusoria nueva estructura de Temti y percatarse de que los tiros seguían yendo por el mismo lado, puro fiasco, pero generador de una ligera porción de optimismo, de que con algunos retoques todo podría enderezarse satisfactoriamente. Los no tan pendientes, algunos motivados a acercarse debido a un nuevo percance en el Imperio de Rouzzer, se toparían con una sorpresa, de la cual seguramente se extrañarían y no mucho más.

También vale notar que mientras todo eso se consumaba, precisamente unos pocos instantes luego de que los cambios salieran a luz oficialmente, cuando recién se estaban enterando unos pocos primeros individuos, ya se estaba realizando uno que otro ajuste, pero que al fin y al cabo ninguno de esos alteraría el chocante resultado final.

A rasgos generales así fue, con poca pena y nada de gloria pasó desapercibido el nuevo estado del sitio, hasta que en un momento al parecer A. le puso punto final a esa incompleta exposición. Pese a que en términos del tiempo corriente en el contexto clónico la gran reacción al silencioso pero a la vez escandaloso cortocircuito demoró bastante, para lo que acostumbra esta nueva realidad temtiana no fue tanto.
De igual forma, no importa, la espera se vuelve a dilatar, solo que esta vez de una manera un poco más decorosa: en una suerte de mantenimiento, Temti imposibilita la exploración de su principal coordenada, haciendo que cualquiera que intente adentrarse más allá en ella sea a modo de bucle regresado una y otra vez a un punto donde su única opción para lograr moverse es retirarse.

Las palabras oficiales usadas para explicar esto logran darle esa incomprobable gran sensación de proximidad y de cercanía, como si verdaderamente no quedara nada para que Temti finalmente o nuevamente aterrice, o para que termine de acomodarse y repararse. Como si este fuera el ideal paso anterior a un nuevo punto de quiebre, uno que volverá a segmentar la historia temtiana, uno que está a la vuelta de la esquina y que es solo cuestión de unos pocos momentos su arribo, parece.

No hay argumentos firmes como para decir que no va a ser así, pero ante reiterados resultados, muy reiterados, una parte del trasfondo va quedando implícitamente evidente. Si una espera tan extensa y puntualizada como la del Reingreso en orbita resultó ser más larga de lo que se esperaba, y no coincidió con la temporalidad que seguían sus espectadores, ¿por qué \emph{//Ya//} que es tan relativa no iría por el mismo camino? La respuesta cae por sí sola.

Aparte, en el Testamento se confirmó que \emph{///Ya no era lo mismo///}, mucho menos lo va a ser ahora. Si ya no es lo mismo, tampoco va a ser lo que parece, pero ¿qué tan diferente?

¿Qué tan transformada está la realidad de Temti de lo que las primeras impresiones visuales manifiestan? ¿Qué tan desvirtuado está el tiempo de la realidad de Temti respecto del que alguna vez supo tener?

Todo es difícil de precisar, todo está fuera de lugar, decir mucho sería poco. Temti dice menos.

\hypertarget{en-continuidad-con-el-desvanecimiento-sexto-capuxedtulo-subcapuxedtulo-vii}{%
\section{En continuidad con el desvanecimiento (Sexto capítulo, subcapítulo VII)}\label{en-continuidad-con-el-desvanecimiento-sexto-capuxedtulo-subcapuxedtulo-vii}}

\begin{quote}
9 de setiembre de 2021, \ldots, 20 de setiembre de 2021
\end{quote}

Aunque sean muy rebuscados, es cierto que se realizaron algunos ajustes, que podrían ser bastante más que detalles. Pero lo más relevante sigue igual, Temti aún dice que va a volver. Y nada.

Profundizando en la ausencia respectiva, tras los primeros días de la corriente nueva espera, no solo dejó de ser posible generar nuevos códigos de acceso, sino que también parte de lo poco que quedaba del nombre del antro terminó de desaparecer de sus territorios, de forma que quien ingrese a Temti no tendría manera de darse cuenta que está dentro de esta si no fuera porque recordara que para llegar allí tuvo que dirigirse a sus explicitas y hoy complejas coordenadas. ¿Eso es lo único que queda?

Por otra parte, fuera del mismo sitio, la confianza en el retorno no está intacta, ha sido muy deteriorada y golpeada como el fiel e incondicional temtitero promedio. Incluso aunque sea evidente que varios aún la conservan y extrañan tal cual cantaba cierta canción, también hay de quienes entregaron y abandonaron su esperanza.
Pero aún no queda claro quién dejó su cráneo en las tierras temtianas\ldots{} En esta circunstancia, pese a que no haya sido manifestado nada al respecto y que decir lo contrario pueda conllevar suscitar una enormidad de cuestiones, ¿Temti aún vive? ¿O lo que vive es la vigencia de su dominio? ¿O la muy escasa identidad aún propagada por sus ínfimos defensores?

Como sea, haya muerto o no, al menos la historia que sus territorios protagonizan no deja de hacerle honor a su nombre, porque hay un gran tema hoy día, la espera, y el tiempo ni hace falta indicar porque. De hecho, es uno de los temas más longevos de toda la historia temtiana, por encima de cualquier otro común y corriente, y aún está por verse hasta cuando.

\hypertarget{lo-uxfanico-que-nunca-cambiuxf3-desenlace}{%
\chapter{Lo único que nunca cambió (Desenlace)}\label{lo-uxfanico-que-nunca-cambiuxf3-desenlace}}

Quizá entre tantas preguntas sin respuesta, haya solamente una capaz de obtener su respuesta concisa, real y fácil de entender.

¿Cuándo termina todo esto?

Si bien entre lo amplio que es el pasado, el presente y el futuro existe el momento en el que Diario Temtiano se estancó y chocó con barreras temporales mientras intentaba seguir con su propósito, luego de ese entonces el camino de la historia temtiana seguirá siendo recorrido eternamente, solo que ya sin su misma compañía. Ni durante la bonanza más prospera, ni cuando se cree que se está viviendo el momento más crítico de todos, nunca la continuidad de la historia temtiana corre peligro de extinción, siempre de alguna forma u otra esta se continuará desarrollando.
Para saber cómo, cuándo, quiénes o qué, y por qué, es necesario estar ahí.

Volviendo a la pregunta, para este caso será necesario serle fiel a unos de los principios temtianos más fuertes y duraderos, y por eso lo más correcto es dejar un final abierto, inconcluso. Da igual el momento y el contexto, esto nunca termina.

Y si después de tanto texto no hay nada que haya quedado claro, no importa, lo fundamental es lo siguiente: no existe punto final.

\end{document}
